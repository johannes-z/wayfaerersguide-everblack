\subsection{Silver Isles}
\label{sec:Silver Isles}

In the centre of the vast sea that separates all continents lies a vast
archipelago called the \emph{Silver Isles}. A large collection of small
islands that stretch out into all directions, and are loosely connected to the
continent of \nameref{sec:Goltir} by a land bridge.

Almost all of the islands either lie on, or very close to the equator of
\hyperref[sec:Aror]{Aror}. Most are thus tropical paradises, with white
beaches, palm trees, clear water and covered with thick and lush
rainforests. This paradise is only a façade however, as most are inhabited by
various tropical beast races, such as medusa, hydras, or nagas, or tribal
humanoids that are fiercely protective of their islands.

As one of the few remaining frontiers, many explorers travel to the silver
isles in the hopes of discovering new land, islands, riches and gold. The
waters however are often treacherous, the island inhabitants dangerous and
many unsavoury groups. Slavers and pirates operate out of bases hidden in
the silver isles, preying on the native population and passing ships alike.

Nevertheless the islands are known for harbouring many exotic plants, fruits,
and herbs, such as tobacco, chocolate and coffee. While the mountains and
volcanoes of the isles are filled with silver, gold, and
\hyperref[sec:Everblack]{everblack}. The trading companies of many large city
kingdoms have outposts and colonies on the isles, and often bitterly rival
over these resources. This brings them in direct conflict with the humanoid
tribes of the isles, called the \emph{Inua} who they often exploit, displace
or even enslave.

\subsubsection{Inua}
\label{sec:Inua}

The Inua are several native humanoid (halfling, dwarven, elves as well as
human) tribes that have lived on the islands of the archipelago since the last
ice age. Inua generally have darker skin, a stout and strong build due to their
reliance on heavy labour, and their own language called \emph{Inue}. They live
in tribes or small towns, and are excellent hunters, gatherers, and boat
builders. They often travel around the archipelago to hunt, fish, or to raid
other Inua tribes that dot the islands.

They often wear minimal clothing made out of fur and hide, and use crude and
primitive weaponry made out of stone, bone and wood. Very few Inua tribes have
mastered weapons out of metal, such as iron or even steel. Their appearance is
often frightening, as they enjoy piercing and tattooing themselves, especially
with white or other bright colours.

They are staunch believers of an entity they call \nameref{sec:Isamir}.
He tells them to honour the sea, the lakes, and all things that dwell within
them. He is said to conjure storms to cleanse islands, but also to grant
smooth weather so that the Inua can fish. He demands sacrifices of his people,
and they often sacrifice large catches they make, and sometimes even go so
far to sacrifice members of other tribes or foreigners to him. In recent years
however the worship of \nameref{sec:Isamir} has been challenged among the
Inua. Worship of \hyperref[sec:Forun]{Elora} (the ``lady of fire'' in Inua)
has become widespread, as she grants divine power, warmth and does not require
living sacrifices to be made to her.

Still the Inua are known for building secret temples in the jungles of their
islands. These temples are holy places, where no foreigner should treat, and
thus they are often guarded by mummified undead that the shamans of the Inua
conjure and create. These undead were once strong and proud warriors that
now fulfil the last great honour of the Inua: defend the holy sites against
intruders for all eternity. The Inua are expert embalmers, and their shamans
are renowned necromancers.

Although proud and strong warriors, the Inua are not as developed in terms of
technology as the rest of the humanoid that come to the silver isles.  Even
though they more primitive than the humanoid races of the city kingdoms, their
fierceness in battle, expert skill in stealthily attacking with small
manoeuvrable boats, unparalleled knowledge of the islands, and their
necromancer shamans allowed them to resist many invasions and attacks. Some
tribes have suffered defeat by the hands of the expeditions that came to the
isles, and have thus become wary, secretive or outright xenophobic. Some
tribes openly trade with the foreigners to their islands, in hopes of
appeasing them, while others are openly hostile and will scare off or kill
anyone that dares to set foot on their islands.

The Inua are in a constant battle against the foreigners that come to their
islands in the hopes of exploiting their resources. They openly wage sea
war against slavers, pirates and ships of trading corporations that seek
to displace them to gain access to their wealth.

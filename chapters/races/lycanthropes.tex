\subsection{Lycanthropes}
\label{sec:Lycanthropes}

Lycanthropes, or were-creatures, are monstrous or humanoids that turn into an
animal shape. Either voluntarily at will, or against their will during a full
moon. As far as modern scholars agree, there are two main types of
lycanthropes: the original, true lycanthropes, and those that
\hyperref[sec:Three Kings]{Miator} has blessed with the gift.

\subsubsection{True Lycanthropes}
\label{sec:True Lycanthropes}

\songquote{Garmarna}{
  Kära du ulver bit inte mig \\
  Linden darrar i lunden \\
  Dig vill jag giva min silversko \\
  Ty hon var vid älskogen bunden \\
  Silversko jag passar ej på \\
  Linden darrar i lunden \\
  Ditt unga liv och blod måst gå \\
  Ty hon var vid älskogen bunden
}

True lycanthropes are immortal (but not invulnerable), and were once cursed by
the \hyperref[sec:Druid]{Druids} for unspeakable crimes against nature and its
inhabitants. The druids cursed them with seeing the world through the very
animals that they had harmed, and are condemned to bring the same harm upon
his very own species. They are forced to turn into were creatures (or half
humanoid, half animal creatures) when one of the moons of Aror are full - so
twice a month, often during the beginning and during the middle of the
month. They retain full awareness and memory of their actions during their
transformation, but are most often unable to control their behaviour. The
curse often cannot be broken through divine magic, except when a horrible or
sadistic condition is met, or when it is lifted by the druid that spoke the
curse.

Since most true lycanthropes are fully aware of their condition, and the
requirements required for their release, they retreat from society to live as
hermits and they often shackle themselves to not hurt others. Those that try
to suppress their urges, often find themselves at odds with society, as their
immortality isolates them from the daily lives of the mortal races. Their
immortality also means that more often than not, the druid that cursed them or
the people they were meant to kill to release the curse, have died a long time
ago, leaving them trapped with the curse.

Some lycanthropes give in to their temptations, and thus also attempt to
fulfil the conditions that would release them from their curse. These
conditions often require sacrifice of other living beings to be broken, making
these lycanthropes a threat to societies and villages.

\subsubsection{Beast Warriors of Miator}
\label{sec:Beast Warriors}

Miator, the god of slaughter and destruction of the \nameref{sec:Three Kings},
also bestows lycanthropy upon his favoured warriors. This state of lycanthropy
is different from \emph{true lycanthropy}, as the warrior retains full control
during the transformation, giving them unspeakable strength and combat prowess.
These \emph{beast warriors}, do not require a full moon to change, and may
change back and forth at will. Unlike true lycanthropes, they are not immortal.

\begin{35e}{Lycanthropes}
  \emph{True lycanthropes} change forcibly into a animal, or hybrid form
  during a full moon. So twice a month, once at the start, once during the
  middle of the month. They retain memories of their time during the
  change, but cannot control the creature they have become, which is always
  \emph{chaotic evil}. They can attempt to stop a single action their animal
  form attempts to make by succeed a \emph{DC: 15 + animal form creature's HD}
  will save. True lycanthropy is a curse, but can only be broken through
  \emph{Wish}, or by fulfilling the condition that the caster of the curse set
  during the ritual. A true lycanthrope is aware of his conditions, and also
  about the condition that must be fulfilled to break the curse.

  \emph{Beast Warriors} behave just like SRD lycanthropes
\end{35e}

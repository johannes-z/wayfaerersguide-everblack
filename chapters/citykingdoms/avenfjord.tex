\subsection*{Avenfjord}

The rebuilt kingdom of the high elves and halflings sits on the southern shore
of the continent of \emph{Goltir}. It settled in the vast and fertile river
delta of the \emph{Al'ahri} river. Avenfjord is one of the youngest of all the
city kingdoms, and also the smallest.

\subsubsection*{Banner}

The banner of Avenfjord depicts two large crowned towers built together by
a bridge, with a backdrop of blue (for the sea) and green (for the fertile
farm land).

\subsubsection*{History}

It was founded in MI:1920 when the giants, an extra planar race of towering
humanoids, destroyed the previous city of the halflings and elves called
\emph{Nen-Hilith}. They had invaded the continent of \emph{Farlar} thirty
years prior, to wage war against the dragons who also call that continent
their home. \emph{Nen-Hilith} was situated on the northern shores of
\emph{Farlar} (just across the sea from where Avenfjord stands now).

For many years the elves and halflings were untouched and neutral in the
war between the dragons and the giants. But as the giants had seized most of
the rivers and lakes south of \emph{Nen-Hilith} and the fighting had crept
north towards the outlying villages and towns the elves and halflings joined
the war on the side of the dragons in MI:1912. After several failed campaigns
to regain control over the city's water supply, the giants began a devastating
siege against the city in MI:1918. After enduring the siege for more than two
years, with the military aid of various allies, the starved and weakened elves
and halflings were forced to abandon \emph{Nen-Hilith} and flee across the sea
to the north. The giants then razed the city to the ground.

\subsubsection*{Formian War}

In MI:1926 a war broke out between a hive of formians who objected to the
elves settling in their lands. The kingdom was still in the early stages of
rebuilding, and the new war threatened the very existence of the nation. The
then ruler of Avenfjord, King Ishmael the II., allowed a travelling wizard
named \emph{Taras} to combat the formians by adapting the \emph{black blight}
to infect and weaken them. Going far beyond anyone's expectation, the modified
plague infected and killed the vast majority of formians living in the Goban
desert north of \emph{Avenfjord}. After realising that he had ordered genocide
upon an entire sentient species \emph{King Ishmael II} exiled \emph{Taras}, who
claimed that he did not know the blight would kill the formians. Later in the
same year King Ishmael II committed suicide. His son \emph{King Ishmael III}
took the throne soon after.

\aren{If he just had enough resolve left to take Taras with him...}

\subsubsection*{Culture}

Although one might suspect that the near destruction would shake the culture
of the elves and halflings to the core, you might be wrong. Instead the
formian war, and the destruction of their previous city hardened the nation's
stance of non-interference. The people of Avenfjord prefer not to meddle in
other people's issues and troubles, and prefer diplomatic solutions over
conflict.

Avenfjord are known for their generous patronage for the arts and sciences and
house many theatres, libraries, galleries and workshops. They fund one of the
largest arcane academies on \emph{Aror} and generally hold good relations with
most other city kingdoms. The elves and halflings of Avenfjord are often
described as jovial, carefree but creative and expert diplomats. They are an
open society, and especially welcome other artists and craftsmen into their
cities.

\subsubsection*{Population}

The city of Avenfjord is now home to roughly two million people, yet the city
of \emph{Nen-Hilith} was home to almost twenty million at its peak. Most of
the citizens are elves (48\%) and halflings (39\%), with dwarves following
third (8\%) and all other races making up the remaining 5\%.

\subsubsection*{Rule}

Avenfjord is a dual monarchy, where both the reigning king or queen of the
elves, and the reigning monarch of the halflings rule together. Neither of
the monarchs has absolute reign alone. Although Avenfjord holds houses of
nobility, their power and influence is minimal compared to the houses of
other city kingdoms. Slavery and serfdom is outlawed, and all citizens of
Avenfjord are free people. The city also has an independent court and police,
that enact the laws the monarch sign into law fairly.

\subsubsection*{Relations}

Avenfjord has not signed the \emph{Vonir Accord} with Norbury. Although this
puts citizens of Avenfjord at risk of enslavement should they travel to
Norbury, such cases are extremely rare. Neither Norbury nor Avenfjord wish
to raise diplomatic issues between the two city kingdoms.

Though traditionally heavily aligned with the city kingdom of Forsby, the
failure to send military aid that was promised during that cities siege
soured relations with most city kingdoms, but with Forsby specifically.

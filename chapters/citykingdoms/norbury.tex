\subsection{Norbury}

\graham{Surely the most vile city kingdom Aror has to offer...}
\aren{Bless thy innocent heart, for you have not lived long enough to see the
  rise of Morkan.}

The second youngest city kingdom, \emph{Norbury} resides on a large island off
the northern coast of the continent \emph{Eilean Mor}. Norbury is a large
walled city, encompassing a quarter of the huge island.

\subsubsection*{History}

It was founded around \emph{GT:15500} as a joint military outpost of
\emph{Hraglund} and other northern baronies of Eilean Mor. It was originally
founded as first line of defence against the many raids of the beast races
that came from the northern most continent of \emph{Iâfandir}. Quickly the
fortress of Norbury grew into a castle, and more and more people were
required to keep the castle and its army supplied. Armies need smiths,
smiths need smelters, smelters need coal huts and miners, and all of
these need food, lodging and entertainment. Within a few generations
Norbury exploded in size and population, all working towards one goal:
keeping the raids and incursion of the beast races away from the main
continent.

\subsubsection*{Banner}

The kingdom's banner shows two silver swords crossed at the blade, against a
light red background. The main colours of the kingdom are silver and red.

\subsubsection*{Districts}

The centre of the city is a large market place, with a big stone steeple in
the middle. The tower both signifies the eternal vigilance, but also houses a
large bell that is rung in case of an attack. Inscribed into the walls of
the steeple are the laws of the kingdom for everybody to read. The market
place is vast circular open courtyard, and all sorts of goods - including
slaves - are sold there.

To the north lies \emph{Norbury} castle, a huge fortified military camp and
seat of the monarch. It oversees the northern sea off the island, and rests
upon a roughly two hundred metre high cliff side.

The surrounding area of the walled city houses farms and smaller villages that
are ruled by the kingdom as well. Since these farms are exposed to sea raids,
the northern shores of the island are dotted with small military camps, guard
towers and outlooks that notify the villages of impending raids.

\subsubsection*{Sea Raids}

A network of watch towers, light houses, guard towers and scout ships
constantly check the sea north of \emph{Eilean Mor} for any impending sea
raiders that embark from the continent of \emph{Iâfandir} if a suspicious
ship or raiding party is discovered, the entire kingdom goes on high alert and
deploys their fleet. Norbury are excellent ship builders, sailors and warriors
on the sea; and so they prefer to capture and destroy raiding parties before
they land on \emph{Eilean Mor}. Most of the ships manned by the beast races of
\emph{Iâfandir} are often not built for prolonged sea battle and thus are
often sunk off the cost of the continent. Those raiders that survive are
fished out of the sea and then brought back home to Norbury as slaves.

Sometimes raiding parties do sneak past the ever vigilant eye of the city,
and then land on the northern shores of Eilean Mor. Most of the baronies
on the shores have increased their armies to repel these raids upon their
lands, however they also often call the mainstay army of Norbury for aid.
Then the soldiers of the barony bars the raiders from entering their land,
while the ships and navy of Norbury attack the landing party from the sea.

\subsubsection*{Religious Civil War}

Ever since its foundation, the religions surrounding \emph{Lor}, the
\emph{Order} and the \emph{Three Kings} vied for the position of dominance
within the city. The \emph{Order} had the most followers, followed by
\emph{Three Kings} and then \emph{Lor}. Although the priests and paladins of
\emph{Order} raised issues with mistreatment of slaves, they were not outright
against slavery, unlike the priests and followers of \emph{Lor} who wished
to see the practice banned. The followers of the \emph{Three Kings} found both
to be weak in the face of their common enemy, and were ready to expel both
religions from the city. Over centuries this conflict hardened and brewed in
the heads of the followers and citizens. Then, in MI:1480 when a follower of
the \emph{Three Kings} beat his slave to death with a pavement stone in the
middle of the market place for disobeying him, a paladin of \emph{Lor}
interfered, killing the slave owner in single combat. This was the drop that
brought the temple to topple over. The followers of the \emph{Three Kings},
moved in retaliation against the temple and followers of \emph{Lor}. The
\emph{Order}, who sided with the mistreated slave, joined the conflict on the
side of the followers of \emph{Lor}. The civil war lasted for two weeks, in
which both the \emph{Order} and the followers of \emph{Lor} suffered heavy
losses and ultimately defeat against a force inferior in numbers. To prevent
the bloodshed to spilling over to civilians, then ruling queen Arianna of
Nordholm, banned the religions associated with \emph{Lor} and \emph{Order} and
exiled their followers for being ``weak in the face of adversity''.

This conflict has long passed, but still the prevailing attitude in Norbury is
that followers of the \emph{Order} and \emph{Lor} were, and are still, weak
and not worthy of honour. Although the ban is still in place, the exile has
since been lifted, allowing priests and paladins of the two gods to visit the
city.

\subsubsection*{Culture}

Since the incursion and raids of the beast races still occur to this day,
and have grown to be more ferocious and demanding, the culture of Norbury
has grown in response. Norbury is from the lowliest slave and peasant up
to the king or queen herself, a meritocracy. You are worth as much to the
city and in the eyes of your fellow citizen as you can contribute to the
defence effort.

Men and women of Norbury pride themselves in the work they are contributing to
the collective defence effort, be it front line combat, creating weapons and
armour for those that do, or aiding the effort in an administrative
fashion. Warrior culture runs strong in the kingdom, their deeds are sung in
taverns, their likeness is made eternal in art. Many fighters and warriors of
Norbury follow the \emph{Three Kings}, although worship of \emph{Forun} is also
wide spread in the kingdom.

Although arcane and divine magic is still frowned upon in the kingdom, the army
of Norbury runs an academy for battle mages and wizards. Arcane research is
often scrutinized by its potential military application, and wizards are also
required to undergo basic military training in weapons and armour.

All citizens of Norbury (and those that wish to become citizens) must complete
a mandatory civil service of at least five years. Many use this mandatory
service to begin an apprenticeship, while others join the military and counter
raids. No one is excluded from this service, not even the children of the
reigning monarchs. Avoiding this service is seen as a great dishonour, and a
major crime.

\subsubsection*{Society}

Titles within the kingdom's society, such as commoner, earl, baron or even
duke and grand duke can be bought by any citizen of Norbury. There is a rather
unusual twist: These titles are not inherited, which means that the son of a
duke is a commoner upon birth. The common census is, that this child has not
done anything yet to earn the rank of earl for him or herself. There is a high
pressure on children to achieve the same status as their parents, or perhaps
even outperform them. Failure is always an option as well, as it is not
unheard of for a noble son to fall into slavery.

Who reigns as king or queen is decided in a ritual combat between all eligible
arch dukes that wish to rise to the challenge. This combat is not to the
death, although many grand duke's have perished in their claim for the
kingdom. A king or queen reins until his reign is challenged by another, or
until they die or resign. The people of Norbury mostly do not care what race
or social background of their king or queen, but instead judge all their peers
by their honour, combat prowess, and contribution to society as a whole.

\subsubsection*{Slavery}

Norbury is the foremost kingdom to practice slavery on a massive scale. Both
\emph{indentured servitude} and \emph{chattel slavery} are encoded in Norbury
law.

\emph{Indentured servitude} may befall anyone who is in debt to another,
commits crimes, or dishonours themselves in battle. It theoretically can
befall anyone, even kings, but is limited in its duration. Once the servitude
has been fulfilled the servant is restored to his previous title.

Many of the raiders that are captured, and most of those that commit major
crimes are enslaved. The wizard academy as constructed special arcane collars
that bind the slave to their owners. Slavery is often a sentence for life, as
there is no legal way to escape slavery, unless the owner releases the slave.
A legal hurdle that is not cheap to the owner. Norbury slaves have no rights
whatsoever, and are marked, registered by number, and tracked down by the
\emph{Hunter's Guild} should they decide to run. Slave collars allow their
owners to track, command and often punish their wearers, as well as making
almost all humanoid and beast races sterile.

The \emph{Hunter's Guild} has also developed a colour classification for
slaves, which can be applied to the slave's collar or rune for easy
identification.  \emph{Black} means that the slave is ``valueless'' or
``useless''. \emph{White} coloured slaves have no special qualities or
redeeming skills. \emph{Green} slaves have a special skill that makes them
valuable, which is often inscribed with a symbol right next to the green
identification mark. \emph{Red} slaves have no current owner, and \emph{gold}
denotes slaves of high value. These are often specialised craftsmen, priced
fighters or perhaps slaves with magical talents. Any of these colours may be
combined with each other to easily signal the current status of the slave.

In the last centuries various counter measures to the slave collars have been
developed, allowing slaves to bet set free. In turn the academy has changed to
permanent slave tattoos instead, which cannot be so easily removed. Slowly the
arcane tattoos (called \emph{slave runes}) are replacing the arcane collars.

Foreigners are allowed to purchase slaves, however they have to pay an
additional fee half the slave's value. This was intentionally done, so that
most of the slaves remain within the city and contribute to the city. Slave
ownership is recognised as lawful in all kingdoms and baronies that have
signed the \emph{Vonir Accord}.

The \emph{Vonir Accord}, named after King Vonir of Norbury, is a treaty signed
by almost all major city kingdoms and most bigger baronies of the north. In
exchange for immunity to enslavement of the signing parties' citizens, all
Norbury slaves found on foreign soil must be returned to Norbury. Further all
signing parties must acknowledge the owner's claim over his slave and not
interfere with his rights to his property. This treaty, along with the
ferocious hunter's guild, and the hurdles of leaving an island, makes it almost
impossible for slaves to escape.

\subsubsection*{Population}

Norbury, and its outlying villages, houses roughly 4 million people. Of which
the vast majority are human and deepkin (39\%), then elves (24\%) followed by
dwarves (15\%) and half races (10\%). The rest (12\%) are beast races, almost
all of which are enslaved. Roughly 41\% of the entire population is either
currently enslaved or in servitude.

\subsubsection*{Hunter's Guild}

The \emph{Hunter's Guild} is the slavery guild of Norbury. It is tasked with
marking and enslaving new slaves, keeping track of existing ones, and tracking
down and returning those that have escaped. It operates an office building and
a large auction hall in the main square of the city. It also employs
\emph{hunters}, often rangers and rogues, that are tasked with tracking down
escaped slaves that fled across the borders of the kingdom. The guilds sole
income is the auction, and various administrative fees it collects for
transferring or freeing slaves. It does not charge for retrieving runaway
slaves, or for registering new slaves into the kingdom. The guild is the main
source of income for most soldiers of the Norbury army, as they pay a share of
the auction's profit for all soldiers who bring them new slaves to sell.

The hunter's guild banner is a skull of a deer (with antlers) painted in red
upon a white background.

\subsection{Hraglund}

\emph{Hraglund} is a humanoid city kingdom on the eastern shores of
\emph{Eilean Mor}. It is one of the oldest city kingdoms on Aror, having been
founded in \emph{GT:2180}.

\subsubsection*{Banner}

The kingdom flies a green banner with a white shield that contains a five
headed hydra. This is also the banner which had been used by the
\emph{Wayfaerer's Guild} since the foundations of the first hunting camp.

\subsubsection*{History}

\emph{Hraglund} grew out of smaller villages that settled around a fortified
hunting camp called \emph{Wayfaerer's Guild}. The hunters proofed to be
effective in deterring the attacks and raids from the monstrous races that
lived in the surrounding area of the guild, and thus attracted more and more
settlers, farmers and workers. Soon the hunting camp grew, first into a small
military outpost and over the span of many centuries into a full kingdom.

Once \emph{Hraglund} was the biggest city on \emph{Eilean Mor}, with over
31 million people living within the kingdom and in the vast outlying lands
that included many smaller and bigger cities, such as the river trading city
of \emph{Braemer}.

\subsubsection*{Siege}

In \emph{MI:-2} the city was under siege by vast armies that had declared war
against \emph{Griannar} and his followers in the name of the
\emph{Silent Queen}. Most of these armies that laid siege to the city, were
mercenaries that had been paid by the priests and priestesses of the Silent
Queen. Hraglund had been a major centre for the faith of Griannar, run by the
cardinal of the Holy Church. At first the city defended itself against the
siege for several months, but the majority of the population grew unruly and
attempted to oust the Church of Griannar from the city. After a civil unrest
and even skirmishes had broken out between the city guard, believers and the
part of the population that wanted to exile the church from the city. They
believed that if the church were ousted from the kingdom, the armies in front
of the gates would abort their siege. During that civil unrest the cardinal
fled the city, but was ultimately betrayed, captured and executed by the
followers of the Silent Queen. Just as the rebellious elements predicted, the
besieging army left soon after.

\subsubsection*{Plague}

In \emph{MI:1980} the city was struck with a major outbreak of the
\emph{black blight}, in which roughly 12 million people lost their lives. The
city was crippled, weakened, and lost most of its military power and influence
in the resulting chaos. The city not only lost many of its inhabitants, but
also they lost their king and the last heir of their noble house of
\emph{Altrizzi} that had ruled the city for many centuries. Now the city is
ruled by a council of high ranking officials and dukes that have been trying
to restore order to the kingdom. It was later revealed that a mad wizard and
follower of the \emph{Aria} had allowed a plague bearer into the city in an
attempt to take revenge against the kingdom. The wizard claimed he had a
personal vengeance against the kingdom, his brother was executed by the kingdom
for necromancy. He was sentenced and publicly hanged for high treason.

\subsubsection*{War with Nerevar}

Many of the surrounding baronies sensed the weakness in Hraglund after the
plague had been defeated in \emph{MI:1982}. After consolidating into a large
alliance called Nerevar the baronies declared war on Hraglund in an attempt to
take control of the city. Although Nerevar was inferior in most ways to
Hraglund the war dragged on for several years.  Hraglund had lost most of its
fighting forces but its overwhelming wealth made it possible for the kingdom
to hire mercenaries to fight the war for them. The war cost both sides
thousands of lives, displaced even more civilians but was ultimately lost by
Nerevar, when Tredegår joined the war as allies on the side of
Hraglund. Nerevar fell apart again into various smaller baronies in the chaos
and aftermath of their defeat. As war reparations the kingdom annexed most of
the border baronies and integrated them into the kingdom. Afterwards the
council of Hraglund, supported by their allies from Tredegår, claimed on
multiple occasions that Norbury had secretly aided the alliance of Nerever to
weakened Hraglund's power in the region. Something that Norbury vehemently
denied.

\subsubsection*{Population}

After the plague the city had roughly 19 million people, of which most were
humans (43\%), elves (25\%), halflings (21\%) and dwarves (9\%) and other
various half races (2\%).

\subsubsection*{Culture}

After both the religious siege and later on the plague, many citizens of
Hraglund lost faith in the new religions of the lesser deities. Most of
the citizens within the city turned towards atheism, science and arcane
studies. While the population of the country region returned to the old
ways. This had caused a trench between the city and country folk that split
the cultures of the kingdom in two. The city folk tend to be educated in
the sciences, arcane arts, enjoy art and are welcoming. While the country folk
returned to the old ways and traditions, including a strict adherence to the
shamanic ways of the three mothers. Although the country folk are known to be
blunt and perhaps a bit abrasive, they are still welcoming, good hosts and see
themselves as citizens of the kingdom first.

\subsubsection*{Society}

The council of dukes and government officials now rules the kingdom in the
absence of a ruling monarch. The city kingdom is known for its stability,
fair laws and for having fought and defeated many problems and enemies in
its past. They have a strong culture, and are proud of their nation, which
their society and culture embodies. Unlike most of its neighbours the kingdom
does not practice slavery, yet still practices indentured servitude as an
alternative to incarceration.

\subsubsection*{Relations}

Hraglund always had an uneasy relationship with neighbouring baronies
and realms, as they are known to aggressively extend their influence by any
means necessary. This often put them in direct conflict with both
Helmarnock as well as Norbury. However the kingdom holds good
relations with both Forsby and Tredegår.

\subsection{Forsby}

\aren{You can tell someone hails from Forsby from the first word that comes
  out of their mouth.}

The city kingdom of \emph{Forsby} was founded in \emph{GT:8840}, and resides
on the western shores of the continent of \emph{Goltir}. Forsby castle was
built the edge of a narrow fjord, upon white cliffs, giving it a grant view of
the sea. The noble district, common area, as well as the main square lie in a
semi circle around the main castle up on the cliff itself. While the harbour
was built on stilts into the sea below the cliff. Several elevators, both for
transporting people and goods, as well as many smaller paths and stairs, lead
down to the harbour. Several houses, taverns, shops and inns were hewn into
the cliff side, giving Forsby its distinct look when one travels in from the
sea. Two large light houses sit high upon the cliff on each side the fjord,
guiding ships into the narrow opening towards the harbour.

\subsubsection*{Banner}

Forsby's banner depicts a yellow hippogriff within a shield, upon a red
backdrop. A crown with three tips rests upon the hippogriff's head.

\subsubsection*{Climate and Surroundings}

Lying the northern parts of \emph{Goltir}, Forsby experiences harsh and cold
winters, and mild summers. While the soil is fertile, only the strongest crops
grow in this harsher climate. Forsby is surrounded by lush and ancient pine
forests, as well as ample farmland.

A river named \emph{Hafon} runs into the sea beyond the northern wall of
Forsby. It is the main source of fresh water for the city, and it also makes
the farmland east of the kingdom fertile for crops. Hafon thunders into the
sea north of the city, and has its origin in a small mountain range east of
the kingdom.

\subsubsection*{Population}

Forsby, and its surrounding outlying villages, is home to roughly 12 million
people, of which most (58\%) are human. Closely followed by deepkin (16\%)
and the elven races (12\%), dwarves (8\%) and various half races (4\%) and
others (2\%) such as Diarim and Umgeher.

\subsubsection*{Pit}

But \emph{Forsby} also has a darker and secretive side. Beneath the main
castle and noble district deep in the cliffs lies the aqueduct that transports
the waste down into the sea. It is simply referred to as the \emph{pit}.
Although a vast and dangerous sewer complex, many people and creatures call
its many levels their home. Existence of the pit is an open secret, and many
entrances to it can be found across the cliff side and in the harbour for
Forsby.

The upper level houses a vast bazaar and active criminal underworld, where
everything and everyone can be bought for the right price. The lower levels
are mainly uncharted, and often house creatures that would not be welcome on
the surface. Sightings of vampires, sentient flesh golems and hags have been
reported in the lower depths of the pit.

\subsubsection*{Culture}

Although \emph{Forsby} has a long standing tradition of chivalry and nobility,
the general population is well aware of the hypocrisy by not moving against the
criminal underworld of the \emph{pit}. The average citizen is nevertheless
proud of the ancient history of the city, its wealth and that a long line of
just kings and queens have lead the kingdom to become one the most powerful on
the planet. \emph{Forsby} is known for its arcane academy, which has produced
many fine and outstanding wizards; as well as for its public schools that
teach any children within the city and kingdom.

The people of \emph{Forsby} are known as stubborn, determined but hard working
to achieve their goals. They are well recognised by their rather cryptic dialect
of the common language called \emph{Reatham}. The city itself is open to every
civilised race, and so are its people.

The most worshipped religions of \emph{Forsby} \emph{Forun}, \emph{Lor}, but
also \emph{Aria} with the less noble elements that work and live in the
\emph{pit}. Many holidays of \emph{Forun} are also national holidays and
celebrations in Forsby, including the harvest festival and the festival of
thousand suns.

\subsubsection*{Devil Siege}

\aren{It is hard to find someone that was not a part or at least affected by
  the war against the devils.}

In MI:2017 the city came under siege by a large force of devils who besieged
the city by land and by sea. After the initial wave of attackers were driven
back by the local knights of the Order, city guards and mages of the tower; a
second, even bigger wave began laying siege on the city. During the initial
weeks of the siege many of the surrounding villages of the kingdom were
destroyed and its inhabitants killed.

As the siege dragged on for weeks, many more allies of Forsby - such as
Hraglund, Norbury, Kilmarnock or Tredegar sent reinforcements. After the elves
and halflings of Avenfjord joined the coalition, the defenders created a
military base on the Silver Isles to jointly attempt to break the sea embargo
that cut off Forsby. In the last moments before the assault, even Morkan
joined with war ships. Although the steel ships of the devils were superior,
the vast amount of ships and men raised by the coalition broke through the sea
embargo. The coalition lost many ships and men, and only a handful of vessels
made it to the harbour of Forsby. Although the elves and halflings promised
ships, none fleet made it to the rally point to aid Forsby.

During the time of the siege a group of heroes recruited by the coalition
secured the help of an elder white dragon named \emph{Northwind}. Northwind,
and his army of monstrous creatures proved invaluable to defeat the armies of
the devils. It is unknown what happened to the leader of the devils, though
the prevailing theory is that he abandoned the battle after the dragon and its
army came to aid the city.

The siege lasted for almost 7 months, and was formally declared won in the
second month of MI:2018. The reigning monarch Ulaf Thorgilson died during the
final battle of the siege, and his son Olaf became king. The dragon remained
nearby, and - although most of his minions perished - still holds a vast
political influence in Forsby.

The siege left all affected kingdoms in a weakened state. Many men perished
during the siege, and towns and hamlets were left empty because of that. Many
ships were lost and the struggle to rebuild became an arms race. But the siege
proved that all kingdoms could work together when faced with a common enemy.
The outcome of the war soured relationship between Forsby and Avenfjord, who
failed to provide promised resources and ships. It also lead to hostility
toward anyone perceived to be a devil sympathiser or spy, causing massive
extra-judicial persecutions and killings of perceived demon worshippers and
summoners, as well as \emph{tieflings}.

\subsubsection*{Nobility}

The city kingdom of \emph{Forsby} has a long standing tradition of noble
families that have been in charge of the cities affairs for several thousand
years. Their names carry weight even outside the borders of the city, and
they are well established in other city kingdoms across the globe.

The three major families are \emph{Banér}, \emph{Liljenar} and \emph{Lorham}
who have, in turn, ruled the city since its foundation. These three houses
have been known to feud and fight amongst each other, yet still they share one
common goal: the survival and prosperity of the kingdom. While these three vie
for the throne, there are several smaller houses of nobility that control
smaller aspects of the city: House \emph{Forholm} runs the cities trading
company and shipping docks, and have been known to do business with the
shadier aspects of the \emph{pit}. The House of \emph{Lemberg} is known to
operate most of the kingdom's farms outside of the city, and are vital to the
city's self sustainability in terms of food and agricultural products.

\subsubsection*{Relations}

\emph{Forsby} holds good relations with most city kingdoms, even with
\emph{Morkan}. The kingdom of \emph{Forsby} is a signer of the
\emph{Vonir Accord}.


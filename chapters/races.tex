\twocolumn
\section{Races}

The people of Aror make up a multi flavoured stew of various races and
backgrounds. Apart from may races that you might already know from your
home world, such as humans, elves, dwarves or gnomes; Aror is also home to some
you might have never heard of before. Two of these are the \emph{deepkin}, the
\emph{diarim} and the \emph{umgeher}.

You will find that race matters little. An elf of \emph{Norbury} often shares
the warrior culture, and faith in the meritocratic societal ideas
of \emph{Norbury}. While a dwarf of \emph{Nen-Hilith} will be accustomed and
to the artistic culture and creative ways of the artisans and performers of
that city.

The union of non-humanoid races with humanoid races is something that does
not exist in \emph{Everblack}. Although you might have met ``dragonborn''
(the unity of a human and a dragon) in other realms, you will not find them
on Everblack. And if you do, they are probably extra-planar visitors much
like yourself.

\begin{note}
These sections enhance or change the existing core races of D\&D. Any lore you
find in the base books that does not interfere with what is written here, still
holds true.
\end{note}

\subsection{Early History}

Roughly sixty thousand years ago an ice age covered most of the world
of \emph{Aror} in vast sheets of ice and snow. Continents that were separated
by sea, where now suddenly accessible through thick sheets of ice. This
allowed the four core humanoid races, the humans, elves, dwarves and halflings
to settle the entire known world. It is still unclear where they originated
from exactly, but the common scientific consensus is that these races trace
their origins back to the lush rainforests, steppes and savannah
of \emph{Arania}.

\subsubsection{Aeon of Strife}
\label{sec:Aeon of Strife}

When the humanoid races arrived in the other continents they found that they
were already inhabited by sentient races. Ogres, minotaurs, trolls, fey,
hobgoblins, gnolls, bugbears already called these continents their home,
alongside non-sentient monstrous species such as hydras, manticores and
wyverns. Through ancient stories of the dwarves, early writings, ancient stone
tablets and even cave paintings, it was revealed that the humanoid ancestors
were in a constant state of conflict with these monstrous races. Wars,
skirmishes, and often the destruction of entire early villages and even cities
was a common occurrence in early history. This early time of constant fighting,
turmoil and wars against the sentient beast races is referred to as the
\emph{aeon of strife}.

The aeon saw the birth of many new races, such as
\hyperref[sec:Vampires]{vampires}, \hyperref[sec:Fey]{fey} and
\hyperref[sec:Lycanthropes]{lycanthropes}. It also saw a vast destruction of
nature, as both sides of the war cut down forests, dried swamps, and forever
corrupted landscapes and areas with dark and foul magic.

Even though the aeon of strife is now thousands years past, it is still
vividly remembered in both humanoid and monstrous cultures through stories,
song, and tradition.

\subsubsection{Schism}
\label{sec:Schism}

The ultimately successful survival strategy to deal with such a hostile
environment during the strife ingrained itself in the culture of the core
humanoid races. The dwarves went underground and organised themselves in
strict hierarchical clans and cities that allowed them to optimise their
societies to the harsh realities and lack of resources of the depths. Some
humans and elves followed the dwarves underground but ultimately failed to
replicate the dwarven's success (with a few notable exceptions). Although
the \emph{deepkin} - the underground dwelling cousins of the humans - built
large civilisations underground they were ultimately defeated by the sentient
races of the depths and driven to the surface. The dark elves instead relied
on small clans, families and by being constantly on the move to ensure the
survival of their species.

Meanwhile on the surface elves and halflings sought to settle as far away from
the monstrous races as possible, leading them to the continent of \emph{South
Goltir} and \emph{Farlar}, as well as becoming the pale elves by settling the
vast ice sheets of the north and south poles. Humans on the other hand used
their ingenuity and skill to build great civilisations and cities that could
potentially withstand the skirmishes and sieges of the sentient monstrous
races. Through many iterations over the course of thousands of years, which
resulted in countless destroyed and ransacked cities and fallen civilisations,
humans have now achieved the unthinkable: dethrone the monstrous races as the
predominant species across all of Aror.

\subsubsection{Exodus from the Depths}
\label{sec:Exodus from the Depths}

Now many elves and halflings have joined the human effort of building large
centres of civilisation, enriching the predominantly human city kingdoms that
dot the world of \emph{Aror}. Their struggle against the sentient monstrous
races is far from over, especially in the central regions of the continents,
or from the still predominantly monstrous continent of \emph{Iâfandir}. The
majority of dwarves have remained underground, continuing their strict ways of
life as it has served them for centuries. A success that was not shared by the
dark elves and deepkin, who have mostly abandoned the deep and returned to the
surface.

\subsection{Deepkin}
\label{sec:Deepkin}

\aren{I had the misfortune of seeing my people's decline with my own eyes.
I was powerless, and unable to stop it. We are but a shadow of our former
selves...}

\graham{But still you live. And once the time is right, you may reclaim what
once was.}

The \emph{deepkin} are the cavern dwelling cousins of humans. They are capable
of seeing in the dark, are however colour blind, and often have white pale
skin, red to brownish hair, and either red, green, blue or yellow eyes.

\emph{Deepkin} society is one of the older societies on \emph{Aror}, with a
long standing history in the arcane arts, building magnificent underground
cities, libraries and workshops. Ancient \emph{deepkin} were master arcane
smiths, golem constructors, inventor of many magic based constructs and
technology still used today.

The ancient deepkin had their dominance challenged by the other races
of the deep, namely the dwarves, dark elves, and a now extinct species called
the \emph{ilians}. Thousand of years of conflicts lead to the decline of their
magnificent ancient civilisation. Instead of perishing however, they fled to
the surface, where they were warmly received by their surface cousins. Nowadays
\emph{deepkin} culture is but a shadow of what it was, although they still try
to reclaim what was once theirs with expeditions into the deep.  Albeit they
were well known city and kingdom builders, only one deepkin settlement rose to
the rank of a formidable city kingdom: \emph{Stenheim}.

Although \emph{deepkin} are treated as their own race, they are still capable
of producing viable offspring with normal humans. The race of the offspring is
always that of the mother.

\begin{35e}{Deepkin Traits}
  \hyperref[sec:Speak Language]{Doresh} is the natural language of the
  Deepkin, and is equivalent to \emph{Undercommon}.

  \begin{itemize}[noitemsep]
    \item Medium: as medium creatures, \emph{Deepkin} have no special bonuses or
    penalties due to their size.
    \item \emph{Deepkin} base land speed is 30 ft.
    \item For all manners regarding racial restrictions or classifications
    \emph{Deepkin} count as humans.
    \item Dark vision out to 120 feet, but colour blind as a result.
    \item Bonus Feat: Just as their human cousins, \emph{Deepkin} can choose a
    bonus feat at first level.
    \item Automatic languages: Doresh, Teranim. Bonus Languages: Any (except
      secret languages)
    \item Favoured Class: Any. When determining whether a multi class takes an
    experience point penalty, his or her highest-level class does not count.
  \end{itemize}
\end{35e}

\subsection{Diarim}
\label{sec:Diarim}

\aren{The expression \emph{explaining freedom to a Diarim} has become
ingrained in many cultures as a metaphor for a fruitless labour.}

The \emph{Diarim} are a race of humanoid creatures that were bread for
specific tasks by the dark sorcerers and witches of the giants living on the
continent of \emph{Farlar}. They are the youngest of the races on \emph{Aror},
and are a mixture of various other humanoid races.

\emph{Diarim} come in in as many shapes and sizes as the giants had uses and
tasks for their engineered slave race. But most share a common set of features:
the light, fair skin of the \emph{Deepkin}, blue hair of the \emph{Snow elves},
and an ingrained sense of duty and loyalty to strict hierarchies from the
dwarves. Very few \emph{Diarim} ever escape the slavery of their masters, and
those few that do, find it hard to shake their eagerness to serve, please and
help that the giants have ingrained into them. Almost all have blue tribal
tattoos all over their body, which identify the current and past owners of the
individual \emph{diarim}.

The most common variant are labourers, small but stout breed, that was created
by introducing more dwarven heritage into the \emph{diarim}. They excel at
physical labour, such as mining and construction. Siegers were bread with
monstrous races, often even giants, and are used as front line soldiers,
gladiators and shock troops. They are larger than any other \emph{diarim}, and
often also serve as slave overseers over the others. Exciters were bred and
selected for their beauty, and are priced possessions to be traded and gifted
to other giants. Their primary role is to entertain their overlords through
song, dance and company.

Most \emph{diarim} have no names. They never refer to themselves with names,
and are only being given a names by the giants if they have distinguished
themselves, either through heroic deeds, or through crimes.

In the recent decades more and more \emph{diarim} have escaped the continent
of \emph{Farlar} and joined the other humanoid races. The giants had to learn
that you cannot suppress the curiosity, love for wandering and freedom for long
when you create a species based on humans, halflings and elves.

\begin{35e}{Diarim Traits}
  \textbf{Diarim Traits (EX)}:
  \begin{itemize}[noitemsep]
    \item Medium: as medium creatures, \emph{Umgeher} have no special bonuses or
    penalties due to their size.
    \item A \emph{Diarim}'s base land speed is 30 ft.
    \item \textbf{Weak Will (EX)}: All \emph{diarim} have a -2 penalty to will
    saves against charms and similar effects.
    \item Automatic languages: Giant, Teranim
    \item Favoured Class: Any. When determining whether a multi class takes an
    experience point penalty, his or her highest-level class does not count.
  \end{itemize}

  \textbf{Sieger Traits (EX)}: the following traits are in \emph{addition} to
  the \emph{diarim} traits, except when noted otherwise.
  \begin{itemize}[noitemsep]
    \item Large size. -1 penalty to Armour Class, -1 penalty on attack rolls,
    -4 penalty on Hide checks, +4 bonus on grapple checks, lifting and
    carrying limits double those of Medium characters.
    \item Space/Reach: 10 feet/5 feet
    \item +8 Strength, -2 Intelligence, -2 Charisma, -2 Wisdom
    \item Favoured Class: Barbarian
    \item Level Adjustment: +1
  \end{itemize}

  \textbf{Exciter Traits (EX)}: the following traits are in \emph{addition} to
  the \emph{diarim} traits, except when noted otherwise.
  \begin{itemize}[noitemsep]
    \item -2 Strength, -2 Constitution, +4 Charisma
    \item Favoured Class: Bard, Sorcerer
  \end{itemize}
\end{35e}

\subsection{Dwarves}
\label{sec:Dwarves}

\emph{Dwarves} are natural fighters, miners and smiths. Among all of the races
they are the most reclusive of all. Dwarves usually stand between 1.2 and 1.4
metres high, but are on average almost has heavy as humans. They often live
up to 250 years. Their skin is usually tan brown, their skin ranges from brown
to black, and they often have brown eyes to match. For most male dwarves the
beard is a symbol of status, and often stands as a symbol for that particular
dwarves caste within the clan. Female dwarves do the same, but with their
head hair.

\subsubsection*{Dwarven Culture}

They hardly join other cultures, and prefer to continue to live like their
ancestors did. \emph{Dwarves} organise themselves into huge clans or families
and live in deep caverns or mountains in a strict caste society. Those that do
not fit into these strict frameworks of societies are cast out, and then look
for other realms to live. Ever since the deep have become ever more dangerous
many smaller dwarven clans have fled the surface near human or elven
settlements, while others have seemingly integrated into the large city
kingdoms. Offering their expertise on trade and smithing to the other races.

Dwarves speak their own language called \emph{Rutari} with its own alphabet of
the same name.

\subsubsection*{Caste System}

Most dwarven clans strictly enforce their caste society to maintain order and
control within their societies. Anyone who does not fit within this system is
met with suspicion, contempt or exclusion and exile at the worst. The lowest
of the caste are slaves, which are drawn from the pool of criminals among the
dwarves, as well as from the other races that live within the dwarven society.
Other races, such as humans, elves, or even \emph{gnomes}, often have no chance
to advance out of the slave caste. Most dwarven clans however allow visitors
into their strongholds.

\begin{35e}{Dwarf Traits}
  \begin{itemize}[noitemsep]
    \item Medium: As Medium creatures, dwarves have no special bonuses or
      penalties due to their size.
    \item Dwarf base land speed is 20 feet. However, dwarves can move at this
      speed even when wearing medium or heavy armor or when carrying a medium or
      heavy load (unlike other creatures, whose speed is reduced in such
      situations).
    \item Dwarves are sturdy and hardy, and thus have a +2 racial bonus to
      constitution.
    \item Stability: A dwarf gains a +4 bonus on ability checks made to resist
      being bull rushed or tripped when standing on the ground (but not when
      climbing, flying, riding, or otherwise not standing firmly on the
      ground).
    \item Languages: Rutari, Teranim. Bonus Languages: Any (except secret
      languages)
    \item Favoured Class: Any. When determining whether a multi class takes an
          experience point penalty, his or her highest-level class does not
          count.
  \end{itemize}
\end{35e}

\subsection{Elves}
\label{sec:Elves}

Elves are tall - often between 1.8 and 2.4 metres - and slender race, with
long and pointy ears. The style of the elven ears varies, with some having
smaller pointy ears facing backwards, while others have longer and sharper
ears that follows the contours of their face. The variety within the elven
ears is vast, but is but a minor cosmetic difference. Their physique and build
is slender as well, with long skinny legs and arms. Elves live up to 400 years
of age, and thus often pick up professions that take longer to master, such as
wizardry, artistry or the sciences. Although elves live long, they often shift
focus in their goals. Apart from the humans, the elves are one of the older
races of \emph{Aror}.

Unlike humans, elves do have distinct sub races that differ from each other in
various physical and biological aspects. The \emph{snow elves} for example
have a natural resistance to cold like no other species have, while the
\emph{dark elves} have adapted to see better in the dark caverns they
inhabit. However they have not diverged so far from one another to not produce
vialable offspring. As with humans and deepkin the race of the sibling always
resembles that of the mother.

There are four major elven races that are recognised across the world of
\emph{Everblack}:

\subsubsection{High Elves}
\label{sec:High Elves}

The most numerous race of the elves are the \emph{high elves} of
\nameref{sec:Avenfjord}. High elves have fair skin with a hint of yellow and
gold. Their hair ranges from blond, fiery red to brown and black. They stand
between 1.8 and 2.4 metres high, towering with their slender appearance over
most other humanoid races with ease. They have pointed, elongated ears which
often point skyward, following the slender contour of their face.

High elves are the most adaptable and curious of the elves, and often live
within human city kingdoms, mixing and integrating well with other cultures
and societies. Even though they have their own kingdom, the vast majority of
high elves live outside the kingdom of Avenfjord. High elves are the only
kingdom building elven race, but do prefer to live with together with their
distant human cousins within baronies, kingdoms and city states.

The high elves are also the longest living of all the elven races, usually
living to grow 400 years old. This gives them ample time to study and learn
several fields and areas of expertise during their early childhood.

High Elves speak \hyperref[sec:Speak Language]{Enro'ad}, a variant of
\emph{Old Teranim}, but use the Taavid, halfling alphabet called \emph{Taavid}
to write it.

\begin{35e}{High Elf Traits}
  \hyperref[sec:Speak Language]{Enro'ad} is elvish, albeit the elven alphabet
  is now written with Taavid the \emph{halfling} script.

  \begin{itemize}[noitemsep]
    \item Medium: As Medium creatures, elves have no special bonuses or
    penalties due to their size.
    \item Low-Light Vision: An elf see twice as far as a human in starlight,
    moonlight, torchlight, and similar conditions of poor illumination. She
    retains the ability to distinguish colour and detail under these
    conditions.
    \item Elvish adulthood lasts longer than in other species, and they thus
      have more time to learn while they grow up. This grants them a +2 bonus
      to intelligence, and elves can select one extra
      \hyperref[sec:Education Feats]{education feat} at level 1.
    \item Automatic languages: Enro'ad, Teranim. Bonus Languages: Any (except
      secret languages)
    \item Favoured Class: Any. When determining whether a multi class takes an
    experience point penalty, his or her highest-level class does not count.
  \end{itemize}
\end{35e}

\subsubsection{Dark Elves}
\label{sec:Dark Elves}

The \emph{dark elves} live mostly underground, have black to blue skin, and
their hair ranges from a faint hint of blue, silver to snow white. Their eyes
are often red, blue or a green. They are the smallest of all elven races, and
range from 1.60 to 1.90 metres in height. In terms of bodily physique they
also more closely resemble humans. Their rounder faces, and their stronger and
more muscular physique sets them apart from their slender high elven cousins.

Dark elves once followed the dwarves underground, when the ancient battles
against the sentient non-humanoid creatures and the fey turned deadly and
dangerous. They have since adapted to that life underground, both physically
and culturally. They can see well in the dark, and their skin turned dark
letting them blend in with the dark surroundings of the depths. They also only
live roughly 120 years, and are thus not as long lived as their high elven
cousins.

Underground they are often live primitive, nomadic lives, adapted to the harsh
realities of the depths below. They have survived the dangerous environment of
the deep caverns by remaining on the move or by hiding from threats. They
value their small communities and family above all else.

Many dark elves also live on the surface, and much like their surface cousins,
they prefer to integrate into already existing humanoid kingdoms and baronies.
Dark elves are rather common all over the world, and can be found on almost all
continents.

\begin{35e}{Dark Elf Traits}
  \textbf{Dark Elf Traits (Ex)}: The following traits are in \emph{addition}
  to the high elf traits, except when noted.
  \begin{itemize}[noitemsep]
    \item Dark elves are considered smaller and more nimble than the other
      elven races, and thus gain a +2 racial bonus to dexterity.
    \item Dark vision out to 120 feet, albeit they are colour blind.
    \item Automatic languages: Doresh, Enro'ad, Teranim. Bonus Languages: Any
      (except secret languages)
  \end{itemize}
\end{35e}

\subsubsection{Snow Elves}
\label{sec:Snow Elves}

\aren{Snow Elves are the pinnacle of beauty...}

Far to the north and south live the \emph{snow elves}, nomadic hunter-gatherer
elves with white to silver blue skin, white, silver, grey or blue hair. Their
eyes are often bright blue, green or yellow. Male snow elves are capable of
growing facial hair, while many female snow elves have light red freckles
in their faces. Snow elves are as tall as their high elven brethren, ranging
from 1.8 to 2.4 metres, and also share most of their facial features with high
elves. They have adapted well to the colder climates, and can withstand the
cold far easier than any other humanoid race. This has also come with a major
drawback: With a life expectancy of 80 years, they are the elven race with
shortest lifespan. Many \emph{snow elves} live in smaller families and tribes,
content with surviving the harsh realities of the polar north and south by
becoming fierce hunters, fishers or even raiders themselves.

These snow elves from northern and southern tribes are generally known to be
calm, and quiet, preferring the solitude of a small group or town over the
vast masses and stretches of cities. They rarely leave the frozen north and
south, and are thus exotic, as in, many other humanoids haven't seen a tribal
snow elf in person. They do not build cities or kingdoms, but smaller tribes
often join to form larger ones (several hundred individuals) if required.

To tribal snow elves the community, family and their own tribe is everything, as
it ensures survival and in the frozen tundras. The harshest punishment within
these communities, reserved for major crimes, is exile which is often equal to
a death penalty. Since these exiles are then also shunned by other snow elven
tribes, they sometimes wander away from their homes and join city kingdoms or
baronies.

Snow elves rarely leave their icy domain and tribal lives willingly, and thus
the ancestors of the snow elves in the city kingdoms were either exiles, or
were once captured and brought there as slaves. Those civilised snow elves do
not harbour any animosity any more about what happened to their ancestors
thousands of years ago, and prefer to remain in the places and cultures they
now call their homes. To the tribal snow elves these city snow elves are equal
to exiles, and are thus not welcome in the frozen north or south.

\begin{35e}{Snow Elf Traits}
  \textbf{Snow Elf Traits (Ex)}: The following traits are in \emph{addition}
  to the high elf traits, except when noted.
  \begin{itemize}[noitemsep]
    \item \textbf{Pale Wastes (Su)}: A pale elf can live comfortably in
      conditions of extreme cold, even with barely any clothing or external
      sources of warmth. This ability functions like a continuous \emph{Endure
      Elements} but for cold conditions only.
    \item Snow elves are known for their great hunting skill which allows them
      to perceive hidden things and small details, and thus have a +2 bonus to
      wisdom.
    \item Automatic languages: Enro'ad, Teranim. Bonus Languages: Any (except
      secret languages)
  \end{itemize}
\end{35e}

\subsubsection{Wood Elves}
\label{sec:Wood Elves}

Wood elves have light brown or greenish skin, and green or red hair. Their
faces are often covered in light or red freckles, and their eyes are often
brown, blue or green. They are as tall as their high elven counter parts,
often ranging from 1.9 to 2.4 metres, and also share the facial features of
the other high elves. Although they are very close to their high elven
cousins, they are counted as their own race based on their ability to easily
build muscle mass. This often makes them the strongest elven race, and a wood
elf can easily be compared to \nameref{sec:Half-Orcs} in terms of bodily
strength. Even though they are known to join the major city kingdoms, they are
rarer than high or dark elves.

There are two major communities of wood elves. The first lives in the vast
temperate and boreal forest of \nameref{sec:Eilean Mor}, known as the
\nameref{sec:Dirgewood}. These wood elves live together with humans, halflings
of the Dirgewood, as well as the dark elves, deepkin and the dwarves of the
\nameref{sec:Great Divide}. They are followers of the \nameref{sec:Old Ways},
and prefer to build small settlements, towns and perhaps tiny cities of their
own. These wood elves are expert trackers, hunters, farmers, as well as shamans
and priests of the old ways.

And in the jungles \emph{Yua'cata} live the \emph{savage elves}, a loose
collection of tribes of cannibalistic and demon worshipping wood elves. They
rarely wander beyond the confines of their jungle and are one of the rarest
elven culture to meet in the civilised areas of \emph{Aror}. They live a
primitive life, adapted to the harsh and dangerous rain forest.

\begin{35e}{Woold Elf Traits}
  \textbf{Wood Elf Traits (Ex)}: The following traits are in \emph{addition}
  to the high elf traits, except when noted.
  \begin{itemize}[noitemsep]
    \item Wood elves have it easier to gain muscle mass than the other elvish
      races, and thus have a +2 racial bonus to strength.
    \item Automatic languages: Enro'ad. Bonus Languages: Any (except secret
      languages)
  \end{itemize}
\end{35e}

\subsubsection{Fey}
\label{sec:Fey}

Fey are magical and evil forest spirits that were created by the earliest druids
to protect nature from destruction. During the \nameref{sec:Aeon of Strife},
both monstrous and humanoid culture uprooted forests, cleared grassland, dried
marshland or outright destroyed or corrupted nature to fuel their war
machinery and ever growing populations.

Druids, from both sides of the war, desperate to preserve nature created
forest spirits that should serve as protectors to nature. But after the druids
realised that the gentle spirits they had created proofed ineffective in
deterring the humanoid and monstrous loggers, they used necromancy to corrupt
these spirits into evil and sadistic creatures.

The fey are now universally hated and feared, and are often the cause for great
disasters, tragedies and destruction in both humanoid and monstrous villages
and communities.

The most common fey are \emph{red caps}, small vicious and sadistic gnomes
that hunt and murder for nourishment.

\emph{Dryads} attack loggers and travellers that seek the shade of the tree
they are meant to protect. They feast upon living flesh, and sacrifice all
those they catch to the very tree they inhabit.

\emph{Nymphs} and \emph{Irrlichter} seek to lure people into a marshy death by
luring them into the forest with illusions.

\emph{Satyrs} were made to emulate the tribal patterns of the humanoid and
monstrous races, and build their own villages. From there, they venture forth
to raid and pillage other villages and towns. They are capable of forging
steel, and are expert hunters and trackers. They are often aided by evil
\emph{sprites} and \emph{pixies}, that aid the satyrs with their inherent
magical abilities.

\begin{35e}{Fey}
  Everblack fey do not vanish upon death, they simply die like any other
  mortal creature. All fey that are \emph{good}, move their alignment towards
  \emph{evil}. Most of them either serve druids unwillingly, or roam the
  forests, plains and mountains seeking wreak havoc on the monstrous and
  humanoid creatures alike.

  An \emph{Irrlicht} has the same statistics as a will-o-wisp, but are of
  type \emph{Fey} instead.
\end{35e}

\subsection{Gnomes}
\label{sec:Gnomes}

\emph{Gnomes} are the children of \emph{halflings} and \emph{dwarves}. Since
they have no identity of their own, no culture of their own, they often try to
integrate with their parents' cultures. Albeit they rarely fit in with either
societies. Gnomes themselves are sterile, and thus rarely create their own
families, they have inherit the longevity of their dwarven parent. All in
all \emph{gnomes} are among the rarest races on \emph{Aror}. They often
dedicate their lives to adventuring and other dangerous businesses, and rarely
settle down.

There are no \emph{gnome} settlements or even kingdoms, and most of them live
scattered across the world in the large city kingdoms. Offering their services
as spies, thieves and adventurers to anyone willing to pay. Most large dwarven
cities and halfling settlements also have a small \emph{gnome} population.

\begin{35e}{Gnome Traits}
  \begin{itemize}[noitemsep]
    \item Small: As a Small creature, a gnome gains a +1 size bonus to Armor
    Class, a +1 size bonus on attack rolls, and a +4 size bonus on Hide
    checks, but he uses smaller weapons than humans use, and his lifting and
    carrying limits are three-quarters of those of a Medium character.
    \item Their dwarven blood makes them hardier and sturdier, and thus have
      a +2 racial bonus to constitution.
    \item Gnome base land speed is 20 feet.
    \item Low-Light Vision: A gnome can see twice as far as a human in
    starlight, moonlight, torchlight, and similar conditions of poor
    illumination. He retains the ability to distinguish color and detail under
    these conditions.
    \item Languages: Teranim, Rutari. Bonus Languages: Any (except secret
       languages)
    \item Favoured Class: Any. When determining whether a multi class takes an
    experience point penalty, his or her highest-level class does not count.
  \end{itemize}
\end{35e}

\subsection{Half-Elves}
\label{sec:Half-Elves}

The union of human and elf was possible in the past, but since then the elves
and humans diverted too far apart to produce viable offspring. Most children
that are now born in the union between elf and men are often born with birth
defects, with severe mental disability, if they - and the mothers giving
birth - survive birth at all.

This was not always the case however, and tens of thousands of years ago the
union was possible. It produced a sizeable amount of half elves back then,
which over the course of the recent history struggled to keep their populace
alive and growing. Constant inter-family marriages over the course of
thousands of years have created a tight-knit clan of half elves that span the
globe. Although they form no kingdoms of their own, they have created a
secondary culture that focuses heavily on the continuation of their
people. Half-elf lore keepers keep meticulous record of what half-elven family
lives where, and with whom they are related. Half-elves are encouraged to
marry within their own species, as introducing either elven or human blood
would ``weaken'' the half-elf population.

\subsection{Halflings}
\label{sec:Halflings}

Most of the halflings are nomads, who travel across the lands in search for
places they could explore. If you think you have found a secluded spot on
\emph{Aror}, where no one else had sat foot before, you can be sure it is
already named after an halfling explorer. They travel in small families, and
in exchange for money offer their services to any small town they come across
on their travels. Halflings are curious, adventurous often find themselves
exploring the depths and other inhospitable places of \emph{Aror}.

A huge group of halfling families once decided to start another adventure: of
their own kingdom. They settled down near an elven kingdom and named their new
kingdom \emph{Brèagha Hilith}. After decades of peaceful coexistence between
the two races, they finally decided to tear down the last barriers and simply
merge the two kingdoms into a new one: \emph{Nen-Hilith}. It became a shining
beacon of civilisation, artistry and stability under the known city kingdoms
of \emph{Aror}.

\aren{Until a few giants stepped on them...}

\begin{35e}{Halfling Traits}
  \begin{itemize}[noitemsep]
    \item Halflings have a +2 racial bonus to dexterity.
    \item Small: As a Small creature, a halfling gains a +1 size bonus to
    Armor Class, a +1 size bonus on attack rolls, and a +4 size bonus on Hide
    checks, but she uses smaller weapons than humans use, and her lifting and
    carrying limits are three-quarters of those of a Medium character.
    \item Halfling base land speed is 20 feet.
    \item Automatic Languages: Taavid, Teranim. Bonus languages: Any
    (except secret languages)
    \item Favoured Class: Any. When determining whether a multi class takes an
          experience point penalty, his or her highest-level class does not
          count.
  \end{itemize}
\end{35e}

\subsection{Half-Orcs}
\label{sec:Half-Orcs}

Less rare than \emph{Gnomes}, but equally outcasts in most places, are the
offspring of the union of human, deepkin or elf with an orc. Much like other
half breed races \emph{half-orcs} are sterile, and thus have no inner need or
desire to settle down or start families. They often live amongst their human
parents offering their brute strength and enduring physique as heavy labourers
or fighters. A human mother giving birth to a \emph{half-orc} has a high
chance of dying during child birth. And since very few women take such a risk,
many \emph{half-orc} children are not the result of a voluntary union. This
sad reality, combined with a short temper, less than flattering appearance and
low living status and conditions, often elicit condescending or outright
demeaning behaviour from the other races towards \emph{half-orcs}.

\begin{35e}{Half-Orc Traits}
  \begin{itemize}[noitemsep]
    \item Their orcish blood makes it easier for half-orcs to gain muscle mass
      and thus have +2 racial bonus to Strength.
    \item Medium: As Medium creatures, half-orcs have no special bonuses or
      penalties due to their size.
    \item Half-orc base land speed is 30 feet.
    \item Orc Blood: For all effects related to race, a half-orc is considered
    an orc.
    \item Automatic Languages: Teranim, Orc. Bonus Languages: Any (except secret
      languages)
    \item Favoured Class: Any. When determining whether a multi class takes an
    experience point penalty, his or her highest-level class does not count.
  \end{itemize}
\end{35e}

\subsection{Humans}
\label{sec:Humans}

\emph{Humans} are one of the most ancient humanoid races of \emph{Aror}, and
also the dominant race of the planet. Human artefacts have been found dating
back hundreds of thousands of years, far beyond the history of any other
humanoid species. There are two separate ``races'' of humans: those inhabiting
the southern part of the hemisphere who usually have darker skin, and the
``northerners'' who usually have fair skin. This distinction is superficial
only. Apart from the tone in skin colour, there is no other biological
difference between the various human tribes and civilisations. Humans live up
to 80 years, and are known for being statesmen, diplomats, farmers,
adventurers, explorers and scientists.

\subsubsection*{Language}

Humans speak \emph{Teranim}, either as their primary language, or as their
secondary language together with their local language. \emph{Teranim} has
become the de-facto language of \emph{Aror} and is usually spoken almost
everywhere, even among the beast races. \emph{Teranim} is written in its own
alphabet of the same name. \emph{Old Teranim} and \emph{Ancient Teranim} exist,
spoken by the ancient humans, and are the root of most other languages and
various local dialects. Although \emph{ancient teranim} is no longer actively
spoken, various books, poems, stories and songs still exist in that language.

Humans living in the southern hemisphere often speak a language that is
stuck halfway between old and new \emph{Teranim}, called \emph{Kalest}. While
the people of \emph{Forsby} (and surrounding regions) have their own distinct
dialect, which can be so hard to understand that it has received its own
classification and name: \emph{Reatham}.

Albeit many local dialects exist, almost all humans, and the other races
living with them are capable of speaking \emph{Teranim}. It is, after all,
the official language of many governments, the language that is printed and
used in official capacity as well as in inter-kingdom cultural exchange and
trade.

\begin{35e}{Common in Aror}
  The language of \emph{Teranim} is equal to \emph{Common} of D\&D. See also
  the \nameref{sec:Speak Language} section for a list of available languages
  and their alphabet.
\end{35e}

\subsubsection*{Human Lands}

Humans can be found everywhere on Aror. But history indicates that they
originated on the southern continent of \emph{Arania}, and migrated to all
other continents during the last ice age tens of thousands of years
ago. Wherever humans settle they build villages, cities, and large kingdoms
and often become the dominant culture and social structure. Human kingdoms
have often endured for thousands of years, and have bested many difficulties
that had driven other societies to ruin. The ingenuity of the humans, their
stubborn attitude and their uncanny ability to adapt to any difficulty makes
them the dominant race of \emph{Aror}.

\subsubsection*{Human Culture}

Like most races, humans have no inherent global culture, tradition or customs.
Instead their believes and customs are ever evolving, and specific to the
realm they live in. But there is one trait that the average human has: ingenuity
in the face of adversity. No other species has managed to settle every corner
of the world and endure, and even build lasting civilisations out of the
hostile environment they found themselves in.

\begin{35e}{Human Traits}
  \begin{itemize}[noitemsep]
  \item Medium: As Medium creatures, humans have no special bonuses or
    penalties due to their size.
  \item Human base land speed is 30 feet.
  \item 1 extra feat at 1st level.
  \item 4 extra skill points at 1st level and 1 extra skill point at each
    additional level.
  \item Automatic Language: Teranim. Bonus Languages: Any (other than secret
    languages).
  \item Favoured Class: Any. When determining whether a multiclass human takes
    an experience point penalty, his or her highest-level class does not count.
  \end{itemize}
\end{35e}

\subsection{Ilians}
\label{sec:Ilians}

The \emph{Ilians} are a monstrous, humanoid and giant race that live in vast
and great cities deep underground. They are masters psionics, and live in a
strictly ordered society.

\subsubsection{Physiology}

The appearance of ilians varied across the castes, but they all had a few
features in common. Their skin was dark grey or light blue, were often above
two metres tall, and had no bodily hair. All ilians have at least some
inherent psionic capabilities.

Ilians reproduce by spewing forth a small worm, which grows into an almost
identical copy of the ``mother'' ilian over the course of several decades.
These worms could not crawl, but swim, and require a lot of nourishment to
grow. They were thus often kept in special maturation tanks filled with Ramesk
to ease their growth.

\subsubsection{History}

Not much is known about their early history, except that ilians already dwelt
in their great underground cities when the dwarves, elves and humans went
underground during the \nameref{sec:Schism}. The ilians did not take kindly to
what they perceived as intruders upon their land, and attempted to drive the
humanoid races back to the surface.

Although the ilians had a massive advantage, as their cities and war
infrastructure was already built, their slow reproductive cycles and strict
and unmoving societal structure ultimately caused them massive problems in
the war against the humanoid races.

During the strife the dark elves, dwarves and deepkin fought the ilians
relentlessly in an attempt to establish themselves in the deep. And although
early centuries of the war often ended in favour for the ilians, the tide
turned against them when the dwarves and deepkin began building
\hyperref[sec:Everblack Golem]{everblack golems} that could withstand the
psionic powers of the ilians. These constant wars cost both sides heavily,
and ultimately culminated in the downfall of the ilian culture and
society across most of Aror. The victory was devastating for the humanoids
as well, and it lead to a
\hyperref[sec:Exodus from the Depths]{great exodus from the depths} for many
deep dwelling humanoids.

Nowadays only ruins remain where once proud ilian cities stood. Burying their
psionic machinery, artefacts and riches beneath them. Some communities and
cities have survived and guard their treasures still, while some retreated
deeper underground into solitude and isolation.

\subsubsection{Society}

The ilian society was strictly ordered into castes, where the members of each
caste had distinct biological traits suited for the work they were required to
perform. There were four main casts: \emph{Arnak}, \emph{Elnak}, \emph{Karek},
and \emph{Elmek}.

Ilian also have their own language, called \emph{ilian} which had no writing
system. Ilians inscribed their thoughts, feelings or knowledge as psionic
energy into \hyperref{sec:Everblack}{everblack} crystals, which could then be
accessed by any psionic race. The language ``ilian'' was used in combination
with telepathy to allow individual ilians to talk to one another. Most of the
time the psionic powers were used to command, or even force, lower caste
members, thus forming a strict hierarchical society that did not allow
dissenting individuals.

They are one of the great societal architects, city builders and engineers of
Aror. Often their everblack powered machinery, artefacts and contraptions work
to this day, even though many of their cities have fallen to ruin.

\subsubsection{Ramesk}
\label{sec:Ramesk}

Ramesk is a blueish, thick fluid that the ilians drank as nourishment. It has
almost no smell, but tastes salty, due to the high mineral content of the
liquid. It was made out of various roots, herbs and mushrooms that grew in the
depths, and was the sole food provided to Arnaks, the ilian worker caste.
Although the Ilians still had humanoid mouths and could eat and digest other
food, Ramesk turned into the main source of nourishment when food became scarce
during the aeon of strife.

\subsubsection{Arnak}
\label{sec:Arnak}

Arnaks are the labourer and thus lowest caste of ilian society. They stand
roughly two and a half metres tall, their bodies are extremely muscular, and
they have sharp claws and teeth they used for hunting and defending the ilian
cities and outlying farms. They are blind, and rely on their senses of smell
and tremor sense to move about in the dark caves.

The least intelligent of all ilians, and were only used as front line troops,
for carving new tunnels, constructing new buildings or tending to underground
farms. Arnaks were often overseen by a few \emph{Karek}, who could force them
into hibernation if their labour was not needed, or food had to be rationed or
preserved.

The most numerous of all the ilians they can still be found underground today,
often leaderless and organised into small groups. Their fierce strength and
uncanny senses make them expert hunters of anything that ventures or lives down
below.

The fall of major ilian civilisations and cities have left them cut off from
their supply of Ramesk, and have thus resorted to drinking humanoid blood,
which is similar in composition to Ramesk. If no suitable sources of nourishment
are available, Arnaks hibernate beneath stalactites. The mineral rich water
that drops from the stones nourishes them enough while they hibernate.

Arnaks are now a major scourge of anyone who ventures underground, and many
cultures use Arnaks as the ``boogey man'' to frighten and scare children away
from caves.

\subsubsection{Elnak}
\label{sec:Elnak}

Elnak are the tinkerer and crafter caste of the ilians. Highly intelligent,
although shorter than any other ilian. They stood roughly two metres tall, and
had extra-ordinarily sharp eyesight and dexterity to allow them to work on
intricate psionic machinery. They were leaner than other ilians, but made up
for their lack of strength with advanced psionic powers and abilities. They
were responsible for forging all psionic artefacts and machinery used by the
Ilians.

\subsubsection{Karek}
\label{sec:Karek}

Karek are the elite warrior and fighter caste of the ilians. They stood between
two and a half, to three metres tall, and had four arms. Highly intelligent
tacticians, powerful psionics, and above all else fierce fighters. There were
but a few Karek per city or community, as they used Arnaks as the main fighting
force. Karek were often high ranking military leaders, akin to captains or
generals.

\subsubsection{Elmek}
\label{sec:Elmek}

Above all others stood the elmek, the ruling caste of the ilians. Highly
intelligent, and powerful psionics. They lead their cities administratively as
well as spiritually. Elmek were also active in psionic research, and were
brilliant engineers and responsible for leading and building entire ilian
communities and societies. They towered over three metres and a half tall,
so that they could easily be identified as the rulers by all other ilians.

\subsection{Lycanthropes}
\label{sec:Lycanthropes}

Lycanthropes, or were-creatures, are monstrous or humanoids that turn into an
animal shape. Either voluntarily at will, or against their will during a full
moon. As far as modern scholars agree, there are two main types of
lycanthropes: the original, true lycanthropes, and those that
\hyperref[sec:Three Kings]{Miator} has blessed with the gift.

\subsubsection{True Lycanthropes}
\label{sec:True Lycanthropes}

\songquote{Garmarna}{
  Kära du ulver bit inte mig \\
  Linden darrar i lunden \\
  Dig vill jag giva min silversko \\
  Ty hon var vid älskogen bunden \\
  Silversko jag passar ej på \\
  Linden darrar i lunden \\
  Ditt unga liv och blod måst gå \\
  Ty hon var vid älskogen bunden
}

True lycanthropes are immortal (but not invulnerable), and were once cursed by
the \hyperref[sec:Druid]{Druids} for unspeakable crimes against nature and its
inhabitants. The druids cursed them with seeing the world through the very
animals that they had harmed, and are condemned to bring the same harm upon
his very own species. They are forced to turn into were creatures (or half
humanoid, half animal creatures) when one of the moons of Aror are full - so
twice a month, often during the beginning and during the middle of the
month. They retain full awareness and memory of their actions during their
transformation, but are most often unable to control their behaviour. The
curse often cannot be broken through divine magic, except when a horrible or
sadistic condition is met, or when it is lifted by the druid that spoke the
curse.

Since most true lycanthropes are fully aware of their condition, and the
requirements required for their release, they retreat from society to live as
hermits and they often shackle themselves to not hurt others. Those that try
to suppress their urges, often find themselves at odds with society, as their
immortality isolates them from the daily lives of the mortal races. Their
immortality also means that more often than not, the druid that cursed them or
the people they were meant to kill to release the curse, have died a long time
ago, leaving them trapped with the curse.

Some lycanthropes give in to their temptations, and thus also attempt to
fulfil the conditions that would release them from their curse. These
conditions often require sacrifice of other living beings to be broken, making
these lycanthropes a threat to societies and villages.

\subsubsection{Beast Warriors of Miator}
\label{sec:Beast Warriors}

Miator, the god of slaughter and destruction of the \nameref{sec:Three Kings},
also bestows lycanthropy upon his favoured warriors. This state of lycanthropy
is different from \emph{true lycanthropy}, as the warrior retains full control
during the transformation, giving them unspeakable strength and combat prowess.
These \emph{beast warriors}, do not require a full moon to change, and may
change back and forth at will. Unlike true lycanthropes, they are not immortal.

\begin{35e}{Lycanthropes}
  \emph{True lycanthropes} change forcibly into a animal, or hybrid form
  during a full moon. So twice a month, once at the start, once during the
  middle of the month. They retain memories of their time during the
  change, but cannot control the creature they have become, which is always
  \emph{chaotic evil}. They can attempt to stop a single action their animal
  form attempts to make by succeed a \emph{DC: 15 + animal form creature's HD}
  will save. True lycanthropy is a curse, but can only be broken through
  \emph{Wish}, or by fulfilling the condition that the caster of the curse set
  during the ritual. A true lycanthrope is aware of his conditions, and also
  about the condition that must be fulfilled to break the curse.

  \emph{Beast Warriors} behave just like SRD lycanthropes
\end{35e}

\subsection{Týnríkke}
\label{sec:Tynrikke}

The \emph{Týnríkke} (Ancient Teranim for the ``forgotten kingdom''), often
shortened to simply ``týn'', is a large tribe of giant humanoids that live in
the north western part of \nameref{sec:Eilean Mor}, and all over
\nameref{sec:Iafandir}.

Their outward appearance is very similar to humans, but their average size
ranges between 2.5 to 3 metres. Although most other races call them giants
because of their sheer size, they are closer to the core humanoid races, than
to the actual giants and giant races of \hyperref{sec:Aror}{Aror}. The týn
have grey-blue skin, green or blue eyes, often blond or light brown hair.
Much like the other humanoid races, the male týn can grow facial hair.

It is unclear how they are related to the core humanoid races, but due to the
fact that they speak ancient Teranim, and their similarities to humans,
scholars agree that the týn and humanoids must share a common ancestors.

\subsubsection{History}

The ancestors of the týn built magnificent cities and temples tens of thousands
of years ago. This knowledge is now lost to the descendants of the týn, and
those great cities and structures are now nothing but ruins, tombs and dungeons
that dot the valleys, forests, and mountains of \nameref{sec:Eilean Mor} and
\nameref{sec:Iafandir}. Ancient legends of the týn say that their ancestors
fought against the fey and beast races during the \nameref{sec:Aeon of Strife},
and in the process lost their lives, their civilisation and ultimately their
future.

\subsubsection{Society}

The týn believe in tradition, family and personal honour above all else, and
practise a heavily ritualised ancestor worship. Their ancestors built
magnificent cities and temples thousands of years ago, which are now nothing
but ruins, but these sites are still holy to them. The týn often live near
such ruins in tribes, or small villages. Since their ancestors do not grant
divine magic, most skalds (poets, and story tellers), witches and shamans
of the týnríkke are often bards and sorcerers.

In the eyes of the týn their ancestors lost their great civilisation and
future, to aid the other humanoid races in establishing themselves during
the \nameref{sec:Aeon of Strife}. So their descendants, who hold that
sacrifice in great honour, often see the other humanoid races as ungrateful
and arrogant. A belief that is often strengthened when humanoid races attempt
to break into and loot the ancient tombs and catacombs of the týn. Many tribes
protect their holy sites fiercely against any intruders, and only allow those
they trust to visit the ruins.

For a týn his personal honour is his highest value in life. Honour is gained
through the hunt, battle or a life in service to family, tribe or protecting
the ancient sites from intruders. Tokens of honour are often prominently
displayed on the týn's clothing. Such tokens are, for example, teeth of an
animal the týn has hunted, items taken from slain enemies, or gifts from
other týn given to them for services rendered. Within a tribe the most
honoured leads a tribe, and he or she swears to serve the clan and the
ancestors. Should a týn lose his honour, by, say, committing a crime, they
are often banished from the tribe, and move permanently into their ancestor
ruins to protect them from looting and destruction. It is there that a
dishonoured týn can restore his honour, and return to the tribe, by serving
his ancestors. Those that die in honour are embalmed and buried in the holy
sites, next to their ancestors.

The týn are capable of producing iron and steel weapons, as well as advanced
weaponry such as crossbows. Furthermore they are fierce, strong, skilled
warriors and cunning tacticians. The týn fight in formations, and often know
how to use the terrain to their advantage, but consider hit-and-run tactics
dishonouring and thus prefer small skirmishes and open battles.

\subsubsection{Relations}

\graham{Thinking it wise to try to drive the týnríkke off their land, is a
  sure way to ruin your barony.}

Although almost all týn tribes allow other humanoids from visiting their
villages, all foreigners are forbidden from entering their holy sites and
ruins. The týn are willing to trade and interact with humanoids, but hold
fierce hatred against any beast races. Since the týnríkke will not integrate
or mingle with other humanoid villages or cities, and are notoriously hard to
drive off their land, most baronies and kingdoms simply ignore the týnríkke.

\begin{35e}{Týn}
  The týn speak \emph{ancient Teranim}. See the section \nameref{sec:Speak
    Language} for details on available languages and their alphabets.

  \begin{itemize}[noitemsep]
    \item +4 Strength, +2 Constitution
    \item Large size. -1 penalty to Armor Class, -1 penalty on attack rolls,
      -4 penalty on Hide checks, +4 bonus on grapple checks, lifting and
      carrying limits double those of Medium characters.
    \item Space/Reach: 10 feet/5 feet.
    \item An týn's base land speed is 40 feet.
    \item Ancestor Defence (Ex): +4 racial bonus to attack and damage while
      defending sacred ruins and holy sites. As well as a +4 dodge bonus
      against enemies while fighting in these holy sites or ruins.
    \item Fierce Appearance (Ex): +2 racial bonus to intimidate and perform
      checks.
    \item Level adjustment: +2
    \item Favoured class: Any.
  \end{itemize}
\end{35e}

\subsection{Umgeher}
\label{sec:Umgeher}

\aren{I was once the midwife of an \emph{Umgeher}...}

\graham{The less shared about this experience the better.}

\emph{Umgeher} are undead humanoids that retain their own individuality,
will and determination across the process that turned them into undead. They
were created by the ancient vampires that reign in the city kingdom of
\emph{Helmarnock} centuries ago, and were given the freedom to reproduce.
Soon they travelled and spread across the entire known world. Although they
are immortal, they are not invincible. More so, their flesh and skin is in a
constant state of decay, and must combat their never ending dissolution with
oils and ointment. They have no bodily hair, and often wear wigs to blend into
normal population.

\emph{Umgeher}, like any species, are free to determine their own fates and
shape their own destinies, they do have to constantly combat the ignorance and
fear of the other races. Many religious institutions and city states have now
allowed \emph{umgeher} to live there without a fear of being persecuted.
However this is a recent trend, and \emph{umgeher} prefer to build their own
little communities, towns and sometimes even small cities.

\begin{35e}{Umgeher Traits}
  \begin{itemize}[noitemsep]
    \item Medium: as medium creatures, \emph{Umgeher} have no special bonuses or
    penalties due to their size.
    \item \emph{Umgeher} base land speed is 30 ft.
    \item As undead creatures \emph{Umgeher} gain all undead traits.
    \item \emph{Umgeher} have a live long experience in disguising themselves as
    humans, and thus gain a \emph{+2 racial bonus} on \emph{Disguise} and
    \emph{Bluff}.
    \item Favoured Class: Any. When determining whether a multi class takes an
    experience point penalty, his or her highest-level class does not count.
    \item Automatic languages: Teranim. Bonus languages: Any (except secret
    languages).
  \end{itemize}
\end{35e}

\subsection{Vampires}
\label{sec:Vampires}

Vampires are undead, that were once humanoid creatures, mostly elves and
humans that now coexist with their mortal brethren. Those civilised vampires
that have forgone blood in exchange for the liquid \nameref{sec:Ramesk}, have
blue finger nails and hair. Their bodies are very slender, almost skinny, and
their skin comes in various shades of grey or white. Only vampires that drink
Ramesk can see, and their eyes are usually blue, while all other vampires are
blind having only the white remaining in their eyes.

\subsubsection{History}

During the \nameref{sec:Aeon of Strife} many humanoid clans prayed to what
they believed was a greater deity called \nameref{sec:Morana}. She in turn
revealed to her followers a path to great power, and many of her followers
accepted it. They were turned, often without truly knowing the consequences,
into blind and blood hungry vampires.

Many of these old vampires turned from their ``mother'' and were then
threatened by Morana, in turn revealing her to be a lesser deity. Not
intimidated by the threat, many vampires turned from her and were then killed
by Morana for their insolence. Those that survived did so by turning toward
other lesser deities, such as the \nameref{sec:Silent Queen}, who protected
them from Morana's wrath.

The surviving vampires, often without a way to turn back, then continued
what they originally set out to do: aid the humanoids in fighting the
monstrous races. The humanoids often protected the vampires during the day,
while the vampires fought and feasted upon the monstrous races, such as the
bugbears, goblins, kobolds and hobgoblins. Although still a threat to the very
humanoids these vampires attempted to protect, they earned themselves respect,
friendship and often a noble title and land for their efforts in the many wars
during the \nameref{sec:Aeon of Strife}.

After the Aeon of Strife ended, most vampires held on to their titles, lands,
and riches, trying to live peacefully together with the mortal humanoid races.
They often feasted on animals, or only drank blood their subjects willingly
donated to the vampires. Some vampires however instead turned on the humanoids,
feasting on their blood and thus became \emph{feral vampires}.

\subsubsection{Ramesk and Vampires}

In \emph{MI:310} the recipe for \nameref{sec:Ramesk}, a mineral rich blueish
liquid, was discovered by a \nameref{sec:Deepkin} expedition in the depths of
the \nameref{sec:Cnamh Mountains}. It took a few years of experimentation to
adapt the liquid to work as nourishment for vampires, and there were many
troubles in growing the ingredients on farmland on the surface. But now the
liquid is widely used by the vampires as a substitute to humanoid blood. Only
a gulp of Ramesk is required for a vampire to be fed for twenty four hours.

Ramesk has a few other benefits for vampires. It makes their nails and hair
grow again, albeit in the same shade of light blue of the liquid. Although
nail colours and hair dyes exist, blue hair has become an identification mark
for vampires to show to the mortal races that they are civilised and pose no
threat. Ramesk also restores the vampire's eyesight, giving them a blue
pupils. It further reduces their primal longing for humanoid blood, giving
them the ability to explore other feelings such as love, anger or
grief. Another important aspect of Ramesk is that vampires become less
sensitive to sunlight. Vampires that drink Ramesk can walk during the
daylight, as long as they wear clothing that shades them from direct exposure
to the sun.

\subsubsection{Born Vampires}
\label{sec:Born Vampires}

The vampire nobility of \nameref{sec:House deVar} heavily researched the magic
that turned the early humans and elves into vampires. In \emph{MI:755}, they
restored the ability for female vampires to birth children on their own, thus
furthering the vampire race without having to turn mortals. These new vampires
are often called \emph{born vampires}, and are less powerful that the original
vampires. They retain the physical appearance of the mortal races of their
parents, and either look elfish or human. Born vampires reach maturity within
a few years, but often go out and seek adventure, or learn skills and even
professions like mortals do.

\subsubsection{Feral Vampires}
\label{sec:Feral Vampires}

Those vampires that actively hunt, feed and kill humanoid races such as
humans or elves, are called \emph{feral vampires}. The taste of mortal blood
makes them savage, primal and often primitive hunters and killers. They are
far and few, but often cause great harm to towns, cities and kingdoms. Feral
vampires are outcasts among the humanoids, monstrous races and even their
civilised brethren, who see them as a threat to the fragile peace between the
mortal and undead races. Feral vampires often return to a worship of their
mother \nameref{sec:Morana}. Much like the original vampires, feral vampires,
are blind and hunt by scent only.

Although feral vampires could be turned from their savage ways by force
feeding them Ramesk, very few undead hunters or civilised vampires
bother. Feral vampires are often hunted and destroyed without prejudice.

\begin{35e}{Vampire}
  Vampires speak whatever language their given by their region or heritage,
  and are not evil by default. Feral vampires have been driven mad by drinking
  humanoid blood, and are usually evil. All vampires, regardless of type, gain
  the \emph{Scent} ability and turn blind, unless they drink Ramesk once every
  twenty four hours.

  Vampires are not afraid of holy symbols, garlic or mirrors, but still take
  penalties from direct exposure to sunlight. Those that drink Ramesk can
  venture into the broad daylight, given that they wear clothing that shields
  their skin from direct exposure. Vampires may also cross running water
  without problems, and may enter any house without being invited.

  \textbf{Born Vampires}:\\
  \begin{itemize}[noitemsep]
    \item Medium: Born vampires are medium creatures, and thus have no special
      bonuses or penalties due to their size.
    \item Type: Undead. Born vampires gain all undead traits, and thus have no
      constitution score.
    \item Feral Blindness (Ex): A born vampire turns blind if he goes without
      Ramesk for twenty four hours.
    \item Blood Drain (Ex): A born vampire can suck blood from a living victim
      with its fangs by making a successful grapple check. If it pins the foe,
      it drains blood, dealing 1d4 points of Constitution drain each round the
      pin is maintained. On each such successful attack, the vampire gains 5
      temporary hit points.
    \item Energy Drain (Su): Living creatures hit by a born vampire's slam
      attack (or any other natural weapon the vampire might possess) gain one
      negative levels. For each negative level bestowed, the born vampire
      gains 5 temporary hit points. A vampire can use its energy drain ability
      once per round.
    \item Scent (Ex): Scent lets a creature detect approaching enemies, sniff
      out hidden foes, and track by sense of smell.
    \item Turn Resistance (Ex): A born vampire has +4 turn resistance.
    \item Level Adjustment: +4
  \end{itemize}
\end{35e}


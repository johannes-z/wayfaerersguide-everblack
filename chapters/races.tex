\twocolumn
\section{Races}

The people of Aror make up a multi flavoured stew of various races and
backgrounds. Apart from may races that you might already know from your
home world, such as humans, elves, dwarves or gnomes; Aror is also home to some
you might have never heard of before. Two of these are the \emph{deepkin}, the
\emph{diarim} and the \emph{umgeher}.

You will find that race matters little. An elf of \emph{Norbury} often shares
the warrior culture, and faith in the meritocratic societal ideas
of \emph{Norbury}. While a dwarf of \emph{Nen-Hilith} will be accustomed and
to the artistic culture and creative ways of the artisans and performers of
that city.

The union of non-humanoid races with humanoid races is something that does
not exist in \emph{Everblack}. Although you might have met ``dragonborn''
(the unity of a human and a dragon) in other realms, you will not find them
on Everblack. And if you do, they are probably extra-planar visitors much
like yourself.

\begin{note}
These sections enhance or change the existing core races of D\&D. Any lore you
find in the base books that does not interfere with what is written here, still
holds true.
\end{note}

\subsection{Early History}

Roughly forty thousand years ago an ice age covered most of the world
of \emph{Aror} in vast sheets of ice and snow. Continents that were separated
by sea, where now suddenly accessible through thick sheets of ice. This
allowed the four core humanoid races, the humans, elves, dwarves and halflings
to settle the entire known world. It is still unclear where they originated
from exactly, but the common scientific consensus is that these races trace
their origins back to the lush rainforests, steppes and savannah
of \emph{Arania}.

When the humanoid races arrived in the other continents they found that they
were already inhabited by sentient races. Ogres, minotaurs, trolls, fey,
hobgoblins, gnolls, bugbears already called these continents their home,
alongside non-sentient monstrous species such as hydras, manticores and
wyverns. Through ancient stories of the dwarves, early writings, ancient stone
tablets and even cave paintings, it was revealed that the humanoid ancestors
were in a constant state of conflict with these monstrous races. Wars,
skirmishes, and often the destruction of entire early villages and even cities
was a common occurrence in early history.

The ultimately successful survival strategy to deal with such a hostile
environment ingrained itself in the culture of these races. The dwarves went
underground and organised themselves in strict hierarchical clans and cities
that allowed them to optimise their societies to the harsh realities and lack
of resources of the depths. Some humans and elves followed the dwarves
underground but ultimately failed to replicate the dwarven's success (with a
few notable exceptions). Although the \emph{deepkin} - the underground
dwelling cousins of the humans - built large civilisations underground they
were ultimately defeated by the sentient races of the depths and driven to the
surface. The dark elves instead relied on small clans, families and by being
constantly on the move to ensure the survival of their species.

Meanwhile on the surface elves and halflings sought to settle as far away from
the monstrous races as possible, leading them to the continent of \emph{South
Goltir} and \emph{Farlar}, as well as becoming the pale elves by settling the
vast ice sheets of the north and south poles. Humans on the other hand used
their ingenuity and skill to build great civilisations and cities that could
potentially withstand the skirmishes and sieges of the sentient monstrous
races. Through many iterations over the course of thousands of years, which
resulted in countless destroyed and ransacked cities and fallen civilisations,
humans have now achieved the unthinkable: dethrone the monstrous races as the
predominant species across all of Aror.

Now many elves and halflings have joined the human effort of building large
centres of civilisation, enriching the predominantly human city kingdoms that
dot the world of \emph{Aror}. Their struggle against the sentient monstrous
races is far from over, especially in the central regions of the continents,
or from the still predominantly monstrous continent of \emph{Iâfandir}. The
majority of dwarves have remained underground, continuing their strict ways of
life as it has served them for centuries. A success that was not shared by the
dark elves and deepkin, who have mostly abandoned the deep and returned to the
surface.

\subsection{Humans}

\emph{Humans} are one of the most ancient humanoid races of \emph{Aror}, and
also the dominant race of the planet. Human artefacts have been found dating
back hundreds of thousands of years, far beyond the history of any other
humanoid species. There are two separate ``races'' of humans: those inhabiting
the southern part of the hemisphere who usually have darker skin, and the
``northerners'' who usually have fair skin. This distinction is superficial
only. Apart from the tone in skin colour, there is no other biological
difference between the various human tribes and civilisations. Humans live up
to 80 years, and are known for being statesmen, diplomats, farmers,
adventurers, explorers and scientists.

\subsubsection*{Language}

Humans speak \emph{Teranim}, either as their primary language, or as their
secondary language together with their local language. \emph{Teranim} has
become the de-facto language of \emph{Aror} and is usually spoken almost
everywhere, even among the beast races. \emph{Teranim} is written in its own
alphabet of the same name. \emph{Old Teranim} and \emph{Ancient Teranim} exist,
spoken by the ancient humans, and are the root of most other languages and
various local dialects. Although \emph{ancient teranim} is no longer actively
spoken, various books, poems, stories and songs still exist in that language.

Humans living in the southern hemisphere often speak a language that is
stuck halfway between old and new \emph{Teranim}, called \emph{Kalest}. While
the people of \emph{Forsby} (and surrounding regions) have their own distinct
dialect, which can be so hard to understand that it has received its own
classification and name: \emph{Reatham}.

Albeit many local dialects exist, almost all humans, and the other races
living with them are capable of speaking \emph{Teranim}. It is, after all,
the official language of many governments, the language that is printed and
used in official capacity as well as in inter-kingdom cultural exchange and
trade.

\begin{35e}{Common in Aror}
The language of \emph{Teranim} is equal to \emph{Common} of D\&D.
\end{35e}

\subsubsection*{Human Lands}

Humans can be found everywhere on Aror. But history indicates that they
originated on the southern continent of \emph{Arania}, and migrated to all
other continents during the last ice age tens of thousands of years
ago. Wherever humans settle they build villages, cities, and large kingdoms
and often become the dominant culture and social structure. Human kingdoms
have often endured for thousands of years, and have bested many difficulties
that had driven other societies to ruin. The ingenuity of the humans, their
stubborn attitude and their uncanny ability to adapt to any difficulty makes
them the dominant race of \emph{Aror}.

\subsubsection*{Human Culture}

Like most races, humans have no inherent global culture, tradition or customs.
Instead their believes and customs are ever evolving, and specific to the
realm they live in. But there is one trait that the average human has: ingenuity
in the face of adversity. No other species has managed to settle every corner
of the world and endure, and even build lasting civilisations out of the
hostile environment they found themselves in.

\begin{35e}{Human Traits}
  \begin{itemize}[noitemsep]
  \item Medium: As Medium creatures, humans have no special bonuses or
    penalties due to their size.
  \item Human base land speed is 30 feet.
  \item 1 extra feat at 1st level.
  \item 4 extra skill points at 1st level and 1 extra skill point at each
    additional level.
  \item Automatic Language: Teranim. Bonus Languages: Any (other than secret
    languages).
  \item Favoured Class: Any. When determining whether a multiclass human takes
    an experience point penalty, his or her highest-level class does not count.
  \end{itemize}
\end{35e}

\subsection{Deepkin}

\aren{I had the misfortune of seeing my people's decline with my own eyes.
I was powerless, and unable to stop it. We are but a shadow of our former
selves...}

\graham{But still you live. And once the time is right, you may reclaim what
once was.}

The \emph{deepkin} are the cavern dwelling cousins of humans. They are capable
of seeing in the dark, are however colour blind, and often have white pale
skin, red to brownish hair, and either red, green, blue or yellow eyes.

\emph{Deepkin} society is one of the older societies on \emph{Aror}, with a
long standing history in the arcane arts, building magnificent underground
cities, libraries and workshops. Ancient \emph{deepkin} were master arcane
smiths, golem constructors, inventor of many magic based constructs and
technology still used today.

The ancient deepkin had their dominance challenged by the other races
of the deep, namely the dwarves, dark elves, and a now extinct species called
the \emph{ilians}. Thousand of years of conflicts lead to the decline of their
magnificent ancient civilisation. Instead of perishing however, they fled to
the surface, where they were warmly received by their surface cousins. Nowadays
\emph{deepkin} culture is but a shadow of what it was, although they still try
to reclaim what was once theirs with expeditions into the deep.  Albeit they
were well known city and kingdom builders, only one deepkin settlement rose to
the rank of a formidable city kingdom: \emph{Stenheim}.

Although \emph{deepkin} are treated as their own race, they are still capable
of producing viable offspring with normal humans. The race of the offspring is
always that of the mother.

\begin{35e}{Deepkin Traits}
  \emph{Doresh} is the natural language of the Deepkin, and is equivalent to
  \emph{Undercommon}.

  \begin{itemize}[noitemsep]
    \item Medium: as medium creatures, \emph{Deepkin} have no special bonuses or
    penalties due to their size.
    \item \emph{Deepkin} base land speed is 30 ft.
    \item For all manners regarding racial restrictions or classifications
    \emph{Deepkin} count as humans.
    \item Dark vision out to 120 feet, but colour blind as a result.
    \item Bonus Feat: Just as their human cousins, \emph{Deepkin} can choose a
    bonus feat at first level.
    \item Automatic languages: Doresh, Teranim. Bonus Languages: Any (except
      secret languages)
    \item Favoured Class: Any. When determining whether a multi class takes an
    experience point penalty, his or her highest-level class does not count.
  \end{itemize}
\end{35e}

\subsection{Elves}

Elves are tall - often between 1.90 and 2.40 metres - tall slender race, with
long and pointy ears. The style of the elven ears varies, with some having
smaller pointy ears facing backwards, while others have longer and sharper
ears that follows the contours of their face. The variety within the elven
ears is vast, but is but a minor cosmetic difference. Their physique and build
is slender as well, with long skinny legs and arms. Elves live up to 400 years
of age, and thus often pick up professions that take longer to master, such as
wizardry, artistry or the sciences. Although elves live long, they often shift
focus in their goals. Apart from the humans, the elves are one of the older
races of \emph{Aror}.

Unlike humans, elves do have distinct sub races that differ from each other in
various physical and biological aspects. The \emph{snow elves} for example
have a natural resistance to cold like no other species have, while the
\emph{dark elves} have adapted to see better in the dark caverns they
inhabit. However they have not diverged so far from one another to not produce
vialable offspring. As with humans and deepkin the race of the sibling always
resembles that of the mother.

There are five major elven races that are recognised across the world of
\emph{Everblack}:

\subsubsection*{High Elves}

The most numerous are the high elves of \emph{Avenfjord}. High elves have
fair skin with a hint of yellow and gold. Their hair ranges from blond, fiery
red and black. High elves are the most adaptable and curious of the
elves, and often live within human city kingdoms, adapting and integrating
well with other cultures and societies. Even though they have their own
kingdom, the vast majority of high elves live outside the kingdom
of \emph{Avenfjord}. High elves were the only civilised elven race for
quite a while, and often have a long standing history living together with
their distant human cousins within baronies, kingdoms and city states.

High Elves speak \emph{Enro'ad}, a variant of \emph{Old Teranim}, but use the
halfling alphabet to write it.

\begin{35e}{High Elf Traits}
  \emph{Enro'ad} is elvish, albeit the elven alphabet is now the \emph{halfling}
  script.

  \begin{itemize}[noitemsep]
    \item Medium: As Medium creatures, elves have no special bonuses or
    penalties due to their size.
    \item Low-Light Vision: An elf see twice as far as a human in starlight,
    moonlight, torchlight, and similar conditions of poor illumination. She
    retains the ability to distinguish colour and detail under these
    conditions.
    \item Automatic languages: Enro'ad, Teranim. Bonus Languages: Any (except
      secret languages)
    \item Favoured Class: Any. When determining whether a multi class takes an
    experience point penalty, his or her highest-level class does not count.
  \end{itemize}
\end{35e}

\subsubsection*{Dark Elves}

The \emph{dark elves} live underground, have black to blue skin, and their
hair ranges from a faint hint of blue, silver to snow white. Their eyes are
often red, blue or a bright yellow. They are the smallest of all elven races,
and range from 1.70 to 1.90 metres in height. In terms of bodily physique
they also more closely resemble humans, rather than their high elven cousins.

They are often live primitive, nomadic lives underground, adapted to the harsh
realities of the depths below. They have survived the harsh conditions of the
deep caverns by remaining on the move or by hiding from threats. They value
their small communities and family above all else. Many dark elves also live
on the surface, and much like their surface cousins, they blend in well with
human societies very easily.

Dark elves are no kingdom builders and instead wish to emulate their fair
skinned brethren that have managed to integrate well into human kingdoms. Their
strong loyalty to their community makes them a welcomed member of any smaller
towns and villages they join.

\begin{35e}{Dark Elf Traits}
  \textbf{Dark Elf Traits (Ex)}: The following traits are in \emph{addition}
  to the high elf traits, except when noted.
  \begin{itemize}[noitemsep]
    \item Dark vision out to 120 feet, albeit they are colour blind.
    \item Automatic languages: Doresh, Enro'ad, Teranim. Bonus Languages: Any
      (except secret languages)
    \item Favoured Class: Any. When determining whether a multi class takes an
    experience point penalty, his or her highest-level class does not count.
  \end{itemize}
\end{35e}

\subsubsection*{Snow Elves}

\aren{Snow Elves are the pinnacle of beauty...}

Far to the north and south live the \emph{snow elves}, nomadic hunter-gatherer
elves with white to silver blue skin, white or blue hair, and bright blue,
green or yellow eyes. Male snow elves are capable of growing facial hair. Snow
elves are as tall as their high elven brethren, ranging from 1.8 to 2.3. They
have adapted well to the colder climates, and are expert hunters and weapon
smiths. Many \emph{snow elves} live in smaller families and tribes, content
with surviving the harsh realities of the polar north and south by becoming
fierce hunters or even raiders themselves. Along with the
\emph{wood elves}, they are rather rare in human societies.

Snow elves are generally known to be calm, and quiet, preferring the solitude
of a small group or town over the vast stretches of cities. They rarely leave
the frozen north and south, and are thus exotic, as in, not many other
humanoids have seen a snow elf in person.

They do not build city or kingdoms, and prefer to live in small tribes and
villages in frozen north and south. Snow elves rarely leave their icy domain,
and most ancestors of the snow elves in the city kingdoms were raided,
captured and brought there as slaves. Those civilised snow elves do not
harbour any animosity any more about what happened to their ancestors, and
prefer to remain in the places and cultures they know call their home.

\begin{35e}{Snow Elf Traits}
  \textbf{Snow Elf Traits (Ex)}: The following traits are in \emph{addition}
  to the high elf traits, except when noted.
  \begin{itemize}[noitemsep]
    \item \textbf{Pale Wastes (Su)}: A pale elf can live comfortably in
    conditions of extreme cold, even with barely any clothing or external
    sources of warmth. This ability functions like a continuous \emph{Endure
    Elements} but for cold conditions only.
    \item Automatic languages: Enro'ad, Teranim. Bonus Languages: Any (except
      secret languages)
    \item Favoured Class: Any. When determining whether a multi class takes an
    experience point penalty, his or her highest-level class does not count.
  \end{itemize}
\end{35e}

\subsubsection*{Wood Elves}

Wood elves have light brown skin, green or red hair. Their faces are often
covered in light or red freckles, and their eyes are often brown, blue or
green. They are as tall as their high elven counter parts, often ranging from
2 metres to 2 metres 40. Although they are only superficially different from
their high elven cousins, they are counted as their own race based on their
different culture in the areas they have settled. Even though they are known
to join the major city kingdoms, they are rarer than high elves, dark elves or
snow elves.

There are two major tribes of wood elves. The first lives in the vast
temperate boreal forest of \emph{Eilean Mor}, known as the \emph{Dirgewood}.
These wood elves live together with humans, halflings of the Dirgewood, as
well as the dark elves, deepkin and the dwarves of the \emph{Great Divide}.
They are followers of the \emph{old ways}, and prefer to build small
settlements, towns and perhaps tiny cities of their own. These wood elves are
expert trackers, farmers, as well as druids and priests of the old ways.

And in the jungles \emph{Yua'cata} live the \emph{savage elves}, a loose
collection of tribes of cannibalistic and demon worshipping wood elves. They
rarely wander beyond the confines of their jungle and are one of the rarest
elves to meet in the civilised areas of \emph{Aror}. They live a primitive
life, adapted to the harsh environment in the depths of the rain forest.

\begin{35e}{Woold Elf Traits}
  \textbf{Wood Elf Traits (Ex)}: The following traits are in \emph{addition}
  to the high elf traits, except when noted.
  \begin{itemize}[noitemsep]
    \item Automatic languages: Enro'ad. Bonus Languages: Any (except secret
      languages)
    \item Favoured Class: Any. When determining whether a multi class takes an
    experience point penalty, his or her highest-level class does not count.
  \end{itemize}
\end{35e}

\subsection{Half-Elves}

The union of human and elf was possible in the past, but since then the elves
and humans diverted too far apart to produce viable offspring. Most children
that are now born in the union between elf and men are often born with birth
defects, with severe mental disability, if they - and the mothers giving
birth - survive birth at all.

This was not always the case however, and tens of thousands of years ago the
union was possible. It produced a sizeable amount of half elves back then,
which over the course of the recent history struggled to keep their populace
alive and growing. Constant inter-family marriages over the course of
thousands of years have created a tight-knit clan of half elves that span the
globe. Although they form no kingdoms of their own, they have created a
secondary culture that focuses heavily on the continuation of their
people. Half-elf lore keepers keep meticulous record of what half-elven family
lives where, and with whom they are related. Half-elves are encouraged to
marry within their own species, as introducing either elven or human blood
would ``weaken'' the half-elf population.

\subsection{Dwarves}

\emph{Dwarves} are natural fighters, miners and smiths. Among all of the races
they are the most reclusive of all. Dwarves usually stand between 1.2 and 1.4
metres high, but are on average almost has heavy as humans. They often live
up to 250 years. Their skin is usually tan brown, their skin ranges from brown
to black, and they often have brown eyes to match. For most male dwarves the
beard is a symbol of status, and often stands as a symbol for that particular
dwarves caste within the clan. Female dwarves do the same, but with their
head hair.

\subsubsection*{Dwarven Culture}

They hardly join other cultures, and prefer to continue to live like their
ancestors did. \emph{Dwarves} organise themselves into huge clans or families
and live in deep caverns or mountains in a strict caste society. Those that do
not fit into these strict frameworks of societies are cast out, and then look
for other realms to live. Ever since the deep have become ever more dangerous
many smaller dwarven clans have fled the surface near human or elven
settlements, while others have seemingly integrated into the large city
kingdoms. Offering their expertise on trade and smithing to the other races.

Dwarves speak their own language called \emph{Rutari} with its own alphabet of
the same name.

\subsubsection*{Caste System}

Most dwarven clans strictly enforce their caste society to maintain order and
control within their societies. Anyone who does not fit within this system is
met with suspicion, contempt or exclusion and exile at the worst. The lowest
of the caste are slaves, which are drawn from the pool of criminals among the
dwarves, as well as from the other races that live within the dwarven society.
Other races, such as humans, elves, or even \emph{gnomes}, often have no chance
to advance out of the slave caste. Most dwarven clans however allow visitors
into their strongholds.

\begin{35e}{Dwarf Traits}
  \begin{itemize}[noitemsep]
    \item Medium: As Medium creatures, dwarves have no special bonuses or
      penalties due to their size.
    \item Dwarf base land speed is 20 feet. However, dwarves can move at this
      speed even when wearing medium or heavy armor or when carrying a medium or
      heavy load (unlike other creatures, whose speed is reduced in such
      situations).
    \item Stability: A dwarf gains a +4 bonus on ability checks made to resist
      being bull rushed or tripped when standing on the ground (but not when
      climbing, flying, riding, or otherwise not standing firmly on the
      ground).
    \item Languages: Rutari, Teranim. Bonus Languages: Any (except secret
      languages)
    \item Favoured Class: Any. When determining whether a multi class takes an
          experience point penalty, his or her highest-level class does not
          count.
  \end{itemize}
\end{35e}

\subsection{Halflings}

Most of the halflings are nomads, who travel across the lands in search for
places they could explore. If you think you have found a secluded spot on
\emph{Aror}, where no one else had sat foot before, you can be sure it is
already named after an halfling explorer. They travel in small families, and
in exchange for money offer their services to any small town they come across
on their travels. Halflings are curious, adventurous often find themselves
exploring the depths and other inhospitable places of \emph{Aror}.

A huge group of halfling families once decided to start another adventure: of
their own kingdom. They settled down near an elven kingdom and named their new
kingdom \emph{Brèagha Hilith}. After decades of peaceful coexistence between
the two races, they finally decided to tear down the last barriers and simply
merge the two kingdoms into a new one: \emph{Nen-Hilith}. It became a shining
beacon of civilisation, artistry and stability under the known city kingdoms
of \emph{Aror}.

\aren{Until a few giants stepped on them...}

\begin{35e}{Halfling Traits}
  \begin{itemize}[noitemsep]
    \item Their small stature makes it harder for them to acquire muscle mass
      that would compete with their taller humanoid brethren, and thus have
      -2 racial penalty on strength.
    \item Small: As a Small creature, a halfling gains a +1 size bonus to
    Armor Class, a +1 size bonus on attack rolls, and a +4 size bonus on Hide
    checks, but she uses smaller weapons than humans use, and her lifting and
    carrying limits are three-quarters of those of a Medium character.
    \item Halfling base land speed is 20 feet.
    \item Automatic Languages: Old Teranim, Teranim. Bonus languages: Any
    (except secret languages)
    \item Favoured Class: Any. When determining whether a multi class takes an
          experience point penalty, his or her highest-level class does not
          count.
  \end{itemize}
\end{35e}

\subsection{Umgeher}

\aren{I was once the midwife of an \emph{Umgeher}...}

\graham{The less shared about this experience the better.}

\emph{Umgeher} are undead humanoids that retain their own individuality,
will and determination across the process that turned them into undead. They
were created by the ancient vampires that reign in the city kingdom of
\emph{Helmarnock} centuries ago, and were given the freedom to reproduce.
Soon they travelled and spread across the entire known world. Although they
are immortal, they are not invincible. More so, their flesh and skin is in a
constant state of decay, and must combat their never ending dissolution with
oils and ointment. They have no bodily hair, and often wear wigs to blend into
normal population.

\emph{Umgeher}, like any species, are free to determine their own fates and
shape their own destinies, they do have to constantly combat the ignorance and
fear of the other races. Many religious institutions and city states have now
allowed \emph{umgeher} to live there without a fear of being persecuted.
However this is a recent trend, and \emph{umgeher} prefer to build their own
little communities, towns and sometimes even small cities.

\begin{35e}{Umgeher Traits}
  \begin{itemize}[noitemsep]
    \item Medium: as medium creatures, \emph{Umgeher} have no special bonuses or
    penalties due to their size.
    \item \emph{Umgeher} base land speed is 30 ft.
    \item As undead creatures \emph{Umgeher} gain all undead traits.
    \item \emph{Umgeher} have a live long experience in disguising themselves as
    humans, and thus gain a \emph{+2 racial bonus} on \emph{Disguise} and
    \emph{Bluff}.
    \item Favoured Class: Any. When determining whether a multi class takes an
    experience point penalty, his or her highest-level class does not count.
    \item Automatic languages: Teranim. Bonus languages: Any (except secret
    languages).
  \end{itemize}
\end{35e}

\subsection{Diarim}

\aren{The expression \emph{explaining freedom to a Diarim} has become
ingrained in many cultures as a metaphor for a fruitless labour.}

The \emph{Diarim} are a race of humanoid creatures that were bread for
specific tasks by the dark sorcerers and witches of the giants living on the
continent of \emph{Farlar}. They are the youngest of the races on \emph{Aror},
and are a mixture of various other humanoid races.

\emph{Diarim} come in in as many shapes and sizes as the giants had uses and
tasks for their engineered slave race. But most share a common set of features:
the light, fair skin of the \emph{Deepkin}, blue hair of the \emph{Snow elves},
and an ingrained sense of duty and loyalty to strict hierarchies from the
dwarves. Very few \emph{Diarim} ever escape the slavery of their masters, and
those few that do, find it hard to shake their eagerness to serve, please and
help that the giants have ingrained into them. Almost all have blue tribal
tattoos all over their body, which identify the current and past owners of the
individual \emph{diarim}.

The most common variant are labourers, small but stout breed, that was created
by introducing more dwarven heritage into the \emph{diarim}. They excel at
physical labour, such as mining and construction. Siegers were bread with
monstrous races, often even giants, and are used as front line soldiers,
gladiators and shock troops. They are larger than any other \emph{diarim}, and
often also serve as slave overseers over the others. Exciters were bred and
selected for their beauty, and are priced possessions to be traded and gifted
to other giants. Their primary role is to entertain their overlords through
song, dance and company.

Most \emph{diarim} have no names. They never refer to themselves with names,
and are only being given a names by the giants if they have distinguished
themselves, either through heroic deeds, or through crimes.

In the recent decades more and more \emph{diarim} have escaped the continent
of \emph{Farlar} and joined the other humanoid races. The giants had to learn
that you cannot suppress the curiosity, love for wandering and freedom for long
when you create a species based on humans, halflings and elves.

\begin{35e}{Diarim Traits}
  \textbf{Diarim Traits (EX)}:
  \begin{itemize}[noitemsep]
    \item Medium: as medium creatures, \emph{Umgeher} have no special bonuses or
    penalties due to their size.
    \item A \emph{Diarim}'s base land speed is 30 ft.
    \item \textbf{Weak Will (EX)}: All \emph{diarim} have a -2 penalty to will
    saves against charms and similar effects.
    \item Automatic languages: Giant, Teranim
    \item Favoured Class: Any. When determining whether a multi class takes an
    experience point penalty, his or her highest-level class does not count.
  \end{itemize}

  \textbf{Sieger Traits (EX)}: the following traits are in \emph{addition} to
  the \emph{diarim} traits, except when noted otherwise.
  \begin{itemize}[noitemsep]
    \item Large size. -1 penalty to Armour Class, -1 penalty on attack rolls,
    -4 penalty on Hide checks, +4 bonus on grapple checks, lifting and
    carrying limits double those of Medium characters.
    \item Space/Reach: 10 feet/5 feet
    \item +8 Strength, -2 Intelligence, -2 Charisma, -2 Wisdom
    \item Favoured Class: Barbarian
    \item Level Adjustment: +1
  \end{itemize}

  \textbf{Exciter Traits (EX)}: the following traits are in \emph{addition} to
  the \emph{diarim} traits, except when noted otherwise.
  \begin{itemize}[noitemsep]
    \item -2 Strength, -2 Constitution, +4 Charisma
    \item Favoured Class: Bard, Sorcerer
  \end{itemize}
\end{35e}

\subsection{Gnomes}

\emph{Gnomes} are the children of \emph{halflings} and \emph{dwarves}. Since
they have no identity of their own, no culture of their own, they often try to
integrate with their parents' cultures. Albeit they rarely fit in with either
societies. Gnomes themselves are sterile, and thus rarely create their own
families, they have inherit the longevity of their dwarven parent. All in
all \emph{gnomes} are among the rarest races on \emph{Aror}. They often
dedicate their lives to adventuring and other dangerous businesses, and rarely
settle down.

There are no \emph{gnome} settlements or even kingdoms, and most of them live
scattered across the world in the large city kingdoms. Offering their services
as spies, thieves and adventurers to anyone willing to pay. Most large dwarven
cities and halfling settlements also have a small \emph{gnome} population.

\begin{35e}{Gnome Traits}
  \begin{itemize}[noitemsep]
    \item Small: As a Small creature, a gnome gains a +1 size bonus to Armor
    Class, a +1 size bonus on attack rolls, and a +4 size bonus on Hide
    checks, but he uses smaller weapons than humans use, and his lifting and
    carrying limits are three-quarters of those of a Medium character.
    \item Gnome base land speed is 20 feet.
    \item Low-Light Vision: A gnome can see twice as far as a human in
    starlight, moonlight, torchlight, and similar conditions of poor
    illumination. He retains the ability to distinguish color and detail under
    these conditions.
    \item Languages: Teranim, Rutari. Bonus Languages: Any (except secret
       languages)
    \item Favoured Class: Any. When determining whether a multi class takes an
    experience point penalty, his or her highest-level class does not count.
  \end{itemize}
\end{35e}

\subsection{Half-Orc}

Less rare than \emph{Gnomes}, but equally outcasts in most places, are the
offspring of the union of human, deepkin or elf with an orc. Much like other
half breed races \emph{half-orcs} are sterile, and thus have no inner need or
desire to settle down or start families. They often live amongst their human
parents offering their brute strength and enduring physique as heavy labourers
or fighters. A human mother giving birth to a \emph{half-orc} has a high
chance of dying during child birth. And since very few women take such a risk,
many \emph{half-orc} children are not the result of a voluntary union. This
sad reality, combined with a short temper, less than flattering appearance and
low living status and conditions, often elicit condescending or outright
demeaning behaviour from the other races towards \emph{half-orcs}.

\begin{35e}{Half-Orc Traits}
  \begin{itemize}[noitemsep]
    \item Their orcish blood makes it easier for half-orcs to gain muscle mass
      and thus have +2 racial bonus to Strength.
    \item Medium: As Medium creatures, half-orcs have no special bonuses or
      penalties due to their size.
    \item Half-orc base land speed is 30 feet.
    \item Orc Blood: For all effects related to race, a half-orc is considered
    an orc.
    \item Automatic Languages: Teranim, Orc. Bonus Languages: Any (except secret
      languages)
    \item Favoured Class: Any. When determining whether a multi class takes an
    experience point penalty, his or her highest-level class does not count.
  \end{itemize}
\end{35e}

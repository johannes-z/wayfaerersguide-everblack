\section{Regions}

Although civilisation can mostly be found in the large city kingdoms, it does
not mean that the more rural areas are filled with savages. Many big regions
share similar cultures or backgrounds, and it is not uncommon to find people
that have moved out of their city kingdoms into these areas. Either to start a
simpler live as farmers, or because a business opportunity or the love of
their life has drawn them there.  Likewise many people move from the country
side to the big cities in the hopes of finding better opportunities, such as
jobs and education there.

The following chapter of this book should introduce you to several regions of
Aror, and its culture.

\subsection{Dirgewood}

The \emph{Dirgewood} is a vast forest south east of the \emph{Great Divide}, a
large mountain chain that splits the continent of \emph{Eilean Mor} in half
north by south. It runs along most of the southern side of the Great Divide,
reaching up the mountains until the tree border, but does not extend to the
shores. It is a vast temperate and boreal forest and marshland, that is split
by a few major rivers that have their origins in the Great Divide. The thick
woods, harsh and unwelcoming marches, and hilly terrain has made it almost
impossible to build large castles in the Dirgewood. However it is dotted with
thousands of smaller towns and villages.

The Dirgewood is mostly settled by humans, wood elves, snow elves, and
halflings. The hilly regions of the Dirgewood are also home to deepkin, and
dark elves. Many monstrous tribes remain, especially hobgoblins, bugbears,
kobolds and goblins as well as many tribes and packs of lycanthropes. The
Dirgewood is also known for still housing many faeries and fey.

Albeit the humanoid villages are as diverse as any city they have two things
in common: Most villages know of each other, and even know people from other
villages. They trade with each other often, intermarry, and also come to each
other's aid in case of an attack. Most of the time they fight monstrous races
and beasts, but have also killed bandits that settled in their land or halted
the expansionist dreams of baronies that board the Dirgewood. Skirmishes,
feuds and even wars between the villages are known to happen, but are far and
few in between.

The other is a staunch belief and adherence to the tradition of the
\emph{Old Ways}. Almost all villages are lead by a village elder, a druid, and
a war chief. Most villages and small hamlets celebrate the rituals,
incantations, spells and sacrifice demanded by them by the three mothers, and
view outsiders that do not believe in the old ways as suspect. They harbour an
open animosity against anyone that would follow a lesser deity, especially
those that would come to the Dirgewood looking to expand their congregation.

Although the old traditions are part of their religion and culture, there exist
a great variety on how strictly they are followed among the various tribes.
For example many no longer sacrifice humanoids to \emph{Marwaid}, or do so
rarely. Also many tribes allow their women to become anything that they want,
instead of being pressured into a life of servitude to the family and the
household. Some clans still preach these old traditions strictly however,
often leading to a spiritual conflict between the tribes and villages that
would interpret the old ways differently.

The many rivers that flow from the mountains to the south-eastern shore lines of
\emph{Eilean Mor} are used to transport people and goods in and out of the
Dirgewood. The villagers use the river network to trade with each other, raid
monstrous encampments and tribes, as well as communicating and trading with
the large city kingdoms that reside on the shores of Eilean Mor.

For many city dwellers that follow the three goddess' of the old ways, the
Dirgewood is a popular destination for a pilgrimage and spiritual
enlightenment. The people of the Dirgewood are known to be highly spiritual,
and to them the stories, spirit worship, chants and spells of Old Ways are
part of their daily lives. They welcome any humanoid species that also follows
the Old Ways, or at least one of the three mothers, and adheres to their rules
and customs.

The people of the Dirgewood are known as hardy survivors, hunters, raiders,
warriors and above all else spiritually enlightened people that take great
care in preserving their aeon old traditions and believes.

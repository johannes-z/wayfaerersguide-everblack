\section{Adventuring Equipment}
\label{sec:Adventuring Equipment}

\subsection{Clothing}
\label{sec:Clothing}

The kind of clothing worn throughout the city kingdoms is similar to that
offered in the \emph{``Player's Handbook''}, from commoner to artisan clothing
to royal garbs. But on Aror the clothes do make the person. People are expected
to dress themselves according to their wealth, and will infer your social
status from the clothes you wear. So do not be puzzled if people in
\nameref{sec:Norbury} mistreat you if you appearance is unkempt and you wear
linen rags like a slave.

\subsection{Food, Drink, And Lodging}
\label{sec:Food}

Food, drink and lodging is offered in most towns and even small hamlets by
the local inn. One can expect to find inns and taverns almost everywhere on
Aror, where most of them cater to a specific demographic. It is not uncommon
to find slave only taverns, as well as taverns where slaves or even monstrous
races are forbidden from entering.

Most inns serve what is called a never-ending stew, a broth that has been
cooking for several weeks, if not months and is replenished in the morning
with fresh ingredients. It is served with bread and watered ale. Most inns
also serve a cold platter of dried meat, bread and cheese. While the southern
regions and \nameref{sec:Forsby} has a long tradition in making and enjoying
tea, while the northern regions prefer coffee. Coffee culture, with lots of
small coffee shops that also serve food and snacks are popular especially
in \nameref{sec:Hraglund}.

Slaves that are not regularly feed by their owners often make their own
secret taverns and inns. These then serve food scraps, water and other low
quality food for free to other slaves. Although these inns and taverns are
often illegal, they are more often than not tolerated and sometimes even
receive food from priests and temples that seek to do good.

A common low quality slave food is a white thickish paste made out of nuts,
roots, fat and law quality meat such as rat or pigeon. It is eaten with
stale or old bread, and has many different names such as ``bird butter'',
or ``dead man's shoe''.

\subsection{Slaves}
\label{sec:Slave Prices}

The prices for slaves varies largely per region. They are most expensive in
regions that do not actively engage in slavery on a massive scale, and are
generally cheaper in the slaving nations. Basic prices for slaves start at
around 50 shards, and then increase or decreased based on economy and
condition of the slave.

Most slaves are bought for menial labour, so healthy and capable workers are
the most common type of slave being sold. Slaves that are wounded, sick, or
otherwise usually cost less, while exotic and specially trained slaves cost
more.

\begin{table*}
  \captionsetup{labelformat=empty,font={large,bf},position=top}
  \caption{Slave Prices} \label{tbl:Slave Prices}
  \rowcolors{1}{white}{light-grey}
  \begin{tabular}{p{10cm} l}
    Humanoid male slave             &  50 shards \\
    Slavery is common in the region & -10 shards \\
    Slavery is rare in the region   & +20 shards \\
    Slave's race is exotic          & +20 shards \\
    Sick or otherwise impaired      & -20 shards \\
    Expertly skilled                & +10 shards per CL or HD \\
    Attractive female               & +50 shards
  \end{tabular}
\end{table*}

\subsection{Services and Spellcasting}
\label{sec:Services}

The various institutions on Aror offer certain services to anyone who has the
shards to pay for them. Most institutions, churches and orders have divisions
in most major city kingdoms, as well as smaller outposts in smaller baronies
or towns.

\textbf{Church of \nameref{sec:Forun}} usually offers shelter, housing and
lodging for the downtrodden, as well as regular medicine and healing for those
that cannot afford to pay a cleric to treat wounds.

\textbf{First Order} often runs churches devoted to the \nameref{sec:Order},
and their priests offer services as judges to settle disputes or act as
mediators in diplomatic meetings. Their churches and holy sites are often
used as neutral ground by warring factions that seek to reconcile through
negotiation and diplomacy. The \textbf{Second Order} offers their libraries,
scholars, and researchers for anyone that seeks knowledge. While the
\textbf{Third Order} are sent if criminals have to be caught, justice has to
be served, or vile and evil creatures have to be captured or destroyed.

\subsection{Transport}
\label{sec:Transport}

There are a wide variety of different modes of transportation available on
Aror. All large city kingdoms are situated near the sea, and thus have large
ports and shipping enterprises that ferry people and goods all over the
world. Transportation over land is largely done through horses and carriages.

Ever since \hyperref[sec:Dragon Teleporter]{dragon teleporters} were installed
to connect most major city kingdoms with each other, travelling became cheaper
and easier. A ticket for a dragon teleporter costs between \textbf{15 and 20
  shins}.

\twocolumn
\section{The World of Aror}
\label{sec:Aror}

The world of \emph{Everblack} is situated on a small planet that is
called \emph{Aror}. \emph{Aror} rotates around a binary star system we
call \emph{Carinae}. Both suns can be seen on the sky, with one being
slightly smaller than the other.

The planet also has two moons, one of which is named \emph{Lilest} and
the other \emph{Storst}, in honour of the two magi that first
identified most of the celestial bodies in our solar sytem. They have
many different names in different cultures and languages, but the
names \emph{Lilest} and \emph{Storst} are always understood to refer
to the two moons in the night sky.

\emph{Aror} takes roughly twenty-four (24) hours to rotate around its
own axis, while \emph{Lilest} goes around the planet in thirty-two
(32) days, the bigger moon \emph{Storst} takes forty-eight (48) days
to make one trip around \emph{Aror}. \emph{Aror} itself takes 385 days
to travel around the binary star constellation.

\emph{Aror} is not alone in the system. It is in fact the third planet
from the suns. The first, we call \emph{Forun}, and is a molten rock
of magma, insane temperatures and poisonous gases. \emph{Parin} is
seated between \emph{Forun} and \emph{Aror}, and can be seen with the
naked eye on a clear night sky as a blue shimmering light. Then follow
\emph{Ivir} and \emph{Novar} which are believed to be huge planets made
of various gases. Far beyond the reaches of \emph{Novar} is
\emph{Piad} a smaller, rocky planet on the edge of our binary star
system.

\subsection{Time Keeping}

A year on \emph{Aror} takes roughly 385 days. Years are prefixed with an
\emph{Aeon}, which are only changed in case of a world-shattering event
or shift, or in case the year numbers become unwieldy. As of the last
edit of this book, the current Aeon prefix is \emph{MI} which stands for
the Aeon of Midaris. In official records you will also find prefixes for
which calendar is used. For example \emph{S/20/05 MI:2002} denotes the
twentieth day of the fifth month in the year 2002 in the Aeon of
Midaris, as described by the \emph{Storst} calendar.

The last \emph{aeon} is now called ``gamla tiden'' (or the ``old age''
in elvish), and it began almost eighteen thousand years ago (18043 to
be precise). Its prefix is \emph{GT}.

\subsubsection{Old Calendar}
\label{sec:Old Calendar}

If you use the orbit of the bigger moon \emph{Lilest} to create a
calendar, you get twelve (12) months, with thirty-two (32) days
each. A month is then often broken down to four (4) weeks with eight
(8) days each. Although the \emph{Lilest} based calendar was used
preliminary in the norther hemisphere, it has fallen out of favour and
has been replaced with the \emph{Storst} calendar by MI:1000. Many
cultures still use the old calendar, but most cultures of relevance
have made the switch to the new calendar.

\subsubsection{New Calendar}
\label{sec:New Calendar}

A calendar utilising the rotation of the moon \emph{Storst} is favoured
in the southern hemisphere. Since most arcane and scientific studies are
done in the southern hemisphere, it has been the de facto calendar of all
of Aror since a few decades.

It separates the year into eight (8) months, with forty-eight (48) days
each, which is then again subdivided into four months with twelve (12)
days each.

\subsection{A Global World}

The world of \hyperref[sec:Aror]{Aror} has a few major city kingdoms that are
vastly populated centres of society and civilisation. These immense cities are
the major power players in the world, and often have power over the fates of
many million people. Not only in their own realm, but also in the lands,
baronies and smaller kingdoms they project their power onto.

Knowledge about the other city kingdoms is wide spread, and almost all people
have a rough understanding on how big the world is, and what culture lives
where. Although travel is only feasible for the middle and upper classes, word
about foreign lands and the intricate politics of the city kingdoms spreads
easily to every corner of the land.

\subsection{Dragon Teleporter}
\label{sec:Dragon Teleporter}

These large city kingdoms are also connected with each other with large
teleporters, that were originally invented by the dragons to inter connect
places of interest on their continent of \nameref{sec:Draigynus}. A dragon
teleporter has three large pillars, that sharpen to a point at their end. They
are arranged in a circle that arch inward, touching each other at the top, to
form a sort of arch over a small area beneath them. Once activated arcane
energy flows from their stem to the tip, where they form a glowing, floating
ball of energy beneath the arch, that acts as the horizon for the
teleportation magic. Once a living person touches the horizon, they are
instantly transported near one of the pillars of the remote teleporter. A
teleporter that receives a person, can still teleport one other person away at
the same time.

Draconic runes are inscribed into the pillar and help with selecting a target
for the teleportation, and can be used to program new targets into an existing
teleporter. A dragon teleporter requires considerable arcane power to operate,
and can only teleport one person at a time to a another, preprogrammed, dragon
teleporter. It may still receive another visitor at the same time as it sends
another traveller to a distant teleporter.

In \emph{MI:782} one of these dragon teleporters was found in the depths of
the \nameref{sec:Great Divide} by a mining expedition. It was reverse
engineered by arcane scholars and wizards and then installed in all city
kingdoms that could afford to buy and maintain one. Now they are often seated
centrally within the kingdom, and everyone can purchase tickets to be
transported to another city kingdom of their choosing. Prices for tickets
often range from five to twenty \hyperref[sec:Shin]{shins}, depending on the
prices of the city kingdom. For almost all city kingdoms the teleporters are a
net loss commercially, but they are aware that the trade and commerce (and
thus taxes) they help facilitate is invaluable to the economy of the kingdom.

This lead to the large city kingdoms to become more and more interconnected.
It helped spread the clean, high version of \emph{Teranim} to be spoken all
over the world, as a lingua franca was required to facilitate trade and
commerce.

The dragon teleporters were designed to transport living creatures (dragons)
across great distances, and are almost incapable of transporting material
goods. One teleport can transport one person plus their light equipment, or
perhaps ten kilograms of innate material (such as armour, weapons and
clothes) at a time. This design choice was deliberate by the dragons, as they
do not wear or own that much innate objects; and thus have designed their
teleporters to transport themselves safely and efficiently across vast
distances.

Many have feared that the trade by sea or land would become obsolete since
teleporters were introduced. But the dragons did not share their blueprints,
and the knowledge about the dragon teleporters is still actively researched
and reverse engineered to this day. Their design limitation and relatively
high power consumption has made them a bad economical alternative for
transporting goods.

\subsection{Global Trade}
\label{sec:Trade}

All city kingdoms and the majority of baronies mint their own coins, but they
are often only used locally. Although their value is held by precious metals
they contain, they have a strong variance in terms of metal purity, weight and
thus actual value. This made conversion of one local coin or currency to
another difficult. To make global trade and commerce easier, two new
currencies were introduces: the \emph{shin} and the \emph{shard} which are
made out of \hyperref[sec:Everblack]{everblack}.

Many banks offer a service to weigh, assess and convert local coins and
gemstones to shin and shard. And almost all baronies and kingdoms accept shins
and shards as payment method. Large purchases, such as properties or ships, are
mostly done in shards, or perhaps in bars of pure gold. Although crystal
everblack is more easily shattered and destroyed than gold, it is lighter
to carry and it cannot be as easily diluted or forged. Manipulating everblack
takes a highly skilled arcane wielder or scientific scholar, and thus cannot
be forged or manipulated like other ore or metals.

Thus many traders might simply refuse to take foreign coin, or even local
coins, preferring to deal with the more save shins and shards of everblack.
Assessing the purity of large amounts of gold or silver coins can be costly
and time consuming, and thus traders that deal with expensive items might
prefer the safety and convenience of the everblack based currency.

\subsubsection{Shin}
\label{sec:Shin}

Small chips of \hyperref[sec:Everblack]{everblack}, roughly a centimetre in
diameter - also referred to as \emph{shins} - are a common currency accepted
everywhere on the world of Aror. These are often useless for magical
applications due to their small size, but still hold material value. They are
roughly equivalent to one copper coin and are used to make everyday purchases.

\subsubsection{Shard}
\label{sec:Shard}

Bigger sticks of everblack, roughly five centimetres long, one centimetre thick
and perhaps one centimetre wide are called \emph{shards}. They are used in the
global economy of Aror for expensive purchases such as building projects,
property and artefacts. Shards are roughly equivalent to ten coins of gold.

\subsection{Everblack}
\label{sec:Everblack}

There is one thing that is unique to the world of Aror: \emph{Everblack}. It is
a pitch black crystal, resilient as adamantine but harder to mine and work than
the metal.

It is found in small quantities all over the world of \nameref{sec:Aror}, but
especially in deep soil and embedded in the stones of the mountains. Some
places are richer in everblack than others, and entire economies are built
around mining the metal. For example the dwarves \nameref{sec:Kesmar} mine
the \nameref{sec:Cnamh Mountains} for everblack, and then sell most of it to
the other city nations. Another large quantity has been found beneath the
city of \nameref{sec:El-Fayam}.

\subsubsection{Mining}

It can be mined easily, as untreated and unheated it is rather brittle. However
the everblack dust that is whirled up during the process is highly toxic when
inhaled. This makes mining the brittle crystal rather dangerous for all miners
and workers involved. Early symptoms include coughing, temporary blindness,
dizziness and diarrhoea. Prolonged exposure can lead to a bloody cough,
permanent loss of the ability to perceive colours, perceived symptoms of
hypothermia, such as being cold, shivering and blue limbs and lips and an
increased risk of heart failure. Very few miners risk working these mines
voluntarily, and thus either enslaved labour or only work these mines for very
high pay voluntarily.

\subsubsection{Arcane Battery}

Everblack is capable of holding and storing magical power, and is also able to
release it in a controlled matter. This makes everblack invaluable in arcane
and divine research, as well as making arcane machinery and artefacts. Almost
all artefacts, wands and even scrolls have trace amounts of everblack, that
holds the arcane or divine energy required to make these magical devices work
and function. It is capable of holding arcane, divine, psionic and even soul
energy, and is thus highly sought after all around the world.

Charged chunks and pieces of the crystal are embedded in magical weapons and
armour, as well as wands and other divine and arcane artefact.

A charged everblack crystal or charged composite everblack is warm to the touch.
It gets warmer and warmer the more power is stored within it. If it is charged
to its capacity it will begin to glow in a low, and orange light. Once charged
beyond its ability to safely store power it will begin a low droning hum and
vibrate. Everblack that is overcharged will explode, shattering the crystal to
dust and damaging everything and everyone caught in the explosion. However if
an overcharged crystal is left alone it will release excess magical power in
the form of light and warmth until it returns to maximum capacity.

The excess storage capacity of everblack is very high, and even small
everblack shards require almost three times the power that would make them
full to cause an explosion. The explosion of a small shard is barely enough to
damage a normal sized humanoid. Although highly expensive, everblack is
sometimes fashioned into bigger explosive devices with devastating results.

\aren{Making everblack explosives is like making catapult ammunition out of
  platinum.}

\subsubsection{Everblack Ink}

Everblack Ink is made by crushing the crystals, mixing them with water and
boiling the resulted mixture down. It is used in the inscription of magical
scrolls, as well as runes and seals. Everblack ink is poisonous if consumed
directly, and one must be careful to avoid prolonged exposure of everblack
ink as it can be absorbed through the skin.

\subsubsection{Power Dampening}

An area filled with natural or artificial everblack crystal acts as a magical
dampening field. Such areas impose a natural and environmental potential for
spell failures upon everyone who seeks to cast spells within them. The
crystals redirect the magic and absorb it, often nullifying the power.
Devices and places are often fashioned deliberately out of everblack, so that
powerful witches and wielders of the arcane arts can be robbed of their power.
A direct application of this power dampening power of the everblack crystals
are \hyperref[sec:Null Stone]{null stones}.

\subsubsection{Composite Everblack}
\label{sec:Composite Everblack}

The crystal itself can also be hardened to incredible strengths by melting
everblack in a blast furnace to remove impurities, and then adding trace
amounts of carbon and iron. This everblack alloy, known as \emph{composite
everblack}, is then harder and denser than adamantine. This alloy does not lose
the ability to store magic, and can be used to build larger everblack crystals
and structures, as well as golems and everblack weapons.

\begin{35e}{Composite Everblack as a Material}
  Weapons, armour, and shields can be fashioned out of composite everblack.
  Weapons, armour and shields made out of the composite have half more hit
  points than normal, and 50 hit points per inch of thickness as well as
  hardness 30. Composite everblack materials are always costly enough that all
  weapons, armour and shields are always made of masterwork quality.  Only
  weapons, armour and shields normally made of metal can be fashioned from
  composite everblack.

  Light armour made out of composite everblack grant a spell resistance of 14
  while worn, but costs 10.000 gp more, medium armour grant a spell resistance
  of 16 while worn but costs 15.000 gp more, while heavy armour made out of
  composite everblack, grant spell resistance 18 while worn but costs 20,000
  GP more.

  Any shield made out of composite everblack can be called upon to store
  spells that would normally target the wearer of the shield. An composite
  everblack light shield can store up to two spell levels of spells but cost
  2000 gp more, a heavy shield can store up to four levels of spells but costs
  4000 gp more, and a composite everblack tower shield can store up to 6 spell
  levels but costs up to 6000 gp more.

  Bludgeoning weapons which have a heavy steel head (such as maces), can have
  their head made out of \emph{composite everblack}, and are then especially
  effective. Their damage dice they make are then doubled. For example a
  \emph{Heavy Composite Everblack Mace} does \emph{2d8} damage instead of
  \emph{1d8}.

  Any weapon made out of composite everblack can store one spell of spell
  level three or lower within itself. Upon the next successful attack with
  that weapon the spell is released upon the target, as if it were cast on
  the target of the attack.
\end{35e}

\subsubsection{Everblack Golem}
\label{sec:Everblack Golem}

It is possible to construct golems out of composite everblack. These arcane
wonders are expensive to make, and almost indestructible and unparalleled in
their power. They absorb all magical energy directed at them, and can then
release it when they strike their attackers. The knowledge and tools to craft
them are often a closely guarded secret among deepkin and dwarven master
artisans.


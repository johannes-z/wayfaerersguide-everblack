\twocolumn
\section{Aror}

The world of \emph{Everblack} is situated on a small planet that is
called \emph{Aror}. \emph{Aror} rotates around a binary star system we
call \emph{Carinae}. Both suns can be seen on the sky, with one being
slightly smaller than the other.

The planet also has two moons, one of which is named \emph{Strnad} and
the other \emph{Eschbach}, in honour of the two magi that first
identified most of the celestial bodies in our solar sytem. They have
many different names in different cultures and languages, but the
names \emph{Strnad} and \emph{Eschbach} are always understood to refer
to the two moons in the night sky.

\emph{Aror} takes roughly twenty-four (24) hours to rotate around its
own axis, while \emph{Strnad} goes around the planet in thirty-two
(32) days, the bigger moon \emph{Eschbach} takes forty-eight (48) days
to make one trip around \emph{Aror}. \emph{Aror} itself takes 385 days
to travel around the binary star constellation.

\emph{Aror} is not alone in the system. It is in fact the third planet
from the suns. The first, we call \emph{Forun}, and is a molten rock
of magma, insane temperatures and poisonous gases. \emph{Parin} is
seated between \emph{Forun} and \emph{Aror}, and can be seen with the
naked eye on a clear night sky as a blue shimmering light. Then follow
\emph{Ivir} and \emph{Novar} which are believed to be huge planets made
of various gases. Far beyond the reaches of \emph{Novar} is
\emph{Piad} a smaller, rocky planet on the edge of our binary star
system.

\section{Time Keeping}

A year on \emph{Aror} takes roughly 385 days. Years are prefixed with an
\emph{Aeon}, which are only changed in case of a world-shattering event
or shift, or in case the year numbers become unwieldy. As of the last
edit of this book, the current Aeon prefix is \emph{MI} which stands for
the Aeon of Midaris. In official records you will also find prefixes for
which calendar is used. For example \emph{E/20/05 MI:2002} denotes the
twentieth day of the fifth month in the year 2002 in the Aeon of
Midaris, as described by the \emph{Eschbach} calendar.

The last \emph{aeon} is now called ``gamla tiden'' (or the ``old age''
in elvish), and it began almost eighteen thousand years ago (18043 to
be precise). Its prefix is \emph{GT}.

\subsection{Strnad calendar}

If you use the orbit of the bigger moon \emph{Strnad} to create a
calendar, you get twelve (12) months, with thirty-two (32) days
each. A month is then often broken down to four (4) weeks with eight
(8) days each. Although the \emph{Strnad} based calendar was used
preliminary in the norther hemisphere, it has fallen out of favour and
has been replaced with the \emph{Eschbach} calendar by MI:1000. Many
cultures still use the old calendar, but most cultures of relevance
have made the switch to the new calendar.

\subsection{Eschbach calendar}

A calendar utilising the rotation of the moon \emph{Eschbach} is favoured
in the southern hemisphere. Since most arcane and scientific studies are
done in the southern hemisphere, it has been the de facto calendar of all
of Aror since a few decades.

It separates the year into eight (8) months, with forty-eight (48) days
each, which is then again subdivided into four months with twelve (12)
days each.

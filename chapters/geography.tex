\section{Geography of Aror}

The land of Aror encompasses eight great continents, of which all but one are
inhabited by sentient races. The planet has two polar ice caps, as well as lush
rainforests around the equator. Both the northern and southern hemispheres
experience moderate to cold climate depending on the latitude. Latitude 0 runs
through what has been determined to be central island in the middle of the
archipelago known as the \emph{Silver Isles}.

\subsection{Goltir}
\label{sec:Goltir}

The largest and westernmost continent of Aror, named \emph{Goltir} stretches
from the northern polar ice caps beyond the equator. It is separated in two
large landmasses by a mountain range called \emph{Torainn}. Although the
entire continent is called \emph{Goltir}, it is often divided into \emph{North
  Goltir} - everything north of Torainn, and \emph{South Goltir} which
encompasses everything south of Torainn.

North Goltir is the home of the city kingdoms of \nameref{sec:Forsby} and
\nameref{sec:Kesmar} and \nameref{sec:Stenheim}. It also known as the ``land
of lakes'' as its entire landmass is dotted with millions of smaller lakes,
rivers and springs. Mountain ranges stretch from the centre to north east,
reaching into the polar ice cap of the north pole. In the west, the continent
stretches out a land tongue was responsible for allowing the early humanoids
to reach the Silver Isles during the last ice age. Far in the north west the
continent is still connected to its western neighbour of \emph{Iâfandir}
through a vast sea of ice and snow called the \emph{pale march}. To the south
the continent features a vast sand desert that reaches the foot of the
mountain chain \emph{Torainn}.

North Goltir is home to many sentient humanoid as well as beast races. Humans,
elves and dwarves live in small baronies in the moderate climate of the
central regions. They share their land with tribes of orcs, bugbears,
hobgoblin, goblins, ogres, trolls, and gnolls who prefer the desert region of
the south.

\subsubsection{Cnámh Mountains}
\label{sec:Cnamh Mountains}

The Cnámh mountain range resides on the eastern shores of North Goltir. They
run from the frozen sea all the way west inward, along the ice shores of the
northern poles. The highest peak of the mountain is 

South of the mountain range of \emph{Torainn} is \emph{South Goltir}. It is
a huge landmass. From the mountain range southward it is mostly steppe and
grassland but turns into a thick rainforest around the equator. The most
prominent feature is the almost 10000 metre high \emph{Goban} mountain that
peeks forth from the impenetrable rainforest. The mountain is also the source
of the \emph{Al'ahri} river, that runs all the way south through the
rainforest, the southern desert into the sea. The desert and jungle are mostly
inhabited by native and nomadic tribes of elves, as well as certain beast
races such as harpies and gorgons. Gnolls and nomadic humanoid tribes roam the
desert, while the far south hosts the city kingdom of \emph{Avenfjord}.

\subsection{Farlar}
\label{sec:Farlar}

South of Goltir lies the smaller continent of \emph{Farlar}. Known for its
diverse biome and lush mountain ranges, it is now home to mostly giants, giant
races (such as ogres and trolls). The north is covered with thick and lush
forests, which are watered by several large streams and lakes. These lakes
and streams are fed by the fresh rainfall that falls against the mountain range
called \emph{Liaswa}. The mountain range splits the continent in half, and
allows for a more moderate climate in the southern part of the continent. Among
all the continents Farlar is by far the smallest, and once housed the sprawling
city kingdom of \emph{Nen-Hilith} before the giants forcefully took the land as
their own.

\subsection{Draigynus}
\label{sec:Draigynus}

\emph{Draigynus}, is elongated and continent that encircles Farlar to the
south in a half-moon shape. It is close to the south pole, and thus features a
temperate continental climate to the north, and a subarctic climate to the
south. Very few humanoid races live there, however a small colony of dragons
has made this continent their home.

\subsection{Karnak}

West of the land of the dragons is the massive continent of \emph{Karnak}. It
has a vast chain of large islands to the north which are mostly covered in
rain forests. The continent itself features a massive lake in its centre,
called \emph{Mu'ut}. The lake is encased in huge mountain ranges, and mostly
inaccessible by land. The northern, eastern and western parts of the continent
are covered in a vast and untouched rain forest called \emph{Yua'cata}. The
continent is mostly inhabited by a vast amount of small tribes of wood elves,
beast races such as gorgons, harpies and many different and often dangerous
animals such as apes, primates, tigers and jaguars.

\subsection{Arania}
\label{sec:Arania}

To the west of Karnak, and in the south western corner of Aror lies the
continent of \emph{Arania}. It is considered the birth place of the ancestors
of all sentient humanoid races, such as the elves, dwarves and humans. A huge
and vast steppe, dotted with the occasional mountain ranges, deserts and oasis,
it is home to many native and nomadic tribes, wild animals such as elephants,
lions as well as home to three large city kingdoms: \nameref{sec:Fes
  al-Bashir} to the north east, and \nameref{sec:Esmayar} and
\nameref{sec:El-Fayam}.

\subsection{Eilean Mor}
\label{sec:Eilean Mor}

Far to the west, and north of \emph{Arania} lies the continent of \emph{Eilean
  Mor}. It is split in half by a large mountain range called the
\nameref{sec:Great Divide} that runs from all the from the north east down to
the south west.  The continent features subarctic climate in the north, and
temperate climate in the middle and the south. Home to many sentient humanoid
races, but most predominantly elves, humans, halflings, and dwarves, it houses
four of the city kingdoms: \nameref{sec:Norbury} off the coast in the far
north, \nameref{sec:Hraglund} to the east, \nameref{sec:Helmarnock} to the
south east, and \nameref{sec:Tredegar} to the south west.

\subsubsection{Eafiadir}
\label{sec:Eafiadir}

North of Eilean Mor lies a massive dormant volcano called \emph{Eafiadir}. It
is the source of many rivers, and is surrounded by many poisonous sulphur pits
and lakes. Although the area around the volcano is more sparsely populated
there are many smaller baronies and kingdoms on the shores of the north and
north west of the mountain.

\subsection{Iâfandir}
\label{sec:Iafandir}

Far to the north west lies the continent of \emph{Iâfandir}. It is often called
the \emph{savage lands}, because of its harsh subarctic climate, cold winters,
and the many beasts and beast races that call it home. Most of the civilisations
and tribes there live on the shores along the southern bank, as the land becomes
a frozen tundra further north. The home of the \emph{snow elves} and the
\emph{týnríkke}, as well as various beast races such as ogres, trolls and
hobgoblins, it became synonymous with a cold climate, harsh conditions and a
primal, more savage way of living.

Although only populated by smaller tribes, villages and perhaps the occasional
snow elven or hobgoblin city, the continent was the birth place of the youngest
city kingdom on Aror: \emph{Morkan}.

\subsection{Silver Isles}

In the centre of the vast sea that separates all the continent, stretching in
from the land tongue of \emph{North Goltir}, lies the archipelago called the
\emph{Silver Isles}. A huge network of thousands of smaller islands and inlets,
which are mostly covered with rain forests and jungles. It is the native home
of many beasts, native tribes as well as newly settled villages and expeditions
that seek to find what the islands are named after: mineral riches.

\subsection{Marwaid}

\emph{Marwaid}, is a true deity, and represents both abstract and literal
sacrifice. Along with \emph{Nyddwr} and \emph{Forun}, who are often described
as sisters in the old stories, \emph{Marwaid} is among the oldest true deities
worshipped by the humanoid races.

\subsubsection*{Personification}

Like her ``sisters'' she is often depicted as a woman, especially as a mother
who was willing to sacrifice her grown children in an effort to make life
more bearable and easier for everyone. This is often represented as a wailing
mother smothering her dead child who appears to have died in battle. But in
the tribes that still follow the old ways, she is rarely directly personified
but instead worshipped through specially made stone altars.

\subsubsection*{Dogma}

Most followers of \emph{Marwaid} follow the old ways, meaning they live in
small villages and tribes far away from civilisation. In these hostile and
dangerous regions where strife against monstrous races, food shortages, war,
monsters and disease are common; \emph{Marwaid} is said to favour anyone who
is willing to sacrifice themselves for others and the common good. She favours
hunters and farmers that go hungry to feed the children, warriors that hold
their ground to let the weak, young and elderly escape, and those that
sacrifice the now for a better future, for example by stockpiling food, and
use vital resources sparingly to ensure there is enough for future generations.

She is often explained as having an erratic will and often tests her trusted
followers. Those that were forced to sacrifice - for example by losing loved
ones to sickness or famine - often see no pattern or purpose in their own
suffering and then attribute it to \emph{Marwaid}'s fickle and unpredictable
nature.

\subsubsection*{Shrines of Marwaid}

Most druids of the old ways follow her, and build shrines to her worship.
These are often situated in clearings or in the centre of small towns, and are
large painted standing stones adorned with personal belongings that the
followers sacrificed. Druids and priests of Marwaid perform ceremonies where
followers either offer either abstract sacrifices, in form of promises and
pledges, or literal sacrifices, in the form of personal belongings, food, life
stock and - albeit rarely - humanoid sacrifice, to these stone shrines to
Marwaid. These sacrifices are then accepted by the druid or priest on her
behalf, and are then added and standing stone as ornaments and decoration.

This often gives the shrines of Marwaid a rather grim appearance, as they are
adorned with skulls, bones, spoiled food, and perhaps even the remains of
humanoid bodies; alongside with personal affects such as weapons, necklaces,
tools, and even clothing. Those that take from the shrines are said to be
cursed until they sacrifice something of importance to the very shrine they
stole from.

\subsubsection*{Relations}

The goddess of \emph{Marwaid} is often said to be related to the other three
female primordial true deities, \emph{Forun} and \emph{Nyddwr}. Her followers
are well respected in the followers of the old ways, and those living on the
country side. However worship of her, and her followers have waned in the
large city kingdoms were such ritualised sacrifice is rarely required to
ensure a prosperous future. This often gives her followers grounds to attack
the ``city folk'', seeing them as spoiled and having lost their strength that
comes with the struggle and replaced it with comfort and security.

\begin{35e}{Marwaid}
  Marwaid is considered as Chaotic Good or Chaotic Neutral, and her favourite
  weapon are the unarmed attack and the quarterstaff. She's favoured by rangers,
  barbarians, druids and those that have a tormented life, such as slaves.
\end{35e}

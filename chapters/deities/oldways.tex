\subsection{Old Ways}

The \emph{Old Ways} are not a god or deity, but instead a set of believes,
traditions, and stories that represent how the ancient humanoids worshipped
the three major deities of Aror.

\subsubsection*{Three Mothers}

The core of the religion are three female deities that are worshipped as the
\emph{three mothers} of all humanoid races. Each of them represents one stage
of a woman's development. \emph{Forun} represents the young, hopeful, and
beautiful woman that raises her children with warmth, compassion and love. She
represents youth, beauty, fire and fertility. \emph{Marwaid} represents the
ageing mother, that has lost children, and sacrificed everything that she has
for her young. She is seen as the hardened, stern and often embittered woman
that raises her children to withstand the harsh realities of life.
\emph{Marwaid} also represents the strong fighter within each woman, that
would fight to the bitter end to protect her children. \emph{Nyddwr} represents
the old woman, the crone, that offers her immense wisdom to aid her already
fully grown children. She represents the tempered, wise, yet hardy old woman
that survived against all odds, and now shares her wisdom with others.

\subsubsection*{Stories}

\graham{I always had nightmares of being told ``The Eyeless Man of the Cave''
  by my mother.}

\aren{We had the same story, but called it ``The Lone Ilian''. The stories of
  the Old Ways survive in many cultures, and will continue to scar and
  traumatise children for centuries to come.}

Another important part of the Old Ways are the stories. A collection of mystic
stories, parables, and myths that are passed orally from one generation to the
other. They often have several purposes, but most stories are told to children
to explain to them the dangers, beauties, and also the horror stories of the
world. The collection of all these stories and myths is called the
\emph{old prose}.

The stories also contain heroic accounts of heroes of old. Bold tales about
people that ventured forth to slay beasts, save the innocent, become kings, or
return a powerful magical artefact back to save the village from certain doom.
But the stories also contain horror stories that end badly, such as the new
mother that goes to the woods only to be eaten by a werewolf before her
husband can save her.

The most important part of these stories however is to teach the core values of
the Old Ways. The most important being dedication, loyalty and honour to your
own clan or tribe. Each member of a clan should give what he can, and take as
little as he requires. This not only extends to love, life, family but also
to nature, with which every follower of the Old Ways must strive to respect.

Some of these stories also tell of failings, crimes, and their appropriate
punishment. While most of these stories tell of compassion for minor offences,
they lay out brutal punishments for serious crimes and even include the death
penalty or slavery for severe crimes such as murder or rape. The same stories
also tell of just rulers, their heroic behaviour and their favourable
personality that made them beloved by their followers.

However these old stories also lay down a strict social construct that often
portraits women as responsible for the house and family, while the man is
supposed to hunt, fight and protect. These structures are then reinforced by
heroic hero stories - who are more often than not - male, and by the stories
of the \emph{three mothers} as a motherly figure and role models for women.
While heroines and thus precedent for a balanced and fair society exist - such
as the great huntress Eigyr that slew more beasts than she could eat, or the
powerful witch of old Gweneth that saved her village from a terrible plague -
many tribes following the \emph{Old Ways} still see the woman's place with
the family.

\subsubsection*{Incantations}

The Old Ways do not have priests, but druids as spiritual leaders. Also the
\emph{old prose} contains incantations that may be cast by anyone versed well
enough in the old stories and \emph{Ancient Teranim}. These incantations are
divine magic, and quite powerful. However they often require hours of
preparations, several people performing the incantation together, complicated
chants and songs in \emph{Ancient Teranim}, and sometimes even live sacrifices
to the three mothers to succeed. Practising witches and witchers of the old
ways are often the spiritual centre of a clan or community, and tasked with
gathering, teaching and performing these old incantations when required.

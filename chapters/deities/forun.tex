\subsection{Forun}
\label{sec:Forun}

A true deity, \emph{Forun} represents warmth, kindness, fire but also
destruction and rebirth. She represents the warmth in cold places, both
physically and spiritually, as well as the destruction that fire brings and
the ashes it leaves behind that aid the rebirth. Forun thus also represents
the concept that sometimes, something has become so old, unmovable, or even
corrupt that it has to be burned down to allow for something new to grow.

\subsubsection*{Personification}

Many followers and priests personify Forun as a woman with fiery, waving, and
flaming red hair. She is often depicted as a loving and caring mother giving
warmth to her children, as well as the ever burning fire within everyone that
is capable of love and passion alike.

\subsubsection*{Prevailing Dogma}

The main church of Forun builds small temples, centred around a large brazier
that must be lit at all times. Priests and priestess of Forun spend their
lives serving others, offering their divine power to aid healing of the sick
and wounded, as well as offering shelter and warmth to those that have
neither. The church of Forun can be found almost everywhere on Aror, and their
followers as numerous as they are liked and loved by the people.

The church of Forun has entrenched itself as a main source for culture and
tradition in my places of Aror. Forun's holy days are celebrated in places
such as Forsby or Helmarnock. The church celebrates two major holy days: The
Day of the Winter's Flower, and the Day of Candles.

\subsubsection*{Day of Winter's Flower}

\graham{A large bonfire is lit, bottles of Schnapps are shared, and nine months
  later the community has grown.}

The \emph{Day of December Flower} is celebrated on the day of first frost or
snow in the coming winter, and thus varies from region to region. A large
bonfire is built and lit, and people are encouraged to dance and celebrate one
last time before the harshness of winter covers the land. The festivity is
officially over when the bonfire no longer burns, and thus heralds the final
arrival of the cold season.

\subsubsection*{Day of Candles}

The \emph{Day of Candles} is an unspecified day in spring where the community
is encouraged to light torches, candles and oil lamps in their windows. The
day is announced by the priest, and people bring their candles and lamps to be
blessed in a morning mass. Each lamp or candle is supposed to welcome the
spring, as well as remember any family member or friend which has not survived
the recent winter. These lights are then affixed to windows, walls or balconies
for all to see, often completely illuminating the night until the morning.

In Forsby the lights are then hung outside the cliff side houses, and can
then be seen from the bay, illuminating the entire stone wall of the cliff. In
Helmarnock the lights are attached to the bridges connecting the islands,
making the central forum and bridges dance in soothing orange light.

Whereas in Helmarnock they are fixed to the bridge that connects the islands
together. During the night the bridge is the beautifully illuminated, and many
people visit it to pray and remember their day.

\subsubsection*{Relations}

The goddess itself, as well as the main church of Forun are popular all around
the globe, due to their caring and selfless attitude. In many large city
kingdoms the church of Forun has been a mainstay of society and culture for
thousands of years.

\begin{35e}{Forun}
  Forun is considered neutral or chaotic good, and her favoured weapon is the
  unarmed strike.

  There is one education feat available related to Forun:
  \featref{Follower of Forun}.
\end{35e}

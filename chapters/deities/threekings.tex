\subsection{Three Kings}
\label{sec:Three Kings}

The \emph{Three Kings}, are three lesser deities that represent conquest, war
and tyranny. The three kings are personified as humanoid knights clan in
armour and heavily armed that raise their weapons into the sky to cross their
swords at the blade.

\subsubsection*{Dogma}

The kings are either worshipped individually, or together as a pantheon. Each
king represents one aspect of war and tyranny. \emph{Aruim} represents the
conquest of war and the rule through might and power. He values cunning,
fierceness and strategy in war. \emph{Miator} represents the chaos and
unpredictability of skirmishes, and values anyone who shows no mercy towards
their enemy. He favours landslide victories, and spurns their followers to crush
the weak, plunder, pillage and rape. \emph{Karor} represents the tyrannical
rule of the intelligent over the strong, and the strong over the weak. He
favours ruling conquered lands with an iron and tyrannical fist.

Albeit they are often worshipped together within a region, many soldiers and
other followers pick one of the three specifically for worship. Their
followers are earls, kings, tribal rulers and counts who seek to dominate
their enemies by war and submission. And are often worshipped by knights,
soldiers and barbarians that pray to them for strength in battle.

Many of the warring sentient monstrous races follow the Three Kings, as do
humanoids that live and die for battle. Their worship is widespread in
Norbury, and among the hobgoblin, ogres and troll clans of \emph{Iâfandir}.

\subsubsection*{Challenge}

With a specific ritual any worthy follower can challenge one of the three
kings to single combat. This requires the follower to open a portal to the
extra-planar realms the kings inhabit, and face and challenge the king to
single combat on their home plane. This single combat is to the death, and
should the challenger prevail he may assume the role of that king among
the other two. This follows the basic principle that only the strongest,
most worthy should be allowed to rule.

\subsubsection*{Relations}

The \emph{Three Kings} stand in direct conflict with the followers of most
other gods and true deities, however their followers harbour resentment to
anyone who would help the weak, including \emph{Forun}, \emph{Lor} and the
\emph{Order}.

\begin{35e}{Three Kings}
  As a pantheon the Three Kings are considered neutral evil, and their domains
  are war, death, destruction and evil.

  Aruim is considered neutral evil and their favoured weapon is the battle axe.
  Miator is considered chaotic evil and their favoured weapon is the bastard
  sword. Karor is considered lawful evil, and their favoured weapon is the
  war hammer.
\end{35e}

\subsection*{Nyddwr}

Nyddwr is a true deity, and the goddess of fate, history and time. She is the
patron of historians, archivists, and everyone who seeks to interpret the
past, the present and the future. She is considered the most ancient of all
the true deities, and depictions of her date back hundreds of thousands of
years to the earliest human civilisations.

\subsubsection*{Personification}

Many personifications of Nyddwr exist, but the most predominant is that of a
six armed female. Her arms are coloured by dried paint to represent the three
stages of time: the lower arms are coloured black, and stand for the distant,
often horrible past. The middle pair of arms are coloured red to symbolise the
often dangerous present, while the upper pair of arms are coloured white to
represent a bright future.

\begin{figure*}[ht!]
  \centering
  \includegraphics[width=.9\linewidth]{media/nyddwr.png}
  \par
  Temple to Nyddwr in \emph{Forsby}, circa GT:16102
\end{figure*}

\subsubsection*{Prevailing Dogma}

She favours anyone who is interested in analysing and learning from the past
to enact positive change in the present that ultimately lead toward a better
future. Nyddwr also favours people that value history, and those that share
the wisdom learned from it with others. This often puts followers of Nyddwr
in direct conflict with those of \emph{Aria}.

\subsubsection*{Seers}

Priestesses of Nyddwr are called \emph{seers}. Seers only accept female
applicants, and there are always three in one group or temple. Some seers
travel the world, while others attend shrines and temples within large
cities or in secluded places of contemplation. Much like the trinity of
their goddess, each seer represents one aspect of time. They are also
required to carry five holy possessions at all times:

\begin{itemize}[noitemsep]
  \item Either red, black and white powder to use as face and body paint. One
  seer represents the past (black), one the present (red) while the other
  represents the future (white).
  \item A small dagger or knife, used as a tool and weapon to defend themselves
  and others.
  \item Either a black, red and white ceremonial robes a seer has to craft
  herself. This does not bar her from wearing more clothes beneath, if the
  weather demands it.
  \item Ornament necklace that also acts as a divine focus and prayer bead.
  \item Mortar and pestle used to crunch the coloured powder with which the
  seers must mark the people they granting their wisdom.
\end{itemize}

The three seers are required to always remain at each others side. They will
enter a town together and offer their services and wisdom to everyone that
seeks it. It is customary to offer seers of Nyddwr food and shelter in return,
which they will accept in exchange for sharing their wisdom. However seers of
Nyddwr are now allowed to amass wealth.

When performing the ritual of guidance, the prospect must kneel in privacy
before the seers, and then explain his past to the black seer. She will
identify events and emotions that might linger still, barring the prospect
from moving onward in his life. She will give guidance on how to overcome
these unresolved issues of the past. Once she has done so, she will mark the
prospects head with black paint. Then the prospect may ask the red seer about
guidance about current problems and troubles. In consolidation with what she
has heard about the prospects past, she will outline immediate changes the
prospect can affect in his or her life to improve it. She will then mark the
prospect's head with red paint. Last but not least the white seer, often the
most wise and intelligent, will attempt to give the prospect both a reading of
the future, as well as outlining possible goals the prospect should be working
towards. Once the ritual of guidance is complete, she will mark the prospect's
head with white paint.

\subsubsection*{Relations}

The goddess itself, and her followers are well respected among most
civilisations. Most city kingdoms have a temple dedicated to her, and in almost
all it is a major offence to attack wandering seers. Her seers are even welcome
among the more savage tribes of Iâfandir. Among fighters and paladins of
\emph{Lor}, \emph{Order} and even the \emph{Three Kings} it is considered an
honour to escort seers on a pilgrimage to their destination.

\begin{35e}
  Nyddwr is considered Neutral Good, and her favourite weapon is the dagger.
  She is favoured by bards, skalds, scholars, archivists and wizards.
\end{35e}

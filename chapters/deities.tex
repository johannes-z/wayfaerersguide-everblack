\section{Religion on Aror}
\label{sec:Religion}

The deities of Everblack have been divided into two major groups by most of
the divine scholars: Actual deities, and powerful extra-planar or
extra-dimensional creatures that either pose as such, or have enough power to
enforce their will as if they were deities.

\textbf{True deities} are manifestations of abstract ideas and higher
metaphysical concepts, that grant those that follow their ideals with
power. These deities \emph{never} speak to their followers directly, cannot be
``visited'' on some obscure plane, will never visit the world in turn as an
``avatar'', and can also not be reasoned with. And you may either chose to
align yourself with their ideals, question their ideals and power, or you
choose to ignore them completely.

However it is a strange fact of live that those that do align their lives,
ideals, and value systems according to specific dogma that manifests these
abstract ideas and concepts in the world are granted seemingly divine favour
and power. Some become paladins, holy knights with an arcane power to heal,
while others are granted a vast array of powerful arcane spells at their
disposal. Deviate from the true deity's path, and you risk losing that
power. Followers of these gods have collected thousands of years of history,
rituals, teachings and dogmas and often brought them under one roof as a
religious order or church devoted to said true deity. While following the
proven and existing dogma will lead to reward from the true deity, it is never
the \emph{only way} to elicit favour from said deity. Often new
interpretations of the ideals and value systems are discovered to be as
favoured by the true deity as the tried and true older dogma. This often leads
to schisms within religious orders and institutions, as well as new orders
being founded based off the new set of dogma. \emph{Nyddwr}, a goddess of
olden times, is an example of an abstract ideal manifest as a religion.

On the other hand \emph{Aror} is under the influence of \textbf{lesser
  deities}, a few powerful extra dimensional or extra planar entities, that
pose as gods. Most of these are malevolent and often twist their followers
into doing evil. While the true deities seem eternal, these entities are prone
to disappearing, or having their power usurped by other entities. Although
followers of the true gods often look down on the follower of these entities,
the divine power granted by these entities cannot be denied. It often rivals,
or even surpasses the powers granted by true deities. \emph{Aria} is a prime
example of a powerful extra dimensional entity that is worshipped as a god
among many people on Aror.


\section{True Deities}
\label{sec:True Deities}

%% Forun
\subsection*{Forun}

\aren{They worship a flaming rock in the night sky.}

\graham{Does it matter? I'd say no, it's their actions that count.}

A true deity, \emph{Forun} represents warmth, kindness, fire but also
destruction and rebirth. She represents the warmth in cold places, both
physically and spiritually, as well as the destruction that fire brings and
the ashes it leaves behind that aid the rebirth. Forun thus also represents
the concept that sometimes something has become so old, unmovable, or even
corrupt that it has to be burnt down to allow for something new to grow.

\subsubsection*{Personification}

Many followers and priests personify Forun as a woman with fiery, waving, and
flaming red hair. She is often depicted as a loving and caring mother giving
warmth to her children, as well as the ever burning fire within everyone that
is capable of love and passion alike.

\subsubsection*{Prevailing Dogma}

The main church of Forun builds small temples, centred around a large brazier
that must be lit at all times. Priests and priestess of Forun spend their
lives serving others, offering their divine power to aid healing of the sick
and wounded, as well as offering shelter and warmth to those that have
neither. The church of Forun can be found almost everywhere on Aror, and their
followers as numerous as they are liked and loved by the people.

The church of Forun has entrenched itself as a main source for culture and
tradition in my places of Aror. Forun's holy days are holy days in places such
as Forsby or Helmarnock. The church celebrates two major holy days.

\subsubsection*{Day of Winter's Flower}

\graham{A large bonfire is lit, bottles of Schnapps are shared, and nine months
  later the community has become bigger.}

The \emph{Day of December Flower} is celebrated on the day of first frost or
snow in the coming winter, and thus varies from region to region. A large
bonfire is built and lit, and people are encouraged to dance and celebrate one
last time before the harshness of winter covers the land. The festivity is
officially over when the bonfire no longer burns, and thus heralds the final
arrival of the cold season.

\subsubsection*{Day of Candles}

The \emph{Day of Candles} is an unspecified day in spring where the community
is encouraged to light torches, candles and oil lamps in their windows. The
day is announced by the priest, and people bring their candles and lamps to be
blessed in a morning mass. Each lamp or candle is supposed to welcome the
spring, as well as remember any family member or friend which has not survived
the recent winter. These lights are then affixed to windows, walls or balconies
for all to see, often completely illuminating the night until the morning.

In Forsby the lights are then hung outside the cliff side houses, and can
then be seen from the bay, illuminating the entire stone wall of the cliff. In
Helmarnock the lights are attached to the bridges connecting the islands,
making the central forum and bridges dance in soothing orange light.

\subsubsection*{Relations}

The goddess itself, as well as the main church of Forun are popular all around
the globe, due to their caring and selfless attitude. In many large city
kingdoms the church of Forun has been a mainstay of society and culture for
thousands of years.

\begin{35e}
  Forun is considered neutral or chaotic good, and her favoured weapon is the
  unarmed strike.
\end{35e}

%% Nyddwr
\clearpage
\incgraph[
  overlay={\node[black] at ([xshift=-5cm,yshift=+0.2cm]page.south east)
    {\large \textbf{Temple to Nyddwr in \emph{Forsby}, circa GT:2101}};
  }
]{media/nyddwr.png}
\clearpage

\subsection{Nyddwr}
\label{sec:Nyddwr}

Nyddwr is a true deity, and the goddess of fate, history and time. She is the
patron of historians, archivists, and everyone who seeks to interpret the
past, the present and the future. She is considered the most ancient of all
the true deities, and depictions of her date back hundreds of thousands of
years to the earliest human civilisations.

\subsubsection*{Personification}

Many personifications of Nyddwr exist, but the most predominant is that of a
six armed female. Her arms are coloured by dried paint to represent the three
stages of time: the lower arms are coloured black, and stand for the distant,
often horrible past. The middle pair of arms are coloured red to symbolise the
often dangerous present, while the upper pair of arms are coloured white to
represent a bright future.

\subsubsection*{Prevailing Dogma}

She favours anyone who is interested in analysing and learning from the past
to enact positive change in the present that ultimately lead toward a better
future. Nyddwr also favours people that value history, and those that share
the wisdom learned from it with others. This often puts followers of Nyddwr
in direct conflict with those of \emph{Aria}.

\subsubsection*{Seers}

Priestesses of Nyddwr are called \emph{seers}. Seers only accept female
applicants, and there are always three in one group or temple. Some seers
travel the world, while others attend shrines and temples within large
cities or in secluded places of contemplation. Much like the trinity of
their goddess, each seer represents one aspect of time. They are also
required to carry five holy possessions at all times:

\begin{itemize}[noitemsep]
  \item Either red, black and white powder to use as face and body paint. One
  seer represents the past (black), one the present (red) while the other
  represents the future (white).
  \item A small dagger or knife, used as a tool and weapon to defend themselves
  and others.
  \item Either a black, red and white ceremonial robes a seer has to craft
  herself. This does not bar her from wearing more clothes beneath, if the
  weather demands it.
  \item Ornament necklace that also acts as a divine focus and prayer bead.
  \item Mortar and pestle used to crunch the coloured powder with which the
  seers must mark the people they granting their wisdom.
\end{itemize}

The three seers are required to always remain at each others side. They will
enter a town together and offer their services and wisdom to everyone that
seeks it. It is customary to offer seers of Nyddwr food and shelter in return,
which they will accept in exchange for sharing their wisdom. However seers of
Nyddwr are now allowed to amass wealth.

When performing the ritual of guidance, the prospect must kneel in privacy
before the seers, and then explain his past to the black seer. She will
identify events and emotions that might linger still, barring the prospect
from moving onward in his life. She will give guidance on how to overcome
these unresolved issues of the past. Once she has done so, she will mark the
prospects head with black paint. Then the prospect may ask the red seer about
guidance about current problems and troubles. In consolidation with what she
has heard about the prospects past, she will outline immediate changes the
prospect can affect in his or her life to improve it. She will then mark the
prospect's head with red paint. Last but not least the white seer, often the
most wise and intelligent, will attempt to give the prospect both a reading of
the future, as well as outlining possible goals the prospect should be working
towards. Once the ritual of guidance is complete, she will mark the prospect's
head with white paint.

\subsubsection*{Relations}

The goddess itself, and her followers are well respected among most
civilisations. Most city kingdoms have a temple dedicated to her, and in almost
all it is a major offence to attack wandering seers. Her seers are even welcome
among the more savage tribes of Iâfandir. Among fighters and paladins of
\emph{Lor}, \emph{Order} and even the \emph{Three Kings} it is considered an
honour to escort seers on a pilgrimage to their destination.

\begin{35e}{Nyddwr}
  Nyddwr is considered Neutral Good, and her favourite weapon is the dagger.
  She is favoured by bards, skalds, scholars, archivists and wizards.
\end{35e}

%% Marwaid
\subsection{Marwaid}
\label{sec:Marwaid}

\emph{Marwaid}, is a true deity, and represents both abstract and literal
sacrifice. Along with \emph{Nyddwr} and \emph{Forun}, who are often described
as sisters in the old stories, \emph{Marwaid} is among the oldest true deities
worshipped by the humanoid races.

\subsubsection*{Personification}

Like her ``sisters'' she is often depicted as a woman, especially as a mother
who was willing to sacrifice her grown children in an effort to make life
more bearable and easier for everyone. This is often represented as a wailing
mother smothering her dead child who appears to have died in battle. But in
the tribes that still follow the old ways, she is rarely directly personified
but instead worshipped through specially made stone altars.

\subsubsection*{Dogma}

Most followers of \emph{Marwaid} follow the old ways, meaning they live in
small villages and tribes far away from civilisation. In these hostile and
dangerous regions where strife against monstrous races, food shortages, war,
monsters and disease are common; \emph{Marwaid} is said to favour anyone who
is willing to sacrifice themselves for others and the common good. She favours
hunters and farmers that go hungry to feed the children, warriors that hold
their ground to let the weak, young and elderly escape, and those that
sacrifice the now for a better future, for example by stockpiling food, and
use vital resources sparingly to ensure there is enough for future generations.

She is often explained as having an erratic will and often tests her trusted
followers. Those that were forced to sacrifice - for example by losing loved
ones to sickness or famine - often see no pattern or purpose in their own
suffering and then attribute it to \emph{Marwaid}'s fickle and unpredictable
nature.

\subsubsection*{Shrines of Marwaid}

Most druids of the old ways follow her, and build shrines to her worship.
These are often situated in clearings or in the centre of small towns, and are
large painted standing stones adorned with personal belongings that the
followers sacrificed. Druids and priests of Marwaid perform ceremonies where
followers either offer either abstract sacrifices, in form of promises and
pledges, or literal sacrifices, in the form of personal belongings, food, life
stock and - albeit rarely - humanoid sacrifice, to these stone shrines to
Marwaid. These sacrifices are then accepted by the druid or priest on her
behalf, and are then added and standing stone as ornaments and decoration.

This often gives the shrines of Marwaid a rather grim appearance, as they are
adorned with skulls, bones, spoiled food, and perhaps even the remains of
humanoid bodies; alongside with personal affects such as weapons, necklaces,
tools, and even clothing. Those that take from the shrines are said to be
cursed until they sacrifice something of importance to the very shrine they
stole from.

\subsubsection*{Relations}

The goddess of \emph{Marwaid} is often said to be related to the other three
female primordial true deities, \emph{Forun} and \emph{Nyddwr}. Her followers
are well respected in the followers of the old ways, and those living on the
country side. However worship of her, and her followers have waned in the
large city kingdoms were such ritualised sacrifice is rarely required to
ensure a prosperous future. This often gives her followers grounds to attack
the ``city folk'', seeing them as spoiled and having lost their strength that
comes with the struggle and replaced it with comfort and security.

\begin{35e}{Marwaid}
  Marwaid is considered as Chaotic Good or Chaotic Neutral, and her favourite
  weapon are the unarmed attack and the quarterstaff. She's favoured by rangers,
  barbarians, druids and those that have a tormented life, such as slaves.
\end{35e}

%% Old Ways
\subsection{Old Ways}

The \emph{Old Ways} are not a god or deity, but instead a set of believes,
traditions, and stories that represent how the ancient humanoids worshipped
the three major deities of Aror.

\subsubsection*{Three Mothers}

The core of the religion are three female deities that are worshipped as the
\emph{three mothers} of all humanoid races. Each of them represents one stage
of a woman's development. \emph{Forun} represents the young, hopeful, and
beautiful woman that raises her children with warmth, compassion and love. She
represents youth, beauty, fire and fertility. \emph{Marwaid} represents the
ageing mother, that has lost children, and sacrificed everything that she has
for her young. She is seen as the hardened, stern and often embittered woman
that raises her children to withstand the harsh realities of life.
\emph{Marwaid} also represents the strong fighter within each woman, that
would fight to the bitter end to protect her children. \emph{Nyddwr} represents
the old woman, the crone, that offers her immense wisdom to aid her already
fully grown children. She represents the tempered, wise, yet hardy old woman
that survived against all odds, and now shares her wisdom with others.

\subsubsection*{Stories}

\graham{I always had nightmares of being told ``The Eyeless Man of the Cave''
  by my mother.}

\aren{We had the same story, but called it ``The Lone Ilian''. The stories of
  the Old Ways survive in many cultures, and will continue to scar and
  traumatise children for centuries to come.}

Another important part of the Old Ways are the stories. A collection of mystic
stories, parables, and myths that are passed orally from one generation to the
other. They often have several purposes, but most stories are told to children
to explain to them the dangers, beauties, and also the horror stories of the
world. The collection of all these stories and myths is called the
\emph{old prose}.

The stories also contain heroic accounts of heroes of old. Bold tales about
people that ventured forth to slay beasts, save the innocent, become kings, or
return a powerful magical artefact back to save the village from certain doom.
But the stories also contain horror stories that end badly, such as the new
mother that goes to the woods only to be eaten by a werewolf before her
husband can save her.

The most important part of these stories however is to teach the core values of
the Old Ways. The most important being dedication, loyalty and honour to your
own clan or tribe. Each member of a clan should give what he can, and take as
little as he requires. This not only extends to love, life, family but also
to nature, with which every follower of the Old Ways must strive to respect.

Some of these stories also tell of failings, crimes, and their appropriate
punishment. While most of these stories tell of compassion for minor offences,
they lay out brutal punishments for serious crimes and even include the death
penalty or slavery for severe crimes such as murder or rape. The same stories
also tell of just rulers, their heroic behaviour and their favourable
personality that made them beloved by their people

\subsubsection*{Incantations}

The Old Ways do not have priests per se, but instead the \emph{old prose} also
contains incantations that may be cast by anyone versed well enough in the
old stories. These incantations are divine magic, and quite powerful. However
they often require hours of preparations, several people, complicated chats in
\emph{Ancient Teranim}, and sometimes even live sacrifices to the three
mothers to succeed. Practised witches and witchers of the old ways are often
the spiritual centre of a clan or community, and tasked to gather, teach and
perform these old incantations when required.

%% Order
\subsection{The Order Above Chaos}

The \emph{Order Above Chaos} or simply the \emph{Order} is a true deity, that
represents order and justice, regardless of prevailing law. It is seen as the
\emph{higher order of all things} intrinsic and true to all societies. The
fundamental laws and rules that govern all species, from which no one can
escape, and that which thrones above everyone and everything, even kings.

\subsubsection*{Personification}

The Order is often depicted as a council of hooded humanoid creatures engaged
in a debate or council. More often than not the Order is not personified at
all, and instead represented by a dagger pointing downward to form a cross.

\subsubsection*{Prevailing Dogma}

The \emph{Order Above Chaos} has two major churches and dogmas: The
\emph{Third Order} on one hand, and the \emph{Holy Order of Aleaste}. Both
follow the basic tenets of the Order Above Chaos, but differ substantially in
the amount they are allowed to interfere.

\subsubsection*{Third Order}

The oldest among the churches of the \emph{Order}, the \emph{Inquisition of
  the Third Order} or simply referred to as the ``Inquisition'' or the
``Third Order'' was founded in in \emph{GT:2810}. It is a knight order that
enacts justice, law and order all over the world. Their knights, paladins and
priests (often called simply called ``inquisitors'') punish the wicked, destroy
evil creatures and charge criminals, and those perceived of wrong doing
regardless on whether the local law makes the perceived wrong doing legal.

Highly unpopular within most city kingdoms, the inquisitors of the Third Order
enact their justice without the aid of the local law enforcement agency, and
often punish people whose actions are made legal by the law. Although the
inquisition is known to side with those that are treated unfairly, their
popularity and support among the general population is limited. Many see it
as a bad omen or a sign of trouble when the Inquisition appears to render
justice. However they are often called to deal with supernatural evil threats,
such as the undead, devils or daemons; as the Third Order has a long standing
history, tradition, knowledge and equipment to handle such threats as they
arise in Aror.

\subsubsection*{Holy Order of Aleaste}

In \emph{MI:890} the high inquisitor of the Third Order, an dark elven woman
named Aleaste of House Eseriel, faced a crisis within the church as many
larger city kingdoms rejected the inquisitors and their justice; often barring
them entry to the land or arresting them outright. As an answer she resigned,
and split the church in two, founding the \emph{Holy Order of Aleaste}.

The Holy Order, instead of sending inquisitors throughout the lands, offers
their knowledge, highly educated and experienced judges and prosecutors to
large kingdoms and baronies as advisers. These clerics and priests then aid
and advise the ruling monarchs on new laws, building a fair justice system and
on how to enforce these new laws in a just manner.

Although tainted by the bad reputation of the Third Order, the Holy Order
gained a lot of influence over the centuries. Especially in smaller baronies
that are plagued with chaos, the highly educated advisers and judges are
welcomed to restore order. Thus many of these smaller baronies and earldoms
rely on the Holy Order to provide a framework for laws and security.

\subsubsection*{Rivalry with Lor}

The followers of the Order are in direct conflict with the followers of the
lesser deity of \emph{Lor}. Their argument is that the entity known as
\emph{Lor} represents chaos disguised as a just crusade, and does nothing to
maintain the order of things in the long term. In return the followers of
\emph{Lor} accuse the followers of the Order to indulge in vengeance instead
of seeking true and fair justice.

\begin{35e}
  The \emph{Order Above Chaos} is considered Lawful Neutral, and their favoured
  weapon is the dagger.
\end{35e}


\section{Lesser Deities}
\label{sec:Lesser Deities}

%% Aria
\subsection{Aria}

\aren{Traitor...}

\emph{Aria} is the lesser deity of secrets, intrigues, knowledge and hidden
things. She is often depicted as a woman clad in back robes in deep thought
or meditation.

\subsubsection*{History}

Aria was once a powerful priestess of the \emph{Silent Queen}, before she
challenged the queen's reign during the conflict against \emph{Griannar}.
During that challenge it is said that she was responsible for killing the
\emph{Silent Queen} and usurping her domain, power and rule over the
extra-planar realm where the Silent Queen resided.

\subsubsection*{Well of Truth}

She openly encourages hiding dangerous knowledge in hidden libraries and
archives. She, and her followers, believe that some knowledge is just too
powerful to be left in the hands of mere mortals. The followers of \emph{Aria}
that go out and seek such knowledge to lock away, are called the \emph{Well of
  Truth}. This knowledge may include powerful artefacts and magical
techniques, that are deemed to dangerous. This puts her followers often in
direct conflict with most \emph{soul magic} practitioners, as well as those
following the \emph{rune master}. Followers of the well do not research
knowledge themselves, but instead see themselves as guards against dangerous
knowledge in the wrong hands. They are known for stealing research from
scientists and wizards, as well as killing researchers so that their secrets
may die with them.

\subsubsection*{The Puppeteer}

Not only does she encourage the gathering of knowledge and information, she
also openly encourages her followers to use said knowledge for personal gain.
Many of her followers are thus often those that seek to control society from
the shadows, while amassing wealth and power in secrecy.

\subsubsection*{Relations}

Aria is in direct conflict with the \emph{Runemaster}, as he gifts powerful
magic and teachings to mortals. She also openly opposes anyone who seeks to
investigate unethical practices such as necromancy. The uncompromising methods
of her followers, such as theft or outright assassination, brings her and her
followers often in direct conflict with the \emph{Order} or the knights of
\emph{Lor}.

\begin{35e}{Aria}
  She considered Lawful Evil and her domains are knowledge, travel, magic and
  trickery. She considered the patron of thieves, wizards, librarians,
  archivists, and researchers. Her favoured weapon is the short sword.
\end{35e}

\begin{35efeats}{Aria's Feats}
  \begin{35efeat}{Adept of Aria}
    \srditem{Description}{You trained as an adept of Aria in your early
      childhood, with a goal of becoming a priestess or holy man of Aria.}
    \srditem{Requirements}{Follower and believer of Aria. You lose the
      benefits of this feat if you fall from Aria's grace.}
    \srditem{Benefits}{You gain a +4 \emph{heritage} bonus to hide and move
      silently and these two skills also become class skills. You can also
      cast \emph{Disguise Self} three times per day as a spell-like ability.}
    \srdeducationfeat
  \end{35efeat}
  \begin{35efeat}{Agent of the Well of Truth}
    \srditem{Description}{You trained as a spy, thief or guardsman of the
      Well of Truth. You stole magical artefacts, knowledge or guarded the
      secret libraries of the well of truth.}
    \srditem{Requirements}{Follower and believer of Aria. You lose the
      benefits of this feat if you fall from Aria's grace.}
    \srditem{Benefits}{You gain a +4 \emph{heritage} bonus to listen and spot
      and these two skills also become class skills. You can also cast
      \emph{See Invisibility} once per day as a spell-like ability.}
    \srdeducationfeat
  \end{35efeat}
\end{35efeats}

%% Griannar
\subsection{Griannar}
\label{sec:Griannar}

\emph{Griannar} was once the lesser deity of light, piety, forgiveness and
repentance, until he was killed by the \nameref{sec:Silent Queen} in
\emph{MI:0}. He was often portrayed as a sunbeam, or as the two suns rising on
the horizon.

\subsubsection{History}

Worship of Griannar stretched back thousands of years, and before his death,
the \emph{Holy Church of His Divine Light} was the dominant religion in most
of the human city kingdoms. His \emph{Holy Church} was the most powerful
institution on \hyperref[sec:Aror]{Aror}, and at its peak, counted tens of
millions of followers. A major source for morality and fundamental values in
most humanoid kingdoms and baronies, the church held vast and unparalleled
influence, and political power. The \emph{Holy Knights of His Divine Light},
the churches military wing, could also rival many armies in terms of manpower,
training and equipment.

\subsubsection{Holy Crusade}
\label{sec:Holy Crusade}

Griannar was always an open rival of the \nameref{sec:Silent Queen}. This
rivalry existed over centuries, and lead to the occasional skirmishes, violent
disputes, and armed clashes between the two religions and their
followers. Over the years the power of the Holy Church grew, and began to
entrench itself deeply in the political landscape of the major city
kingdoms. Since it openly tried to enforce a policy of zero tolerance against
corruption, impurity, debauchery, evil and the undead (regardless on whether
they were evil or not), many noble houses began to secretly support the
followers of the Silent Queen.

The Queen's followers, who where often rich thieves, smugglers, corrupt barons
and nobles, began to fund mercenaries and assassins, to drive the followers of
Griannar of their land, or to assassinate powerful priests and bishops of the
Holy Church. The church retaliated by sending her knights to defend churches,
pilgrims and protect the bishops. As the open engagements between the Queen's
mercenaries and the knights became frequent, more and more noble houses, who
saw the Church as a threat to their power, began funding the Queen's war
campaign.

In \emph{GT:18030} the holy church openly called for a \emph{Holy Crusade}
against the evil that is the Silent Queen and her followers. The declaration
was met by a rival declaration by the \nameref{sec:House deVar}, who openly
supported the Silent Queen in the crusade. This plunged \nameref{sec:Helmarnock}
into war against other city nations that openly supported Griannar, including
\nameref{sec:Hraglund} and \nameref{sec:Forsby}.

The Holy Crusade lasted for almost twenty years, and reached its conclusion
at the decisive siege of \nameref{sec:Hraglund} in \emph{GT:18040}. The
siege lasted three years, and finally ended when the archbishop of Griannar
tried to flee the city in secret. He was betrayed by the nobles of the city,
and delivered to be executed by the high priestess \nameref{sec:Aria} of the
Silent Queen.

\subsubsection{Death}

Scholars still debate Griannar's death to this day, but in \emph{MI:0}, all
priests of Griannar lost their divine power, and their prayers went
unanswered. Alongside him, his rival the \nameref{sec:Silent Queen} disappeared
(or died) as well, giving rise to a new religion surrounding the high priestess
\nameref{sec:Aria}.

\subsubsection{Legacy}

Over the course of many decades the Holy Church of His Divine Light slowly
lost influence and power, until it slid into obscurity. Ruined temples and
churches of the church can still be found all over the world, while the major
sites of worships within the city kingdoms were either demolished or have been
taken over by other faiths.

The death of such a major deity was a major event, and the scholars of
\nameref{sec:Fes al-Bashir} tried for a long time to understand the
intricacies of such a world shattering event. The sad occasion was
immortalised in the dawn of the new aeon of \emph{Midaris} that began with
year zero in the year of Griannar's death.

\begin{35e}{Griannar}
  Griannar was considered lawful-good, his favoured weapon was the longsword.
  Griannar is considered dead, and his followers and priests no longer receive
  divine power.
\end{35e}

%% Isamir
\subsection{Isamir}
\label{sec:Isamir}

\emph{Isamir} is a lesser deity who claims sovereignty over the sea, sea
creatures, storms, rains and the weather. He is often portrayed as a deep
sea dragon that waits and lurks beneath the surface.

\subsubsection{Inua}

The \nameref{sec:Inua} worship him as their patron deity, and it is from
them that he became known to the wider world. The Inua see him as malevolent
deep sea creature that must be appeased with prayer and sacrifice, otherwise
he sends storms and thunder in retaliation. He grants power to those that
worship him, and is said to conjure storms and against those that displease
him.

He preaches respect against the sea, its creatures, and that you should not
defile or pollute his seas, lakes or rivers. His followers should never take
more from the waters than they require, and must not allow buildings that
alter lakes and rivers, such as damns.

Isamir openly encourages the tribes of the Inua to raid and attack foreigners,
as well as their own tribes that stray from the path that he has laid out for
them. Ever since the other city kingdoms have come to the
\nameref{sec:Silver Isles} his worship is threatened by the other religions
of Aror. He especially takes a great dislike against any tribe that would
worship \nameref{sec:Forun}.

The Inua are the only people who build large temples, shrines and places of
worship in his name. These are often hidden deep in the jungles of the
\nameref{sec:Silver Isles}.

\subsubsection{Sailors}

Isamir is also a popular deity to sailors, who pray to him for safe passage
over the sea, lakes and rivers. Sailors that follow Isamir often sacrifice
to him, by throwing provisions or even money overboard to feed the beast
that sleeps beneath the surface. Like with the Inua, he demands that his
lakes, rivers and the sea are respected, and demands that they be not
polluted or even destroyed by damns and lakes.

\subsubsection{Relations}

He openly denounces anyone who would destroy, damage or overfish lakes,
rivers and the sea. And he is a fierce opponent of anyone who follows
\nameref{sec:Forun}.

\begin{35e}{Isamir}
  Isamir is considered \emph{chaotic neutral} or \emph{neutral evil},
  and his favoured weapon is the long spear. His domains are water,
  strength, war and chaos.
\end{35e}

%% Lor
\subsection{Lor}

Lor is a lesser deity of justice, law, discipline, good and the fight against
all evil. He is often depicted as an angelic humanoid creature that kneels
down with his two handed sword buried in the soil.

\subsubsection*{History}

He is among the older lesser deities of Aror, and was always in direct
conflict with evil elements of the world. His worship is well established
all across the world, and many welcome his knights and paladins as a means
to establish law and order, as well as a way to fight evils, such as
daemons and undead.

\subsubsection*{Dogma}

The followers of Lor have one prevailing Dogma, which is strictly controlled
by the Church of Lor. The church, and its leader the reigning patriarch or
matriarch, reside in the city kingdom of \emph{Hraglund}. The church preaches
austerity, monogamy and a strict life that rewards those that help the weak
and wounded. Daily prayers are required, as well as weekly attendance to
masses. The priests of Lor often guide a community of followers either alone,
or with an adept they teach. The followers of Lor are meant to seek out and
destroy evil, even the evil that resides within themselves. He instructs his
followers to repent for their signs and purge evil within them through
flagellation, as well as help and aid others to atone for their signs.

To the followers of Lor undead, devils and fey are evil, and wicked. And Lor
openly encourages seeking out and destroying such evils.

Most followers that dedicate themselves to Lor become priests, knights,
paladins or monks. Lor values dedication, hard work as well as spiritual
strength and resolve.

\subsubsection*{Knight Order of Tavos}

The Order of Tavos is a holy knight order of Lor that is under direct control
of the patriarch or matriarch. Only members of nobility are allowed to join as
knights, while any can join as priests or as adepts. To the order virtue,
sacrifice and personal honour are as important as are devotion to Lor and the
Knight Order. The Knight Order never fights wars directly, but is instead
charged with securing and protecting the churches and followers of Lor from
threats, as well as seeking out and destroying evil creatures.

Tavos was the patriarch that established the order in \emph{GT:7600} and was
known as a divine and holy man whose virtue is a example to any that join
the Order or follow Lor.

\subsubsection*{Holy Order of Sir Ceartas}

The Holy Order of Sir Ceartas is a monastic order. They live mostly in
reclusive temples and monasteries in austerity and poverty. They must not
accumulate wealth, give what they earn to the needy and poor, and must fight
evil anywhere they encounter. Often the Holy Order sends forth their pupils to
learn and experience the wide world, and fight evil to defend the
weak. Knights and priests are often sent alongside a more experienced member
of the order, so the master may teach the pupil. Members of the Holy Order
must remain celibate. Much like the Knight Order of Tavos, they are not
allowed to interfere with local laws, and may not join wars.

Sir Ceartas was a knight that was sickened up with the wealth that the church
had amassed, and thought that mingling with the local governments to vie for
power was a despicable and an act unworthy of Lor. Instead he founded his own
monastic order that barred people from acquiring wealth, and teaches them to
focus on the important dogmas of Lor: helping the weak and fighting evil.

\subsubsection*{Relations}

Lor is in direct conflict with pretty much all major evil or neutral gods and
religions, especially with \emph{Three Kings}, the \emph{Order}, as well as
the three other major gods \emph{Forun}, \emph{Nyddwr} and \emph{Marwaid}. Lor
sees these as heretic and archaic forms of worship, if not outright evil.

\begin{35e}
  Lor is considered Lawful Neutral, and he is the patron of knights, paladins,
  and monks and anyone who seeks to destroy evil. His favoured weapon is the
  two handed sword.
\end{35e}

%% Morana
\subsection{Morana}
\label{sec:Morana}

\emph{Morana}, often called \emph{lady death} or the \emph{great betrayer}, is
the lesser deity of death and the undead. In the old manuscripts she is often
depicted as a black veiled female with pale skin. In newer works of art and
literature, made after the her deceit was discovered, she is often shown as a
blue, translucent female ghost, stealing or shepherding souls.

\subsubsection{Great Betrayer}
\label{sec:Great Betrayer}

Originally she was believed to be a greater deity of death, and thus was often
seen as the fourth sister to \nameref{sec:Forun}, \nameref{sec:Marwaid} and
\nameref{sec:Nyddwr}. She was seen as the last stage of motherhood: the old
woman that died, but still holds her protective hand over her children from
the afterlife. Welcoming, and beckoning her children to her side once their
time had come. She was thus a major part of the \nameref{sec:Old Ways} once,
before she was almost unilaterally rejected, and now holds the role of a
villain in the religion.

During the \nameref{sec:Aeon of Strife} some of her followers prayed to her to
give them strength against the monsters and monstrous races that threatened
the humanoid races. As an answer she revealed to the early humanoids the
knowledge on how to create \nameref{sec:Vampires}. She did so secretly and
indirectly, as to not arouse suspicion that she was not in fact a greater
deity.

Her followers were promised great power, strength and eternal life, but Morana
did not reveal the many drawbacks and sacrifices that came with the undead
life. Many of her followers accepted her gift. Upon realising that their new
form was savage, evil and animalistic in nature, and a danger to the very
humanoids they sought to protect, the majority of her followers turned away
from her. This angered Morana greatly, and in retaliation she openly
threatened the vampires and high priests with death should they abandon her
faith.

Since higher deities do not speak to their followers, as they are abstract
concepts and not extra planar entities that live and breathe, her deception
was brought to the light. This deception and betrayal angered almost all of
her followers who in turn abandoned her. Morana however made good on her
threat, and smote and killed most of her undead followers and arch priests.

This betrayal was never forgotten and she's now simply known as the
\emph{great betrayer} among most of the humanoid species. Her name is never
spoken directly, as it is considered too much honour for a being so petty,
deceitful and evil. Her remaining followers call her either \emph{lady Death}
or \emph{Morana}.

\subsubsection{Followers}

Most baronies and city kingdoms ban her worship. The duties of burying the
dead have been taken over by the church of \nameref{sec:Forun} or the church
of \nameref{sec:Lor}. Although she has followers among the \nameref{sec:Inua},
as well as less civilised undead, such as feral vampires or liches, her worship
is rejected among the civilised undead such as \nameref{sec:Vampires} and
\nameref{sec:Umgeher}.

\subsubsection{Teachings}

Morana's modern teachings openly encourages the creation and spreading of
evil undead, and she often helps powerful necromancers to achieve lichdom.
She also welcomes anyone who wishes to practise necromancy, and sees all
undead as her children, even if they reject her.

\subsubsection{Relations}

She now holds poor relations with most other deities and their followers,
especially the true deities that are still worshipped in the
\nameref{sec:Old Ways}. Her direct enemy is \nameref{sec:Lor}, who openly
opposes her followers for creating and summoning undead.

Morana and \nameref{sec:Isamir} appear to be on good terms, as they are both
worshipped together among the \nameref{sec:Inua} of the \nameref{sec:Silver
  Isles}.

\begin{35e}{Morana}
  She is considered \emph{neutral evil}, and her favoured weapon is the
  morning star. Her domains are evil, death, destruction and knowledge.
\end{35e}

%% Runemaster
\subsection*{Runemaster}

The \emph{Runemaster} is an unnamed, extraordinarily powerful devil from the
hellish planar realms, that entices mortals with promises of great power and
knowledge in exchange for sacrifices.

\subsubsection*{Dogma}

The \emph{Runemaster} has no church or concrete following, but his minions can
be summoned through arcane means and will strike deals with mortals. In these
dealings his followers often accept living humanoid sacrifices, powerful
artefacts, or knowledge in exchange for arcane knowledge and power.

\subsubsection*{Runes}

He gained his name from offering a special variant of arcane magic to those
that summon his followers: \emph{rune magic}. With this technique he enables
those that cannot cast magic themselves an easy way to inscribe powerful
arcane runes into the flesh and skin of their own bodies. The magic
incantations to inscribe these runes often require gruesome ingredients, such
as body parts of animals or even humanoids to cast. Although the runes of
lower power require just animal organs, the ingredients become more gruesome as
they grow in power; ultimately peaking at ritual humanoid sacrifice in the
name of the Runemaster.

Most that practise rune magic stop at the lower tiers, never going beyond what
society or standing laws would allow them to acquire in terms of
ingredients. Some however are corrupted by the lust for power, are desperate
or perhaps morally challenged to begin with, and then proceed to fulfil the
Runemaster's ultimate goal: humanoid sacrifices in his name.

\subsubsection*{Runic Lexicon}

He sometimes chooses one of his most loyal follower to write a book about how
to craft these runes, which is simply called the \emph{Runic Lexicon}, in the
hope that this spreads his influence among the mortal races. These books are
priced artefacts, as very few of these actually exist at any given time. Many
are destroyed by the various enemies of the \emph{Runemaster}, and those that
do exist are often shared and traded in secrecy.

\subsubsection*{Relations}

The \emph{Runemaster} is in direct conflict with most good deities that seek
to reign in his powers, and wish to stop his followers from becoming so corrupt
that they'd perform humanoid sacrifice. His enemies include \emph{Forun},
\emph{Lor}, \emph{Order} but especially \emph{Aria}, as she sees everything he
does as an affront to her teachings.

\begin{35e}
  The \emph{Runemaster} is, as most of the devils, \emph{lawful evil} and his
  favoured weapon is a ritual knife (a dagger) that is used to inscribe the
  runes.

  Rune magic is described later in this book.
\end{35e}

%% Silent Queen
\subsection{Silent Queen}
\label{sec:Silent Queen}

The \emph{Silent Queen} was a powerful lesser deity of the night, shadows,
theft, magic and subterfuge, until she was killed by her high priestess
\nameref{sec:Aria} in \emph{MI:0}. She was often depicted as a hooded woman who
had her mouth sewn shut.

\subsubsection{History}

She was once the patron of all people that worked in the shadows, such as
thieves, spies, smugglers and assassins, and those that practised their
forbidden arcane arts in secrets such as necromancers. She was never openly
worshipped, but did find a few large congregations among the ancient
\hyperref[sec:Dark Elves]{dark elves} and \hyperref[sec:Deepkin]{deepkin}.

\subsubsection{Holy Crusade}

As the goddess of night she was always the opposite of \nameref{sec:Griannar},
and the two religions often clashed in violent disputes and skirmishes. These
clashes escalated to a full fledged war, which the followers of Griannar simply
called the \nameref{sec:Holy Crusade}. Although the queen emerged victories
from the war, she was ultimately slain and usurped by her then high priestess
\nameref{sec:Aria}.

\subsubsection{Worship}

Not much is known about her worship any more, since the followers of Aria are
actively engaged in hoarding and destroying any knowledge of the Queen's
teachings and religious practises. Especially the \nameref{sec:Well of Truth}
is actively hunting and collecting ancient tomes of the Queen, but it is
unknown whether the tomes are kept or destroyed.

Although knowledge of her existence is widespread among scholars, she has
drifted into obscurity amidst the general populace. Her rites, teachings
and knowledge is all but lost. All that remains of her are the occasional
statues in honour of the Silent Queen in the caverns of the deep.

\begin{35e}{Silent Queen}
  She was considered \emph{chaotic evil}, but since her death, her follower
  and priests no longer receive divine power.
\end{35e}

%% Three Kings
\subsection{Three Kings}
\label{sec:Three Kings}

The \emph{Three Kings}, are three lesser deities that represent conquest, war
and tyranny. The three kings are personified as humanoid knights clan in
armour and heavily armed that raise their weapons into the sky to cross their
swords at the blade.

\subsubsection*{Dogma}

The kings are either worshipped individually, or together as a pantheon. Each
king represents one aspect of war and tyranny. \emph{Aruim} represents the
conquest of war and the rule through might and power. He values cunning,
fierceness and strategy in war. \emph{Miator} represents the chaos and
unpredictability of skirmishes, and values anyone who shows no mercy towards
their enemy. He favours landslide victories, and spurns their followers to crush
the weak, plunder, pillage and rape. \emph{Karor} represents the tyrannical
rule of the intelligent over the strong, and the strong over the weak. He
favours ruling conquered lands with an iron and tyrannical fist.

Albeit they are often worshipped together within a region, many soldiers and
other followers pick one of the three specifically for worship. Their
followers are earls, kings, tribal rulers and counts who seek to dominate
their enemies by war and submission. And are often worshipped by knights,
soldiers and barbarians that pray to them for strength in battle.

Many of the warring sentient monstrous races follow the Three Kings, as do
humanoids that live and die for battle. Their worship is widespread in
Norbury, and among the hobgoblin, ogres and troll clans of \emph{Iâfandir}.

\subsubsection*{Challenge}

With a specific ritual any worthy follower can challenge one of the three
kings to single combat. This requires the follower to open a portal to the
extra-planar realms the kings inhabit, and face and challenge the king to
single combat on their home plane. This single combat is to the death, and
should the challenger prevail he may assume the role of that king among
the other two. This follows the basic principle that only the strongest,
most worthy should be allowed to rule.

\subsubsection*{Relations}

The \emph{Three Kings} stand in direct conflict with the followers of most
other gods and true deities, however their followers harbour resentment to
anyone who would help the weak, including \emph{Forun}, \emph{Lor} and the
\emph{Order}.

\begin{35e}{Three Kings}
  As a pantheon the Three Kings are considered neutral evil, and their domains
  are war, death, destruction and evil.

  Aruim is considered neutral evil and their favoured weapon is the battle axe.
  Miator is considered chaotic evil and their favoured weapon is the bastard
  sword. Karor is considered lawful evil, and their favoured weapon is the
  war hammer.
\end{35e}

